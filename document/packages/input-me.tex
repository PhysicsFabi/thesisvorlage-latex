%\usepackage{layouts}
\usepackage[utf8]{inputenc}                     % for using utf8 chars (äöüß)
\usepackage[T1]{fontenc}                        % for correct font encoding (you
                                                % can't search for non-ascii 
                                                % chars in pdf otherwise)
\usepackage{lmodern}                            % the font used by T1 font enc
\usepackage[protrusion=true,                    % microtype package for better 
            expansion,                          % line breaks, kerning, etc.
            tracking,
            kerning,
            spacing]{microtype}
\microtypecontext{spacing=nonfrench}            % option for scrbook to allow 
                                                % microtype->nonfrenchspacing
                                                % for info on nonfrenchspacing:
                                                % en.wikipedia.org/wiki/       \
                                                % History_of_sentence_spacing# \
                                                % French_and_English_spacing

\usepackage[english, french, ngerman]{babel}             % babel language styles. you may
                                                % add languages, if need to be.
%\usepackage[fixlanguage]{babelbib}              % use babel bibtex. fixlanguage 
                                                % = don't switch language
                                                % between bib items
%\usepackage{cite}                              % for bibtex without babel
\usepackage[backend=biber,style=authoryear]{biblatex}
%\addbibresource{lit/lit.bib}
%\addbibresource{chapter/neutrinos/lit.bib}
%\addbibresource{chapter/katrin/lit.bib}
%\addbibresource{chapter/energyDependentCrossSec/lit.bib}
    
\usepackage{amsmath}                            % lots of math packages
\usepackage{amssymb}                            % ... even more math packages
%\usepackage{mathtools}                         % ... (not necessary package)
%\usepackage{bbm}                               % blackboard style fonts.
                                                % You may prefer the provided 
                                                % \mathbbm{R} to ordinary 
                                                % \mathbb{R}.
                                                
\usepackage{upgreek}                            % use $\uppi$ etc.
\usepackage{nicefrac}                           % use \nicefrac{1}{2} 
                                                % for inline text, if you want
\usepackage[load-configurations=abbreviations]
{siunitx}                                       % package for units. Use 
                                                % \si{\ampere} or 
                                                % \SI{0.3}{\angstrom}.
\sisetup{per-mode=fraction}                     % style of writing units
\sisetup{separate-uncertainty=true}             % to allow for $\pm...$ as un-
                                                % certainty instead of 0.34(12)
\usepackage[version=4]{mhchem}                  % to nicely write chemcial 
                                                % elements (e.g.
                                                % \ce{^{227}_{90}Th+})

\usepackage{vmargin}                            % Adjust margins in a simple way
                                                % (is used below).
\usepackage{fancyhdr}                           % Define simple page headings 
                                                % (is used below).
\usepackage{placeins}                           % To use \FloatBarrier

\usepackage[nottoc,                             % So that the Bibliography will 
            notlot,                             % be added to the ToC
            notlof]{tocbibind}

\usepackage{varioref}                           % Intelligent page 
                                                % references. Provides \vref 
                                                % and \vpageref

\usepackage{booktabs}
\usepackage[table]{xcolor}                      % used for some coloring 
                                                % commands. table option for 
                                                % beautiful colored tables.
\definecolor{kitcolor}{rgb}{0 0.61 0.50}        % color is defined here and 
                                                % later used for several 
                                                % definitions

\usepackage[pdftex]{graphicx}                   % including images

\usepackage[normal,
                font={small,color=black},
                labelfont=bf,
                margin=2em]{caption}            % create graphics/
                                                % tabular captions
\usepackage{subcaption}                         % create subcaptions
                                                % for subpictures

\usepackage[absolute,overlay]{textpos}          % used for titlepage
\usepackage{tikz}                               % used for titlepage, 
                                                % but may be useful for other 
                                                % stuff, too.
\usepackage{ifthen}                             % for some control 
                                                % sequences in this class
\usepackage[fit,breakall]{truncate}             % used for creating 
                                                % textfields on titlepage
\usepackage{etoolbox}                           % necessary to update page 
                                                % numbering for chapters
\usepackage{xstring}                            % used for creating substring 
                                                % out of \appendixname
\usepackage{multicol}                           % used for titlepage. Usefull 
                                                % for other stuff, too.

\DeclareGraphicsRule{*}{mps}{*}{}               % enables compiling 
                                                % images with pdf-latex
\DeclareGraphicsExtensions{.eps,                % different image exts used
                           .pdf,
                           .png,
                           .jpg,
                           .jpeg,
                           .mps}
\usepackage{epstopdf}                           % may be necessary for miktex to
                                                % convert eps to pdf figures

\usepackage[raiselinks=true,                    % package enables hyperlinks in
            bookmarks=true,                     % document. hyperref also incl.
            bookmarksopenlevel=1,               % url package for \url{...}
            bookmarksopen=true,
            bookmarksnumbered=true,
            hyperindex=true,
            plainpages=false,
            pdfpagelabels=true,
            pdfborder={0 0 0.5},
            colorlinks=false,
            linkbordercolor=kitcolor,
            citebordercolor=kitcolor]{hyperref}

\RequirePackage{scrlfile}                       % Prevent svg (<2.0) package 
\PreventPackageFromLoading{subfig}              % from loading subfig.
                                                % See also iss. #13
\usepackage{svg}                                % include svg graphics
\usepackage[acronyms,nonumberlist]{glossaries}
\makeglossaries

\usepackage[shortlabels]{enumitem}
\usepackage{makecell}
\usepackage{slashed}
\usepackage{mathtools}
\usepackage{xargs}
\usepackage{layouts}