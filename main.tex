\documentclass{document/thesisclass}
%% -------------------------
%% |    Thesis Settings    |
%% -------------------------
% english or ngerman (new german für neue deutsche Rechtschreibung statt german)
\SelectLanguage{english}
% details on this thesis
\newcommand{\thesisauthor}{Fabian Leven}
\newcommand{\thesistopic}{Analysis of Selected Models for Inelastic Electron-Scattering in the KATRIN Gaseous Tritium Source}
\newcommand{\thesisentopic}{}
\newcommand{\thesislongtopic}{}
\newcommand{\thesisinstitute}{Institute of Experimental Particle Physics (ETP)}
\newcommand{\thesisreviewerone}{Prof. Dr. G. Drexlin}
\newcommand{\thesisrevieweroneinstitute}{Institute of Experimental Particle Physics, KIT}
\newcommand{\thesisreviewertwo}{Dr. K. Valerius}
\newcommand{\thesisreviewertwoinstitute}{Institute for Nuclear Physics, KIT}
\newcommand{\thesisadvisorone}{M. Sc. M. Machatschek}
\newcommand{\thesisadvisoroneinstitute}{\thesisrevieweroneinstitute}

\newcommand{\thesisadvisortwo}{}
\newcommand{\thesistimestart}{July 15th, 2018} % on titlepage
\newcommand{\thesistimeend}{July 14th, 2019} % on titlepage
\newcommand{\thesistimehandin}{14.07.2019} % on second page 'preamble'
\newcommand{\thesispagehead}{\thesisentopic} % page heading
\input{include/commands/misc.tex}
%% ------------------------------------
%% |    Quantum Mechanics and Math    |
%% ------------------------------------
\newcommand{\ket}[1]{\left|#1\right\rangle}           % \ket{X}  ->  |X>
\newcommand{\bra}[1]{\left\langle#1\right|}           % \bra{X}  ->  <X|
\newcommand{\braket}[2]                               % \braket{X}{Y}  ->  <X|Y>
{\left\langle#1 \middle| #2\right\rangle}
\newcommand{\bratenket}[3]                            % \bratenket{X}{Y}{Z}  ->
{\left\langle#1 \middle|\middle| #2 \middle|\middle|  % <X|Y|Z>
#3\right\rangle}
\newcommand{\anglemean}[1]                            % \anglemean{X}  ->  <X>
{\left\langle #1 \right\rangle}                       % \norm{X}  ->  || X ||
\newcommand{\norm}[1]{\left\lVert#1\right\rVert}

\newcommand{\updownarrows}                            % \ket\updownarrows  ->
{\text{\rotatebox[origin=c]{90}{$\rightleftarrows$}}} % |↑↓> (cmt is utf8!)
\newcommand{\downuparrows}                            % \ket\updownarrows  ->
{\text{\rotatebox[origin=c]{270}{$\rightleftarrows$}}}% |↓↑>
\newcommand{\neswarrows}                              % \ket\neswarrows  ->
{\text{\rotatebox[origin=c]{45}{$\rightleftarrows$}}} % |↗↙>
\newcommand{\swnearrows}                              % \ket\swnearrows  ->
{\text{\rotatebox[origin=c]{225}{$\rightleftarrows$}}}% |↙↗>

\newcommand{\cre}{c^\dagger}                          % annihalation operator
\newcommand{\anh}{c^{\vphantom{\dagger}}}             % creation operator
\newcommand{\numb}{n^{\vphantom{\dagger}}}            % number operator

\newcommand{\fullstop}{\text{\,.}}                    % fullstop or comma in
\newcommand{\comma}{\text{\,,}}                       % math mode for use
                                                      % after equations
                                                      
\newcommand{\vect}[1]{\boldsymbol{\mathbf{#1}}}      % bold face
\newcommand{\unitvect}[1]{\hat{\vect{#1}}}          % bold face with hat

\newcommand{\ii}{\mathrm{i}}

\DeclarePairedDelimiter\abs{\lvert}{\rvert}

\newcommand{\Conv}{%
  \mathop{\scalebox{1.5}{\raisebox{-0.2ex}{$\circledast$}}
  }
}

\newcommand{\conv}{\ensuremath{\circledast}}
\newcommandx{\inputpdftex}[2][1=\linewidth]{	    
    \begin{center}
        \def\svgwidth{#1}
        \graphicspath{{#2/}}
        \input{#2/img.pdf_tex}
    \end{center}
}
\newcommand{\aeneutrino}{\bar{\nu}_{e}}
\newcommand{\proton}{p^+}
\newcommand{\neutron}{n}
\newcommand{\positron}{e^+}

\input{include/commands/my-abbreviations.tex}




%% ---------------------
%% |    PDF - Setup    |
%% ---------------------
% This information will appear embed into the PDF file as meta data, but will 
% not be printed anywhere
\hypersetup
{
    pdfauthor={\thesisauthor},
    pdftitle={\thesistopic},
    pdfsubject={\thesistopic},
    pdfkeywords={kit,physik,\thesisauthor}
}
%% --------------------------------------
%% |    Settings for Word Separation    |
%% --------------------------------------
% Help for separation:
% In German package the following hints are additionally available:
% "- = Additional separation
% "| = Suppress ligation and possible separation (e.g. Schaf"|fell)
% "~ = Hyphenation without separation (e.g. bergauf und "~ab)
% "= = Hyphenation with separation before and after
% "" = Separation without a hyphenation (e.g. und/""oder)

% Describe separation hints here:
\hyphenation
{
	KATRIN
    über-nom-me-nen an-ge-ge-be-nen
    %Pro-to-koll-in-stan-zen
    %Ma-na-ge-ment  Netz-werk-ele-men-ten
    %Netz-werk Netz-werk-re-ser-vie-rung
    %Netz-werk-adap-ter Fein-ju-stier-ung
    %Da-ten-strom-spe-zi-fi-ka-tion Pa-ket-rumpf
    %Kon-troll-in-stanz
}

\begin{document}
    \FrontMatter

    \input{document/titlepage}
    \input{document/preamble}

    \begingroup \let\clearpage\relax    % in order to avoid listoffigures and
    \tableofcontents                    % listoftables on new pages
    \listoffigures
    \listoftables
    \endgroup
    \cleardoublepage

    \MainMatter
    
    \newacronym{standardmodel}{SM}{Standard Model of particle physics}
\newacronym{lep}{LEP}{Large Electron Positron Collider}
\chapter{Neutrino Physics}
    
    \section{Neutrinos until the 1960s}
    The rich history of neutrino physics covers more than a century. The following paragraphs outline selected milestones on the way to their today's understanding.
    
    In 1895 Becquerel reported results on experiments with phosphorescent substances, especially uranium salts, on photographic plates \cite{Becquerel:1}. His experiments are marked as the discovery of radioactivity and triggered manifold subsequent investigations.
    
    In 1899 Rutherford published a classification of radioactive decays into $\upalpha$ and $\upbeta$ types according to their penetration strength \cite{Rutherford:1}.
    
    In 1900 Becquerel determined the mass-charge-ratio of $\upbeta$ decay particles and identified them as the electron previously described by Thomson \cite{Becquerel:2} \cite{Thomson:1}.
    
    In 1914 Chadwick measured a continuous electron energy spectrum in the $\upbeta$ decay ($\upbeta$ spectrum) of lead-214 and bismuth-214 \cite{Chadwick:1}.
    
    In 1927 Ellis and Wooster conducted a calorimetric measurement of the $\upbeta$ decay energy of radium and demonstrated that the continuity of the $\upbeta$ spectrum was intrinsic to the decay as opposed to caused by secondary effects as e.g. scattering of the electrons within an atom \cite{Ellis:1}.
    
    In 1930 a $\upbeta$ decay was thought of as a two-body-decay $A \rightarrow B + C$. Assuming relativistic kinematics, especially conservation of energy and momentum, in a two-body-decay the momenta of the daughter particles $B$ and $C$ are solely determined by their masses and the energy constitution of the parent particle $A$. According to Bohr there was no reason to believe that different nuclei of the same element $A$ should have a different energy constitution in a $\upbeta$ decay. Hence, the continuity of the $\upbeta$ spectrum could not be explained \cite{Bohr:1}. As a possible solution Pauli suggested the $\upbeta$ decay to be a three-body-decay and postulated a electrically neutral particle that carries part of the decay energy \cite{Pauli1930}.
    
    In 1934 Fermi developed a quantitative theory of $\upbeta$ decay that could describe the preceding experimental results. It comprises a four-fermion contact interaction respectively a three-body-decay model. It was the first description of the so-called ``weak interaction''. Furthermore, Fermi coined the term ``neutrino'' for the particle postulated by Pauli. \cite{Fermi1934}
    
    Fermi's theory inspired the idea to use the so-called ``inverse $\upbeta$ decay'' or ``neutrino capture'' to detect neutrinos, which in today's nomenclature is written as
    \begin{equation*}
        \aeneutrino + \proton \rightarrow \neutron + \positron \fullstop
    \end{equation*}
    In 1956 Cowan and Reines published results of a corresponding experiment. It was conducted using the sufficiently high neutrino flux of the nuclear reactor of the Savannah River Plant. The neutrinos originating in the reactor passed a tank of water and cadmium chloride triggering the above process. The emerging neutron was captured by the cadmium which emitted a photon in a \SIrange{3}{11}{MeV} range. The emerging positron annihilated with an electron which produced two photons of \SI{0.5}{MeV} each. A coincidence measurement of the corresponding photons enabled discriminating signal and background events. Based on their results they reported the free neutrino's discovery. \cite{Cowan103}
    
    In the same year, 1956, Lee and Yang published an article on parity conservation. Parity conservation implies that a mirrored physical process behaves the same as its non-mirrored counterpart. They pointed out that parity conservation might be violated in weak interactions and suggested several probing methods. \cite{Lee1956}
    
    In 1957 Wu et. al. conducted one of the corresponding probing methods based on $\upbeta$ decay. The parity operation respectively ``the mirroring'' corresponded a change of the magnetic field orientation in the experiment. The results showed that parity is violated. \cite{Wu1957}
    
    In 1958 Goldhaber et al. measured the helicity $H$ of the neutrino. Its helicity is defined as $ H = \unitvect{\sigma} \cdot \unitvect{p}$, where $\unitvect{\sigma}$ is the spin unit vector and $\unitvect{p}$ is the momentum unit vector of the neutrino. The experiment found $H = -1$ which corresponds a maximal parity violation. In other words, only left-handed neutrinos and right-handed antineutrinos interact weakly.~\cite{Goldhaber1958}
    
    In 1962 Danby et. al. reported on a second type of neutrinos. A beam of pions generated at the Alternating Gradient Synchrotron in Brookhaven decayed according to $\pi^{\pm} \rightarrow \mu^{\pm} + \nu / \bar\nu$. The emerging neutrinos penetrated a 13.5-meter iron shield wall and their interaction could be detected in a 10-t aluminum spark chamber. The observed interactions were path-like as opposed to shower-like, which implied the production of muons as opposed to electrons. This was marked as the discovery of the $\mu$ neutrino. \cite{Danby1962}
    
    The attempts to uniformly describe the manifold discoveries in the field of particle physics in a combined theory converged over the course of the second half of the 20th century into the so-called \gls{standardmodel}.
    
    \section{Neutrinos in the Standard Model of Particle Physics}
        \begin{table}[tb]
        \caption[Gauge field content of the Standard Model]{Gauge field content of the Standard Model of particle physics.}
        \centering
        \begin{tabular}{lrl}
        \toprule
        symbol & gauge symmetry & associated charge \\
        \hline
        $g$ & SU(3) & color \\
        $W$ & SU(2) & isospin $\vect{T}=(T_1, T_2, T_3)$ \\
        $B$ & U(1) &  hypercharge $Y$\\
        \bottomrule
        \end{tabular}
        \label{tab:SMGaugeFields}
    \end{table}
    
    
    \begin{table}[bt]
        \caption[Matter field content of the Standard Model]{Matter field content of the Standard Model of particle physics and corresponding properties (color charge omitted).}
        \centering
        \begin{tabular}{llrrlll}
        \toprule
        \makecell[tl]{} & 
        \makecell[tl]{symbol} & 
        \makecell[tl]{spin} & 
        \makecell[tl]{generations} & 
        \makecell[tl]{isospin \\$(\abs{\vect{T}}, T_3)$} & 
        \makecell[tl]{hypercharge \\$Y$} \\
        \hline
        higgs & $\phi=(\phi^+, \phi^0)^T$ & 0 & 1 & $(1/2, \pm 1/2)$ & $\pm1/2$\\
        \hline
        quark & $q=(u_L, d_L)^T$ & $1/2$ & 3 & $(1/2, \pm 1/2)$ & $\pm 1/3$ \\
        antiquark & $\bar{d}_R$ & $1/2$ & 3 & $(0, 0)$ & $+1/6$ \\
                  & $\bar{u}_R$ & $1/2$ & 3 & $(0, 0)$ & $-4/3$ \\
        \hline
        lepton & $l=(\nu_L, l_L)^T$ & $1/2$ & 3 & $(1/2, \pm 1/2)$ & $\mp1/2$ \\
        antilepton & $\bar{l}_R$ & $1/2$ & 3 & $(0, 0)$ & $+2$ \\
        \bottomrule
    \end{tabular}
        \label{tab:SMMatterFields}
    \end{table}
    
    
    \begin{figure}[t]
        \begin{center}
            \def\svgwidth{\linewidth}
            \inputpdftex{chapter/neutrinos/fig/standardmodel}
        \end{center}
    	\xcaption{Standard Model of particle physics}{Standard Model of particle physics.}{The fermions are framed continuous  and the bosons dotted. Among the fermions the quark sector is marked by a thick frame and the lepton sector by a thin one. (Illustration adapted from \cite{SeitzM2019})}
	    \label{fig:standardmodel}
    \end{figure}
    The \glsentrylong{standardmodel} is a well-tested and established theory, which is evident by e.g. the extensive Review of Particle Physics of the Particle Data Group \cite{ReviewOfParticlePhysics}. It is a gauge quantum field theory exhibiting the gauge symmetry $\text{SU}(3)\times\text{SU}(2)\times\text{U}(1)$. Table \ref{tab:SMGaugeFields} contains the \gls{standardmodel}'s gauge fields and their associated charges. The $B$- and $W$-fields mix to the observable $W^+$, $W^-$ and $Z^0$ bosons as well as to the photon $\gamma$. Table \ref{tab:SMMatterFields} contains the \gls{standardmodel}'s matter fields and their properties.

    Experimentally, a ``disturbance'' of a quantum field is often localized in space-time, which makes it practical to speak of the \gls{standardmodel}'s content as particles with intrinsic properties. Figure~\ref{fig:standardmodel} depicts the particles of the \gls{standardmodel} along with their masses and electric charge.
    
    In the \gls{standardmodel} neutrinos are leptons with the following properties:~
    \begin{center}
    \begin{tabular}{ll}
        \toprule
        flavors & 3 $(\nu_e, \nu_\mu, \nu_\tau)$ \\
        electric charge & 0 (neutral) \\
        spin & $\frac{1}{2} \hbar$ (fermion) \\
        helicity & \makecell[lt]{left-handed neutrinos \\ right-handed antineutrinos} \\
        \bottomrule
    \end{tabular}
    \end{center}
    
    \subsection{Flavors}
    A particle's flavor is its eigenstate with respect to the weak interaction. In the 1990s a precision measurement of the width of the $Z^0$-boson resonance at the \gls{lep} yielded a number of light neutrino flavours consistent with 3 \cite{NumberOfNeutrinos}. In this context ``light'' refers to a neutrino mass smaller than half the mass of the $Z^0$ boson. The \gls{standardmodel} incorporates these three flavors, usually labeled $\nu_e$, $\nu_\mu$ and $\nu_\tau$. The last one to be discovered was the $\tau$ neutrino in 2001 by the DONUT collaboration~\cite{Kodama2000}.
    
    \subsection{Mass Generating Mechanisms}
    For a theory to account for neutrino masses its Lagrangian density must exhibit corresponding mass terms. According to \cite{zuber2011neutrino} the formalism can be summarized: The form of a mass term is given by the Dirac equation, which is produced by applying the principle of least action to a suitable Lagrangian density $\mathcal{L}$. The mass terms have to be quadratic in the fermion fields $\psi$ and must leave the Lagrangian density hermitian. Furthermore, a field $\psi$ must have a left- and right-handed component in order for the mass terms not to vanish. Two possible term forms are named after Dirac and Majorana. Whether one or a mixture of both forms correspond the neutrino's reality is an open question.
    
    \paragraph{Dirac Masses}
    A Dirac mass term with mass $m_D$ split in its chiral components (Weyl spinors) $\psi_{L,R}$ has the form
    \begin{equation}
        \mathcal{L}_D =  -m_D \bar\psi\psi = -m_D \left(\bar\psi_L\psi_R + \bar\psi_R\psi_L \right) \fullstop
    \end{equation}
    Applying this to neutrinos it requires both, a left- and a right-handed Dirac neutrino. Right-handed neutrinos have not yet been observed. If they exist, they don't interact weakly and hence, are called sterile.
    \paragraph{Majorana Masses}
    For Majorana mass terms, one uses the CP-conjugate $\psi^C$ of a fermion spinor $\psi$. Note that if $\psi$ is left-handed, $\psi^C$ is right-handed and vise versa. One can then define a Majorana field $\phi$ and construct a corresponding mass term $\mathcal{L}_M$ with a mass $m_M$
    \begin{equation}
        \phi = \psi + \psi^C \qquad \mathcal{L}_M = -\frac{1}{2}m_M \bar\phi \phi \fullstop
    \end{equation}
    As $\phi^C=\phi$, the described Majorana particle is its own antiparticle, which due to charge conservation is only possible for neutral particles, such as a neutrino.
    
    \section{Neutrino Oscillations}
    \subsection{Demonstrative Formalism}
    According to \cite{zuber2011neutrino} a formula demonstrating neutrino oscillations can be derived:
    If the neutrino's mass eigenstates $\ket{\nu_i}$ ($i \in \left\{1, 2, 3\right\}$) of the free Hamiltonian differ from their flavor eigenstates $\ket{\nu_\alpha}$ ($\alpha \in \left\{e,\mu,\tau\right\}$) of the weak interaction, neutrinos oscillate in flavor space. The corresponding matrix for a basis change $U$ is called Pontecorvo-Maki-Nakagawa-Sakata matrix (PMNS matrix)
    \begin{equation}
        \ket{\nu_\alpha} = \sum_{i} U_{\alpha i} \ket{\nu_i}
        \fullstop
    \end{equation}
The PMNS matrix $U$ is commonly parametrized using three angles $\theta_{12}, \theta_{23}, \theta_{13} \in [0,2\pi)$, a CP-violating phase $\delta \in [0,2\pi)$ and two Majorana phases $\alpha, \beta \in [0,2\pi)$:
    \begin{align}
        \label{eq:PMNSmatrix}
        U =  
        &\begin{pmatrix} 
        1 & 0 & 0 \\ 
        0 & \cos\theta_{23} & \sin\theta_{23} \\ 
        0 & -\sin\theta_{23} & \cos\theta_{23} 
        \end{pmatrix}
        \begin{pmatrix} 
        \cos\theta_{13} & 0 & \sin\theta_{13}e^{-\ii\delta} \\ 
        0 & 1 & 0 \\ 
        -\sin\theta_{13}e^{\ii\delta} & 0 & \cos\theta_{13} 
        \end{pmatrix} \\ \notag
        &\begin{pmatrix} 
        \cos\theta_{12} & \sin\theta_{12} & 0 \\ 
        -\sin\theta_{12} & \cos\theta_{12} & 0 \\ 
        0 & 0 & 1 
        \end{pmatrix}
        \begin{pmatrix} 
        1 & 0 & 0 \\ 
        0 & e^{\ii\alpha} & 0 \\ 
        0 & 0 & e^{\ii\beta} 
        \end{pmatrix}
        \fullstop
    \end{align}
    The evolution of a neutrino's flavor state on a 1-dimensional path starting at position $x=0$ at time $t=0$ with momentum $p_i$  and energy $E_i$ of its mass eigenstates $\ket{\nu_i}$ is
    \begin{equation}
        \ket{\nu_\alpha(x,t)} = \sum_{i} U_{\alpha i} e^{-\ii (E_i t-p_ix)} \ket{\nu_i} \fullstop
    \end{equation}
    This leads to the transition amplitudes
    \begin{subequations}
        \label{eq:nuOsciTransAmp}
        \begin{equation}
        A(\alpha \rightarrow \beta)(t) 
        = \braket{\nu_{\beta}}{\nu_{\alpha} (L)} 
        = \sum_i U^*_{\beta i} U_{\alpha i} e^{-\ii (E_i t-p_ix)t}
        \end{equation}
        \begin{equation}
        A(\bar\alpha \rightarrow \bar\beta)(t) 
        = \braket{\bar{\nu}_{\beta}}{\bar{\nu}_{\alpha} (L)} 
        = \sum_i U_{\beta i} U^*_{\alpha i} e^{-\ii (E_i t-p_ix)t}
        \fullstop
        \end{equation}
    \end{subequations}
    If $\delta \neq 0$ ($ \Rightarrow U \neq U^*$), (\ref{eq:nuOsciTransAmp}) implies $CP$-violation.
    
    The following assumptions allow for a demonstrative form of the transition probability:
    \begin{itemize}
        \renewcommand{\labelitemi}{$\bullet$}
        \renewcommand{\labelitemii}{$\circ$}
        \item The neutrinos are relativistic: 
        \begin{itemize}
            \item Their momentum equals approximately their energy which is by far larger than their mass $p_i \approx E_i \gg m_i$. This also implies the energy can be expanded in the mass-momentum-ratio $m_i/p_i$.
            \item They travel the distance $x=L=ct$ at the speed of light $c$.
        \end{itemize}
        \item All neutrino generations have approximately the same momentum $E \approx p \approx p_i$.
        \item The $CP$-violating phase vanishes $\delta=0$.
    \end{itemize}
    Then, the transition probability from one flavor $\alpha$ to another $\beta$ is
    \begin{equation}
        \begin{split}
        P(\alpha \rightarrow \beta)(L) 
        &= \abs{\braket{\nu_\beta}{\nu_\alpha(L)}}^2 \\
        &= 
        \delta_{\alpha\beta}-
        4\sum_{i}\sum_{j>i} U_{\alpha i} U_{\alpha j} U_{\beta i} U_{\beta j} 
        \sin^2\left( \frac{(m_i^2-m_j^2)}{4} \frac{L}{E} \right)
        \fullstop
        \end{split}
        \label{eq:nuOsci}
    \end{equation}
    
    Equation (\ref{eq:nuOsci}) shows oscillatory behavior if the mass of at least two flavors differ and the corresponding off-diagonal elements of the PMNS matrix $U$ are non-vanishing. Furthermore, neutrino oscillation experiments are sensitive to the difference of squared masses 
    \begin{equation}
        \label{eq:massSquaredDiff}
        \Delta m^2_{ij} =  \abs{m^2_i - m^2_j} \fullstop
    \end{equation}
    which only yields two independent observables for three masses. Thus, these experiments can not be used to determine the absolute mass scale of neutrinos.
    
    \subsection{Solar Neutrino Experiments}
    Solar neutrino experiments led to the discovery of neutrino oscillations. In the end of the 1930s Bethe, von Weizecker and Critchfield showed that the so-called pp cycle is the primary source of solar neutrinos \cite{Weiz1938, Bethe38, Bethe39}. Its initial reaction and the one with the broadest neutrino energy spectrum are
    \begin{equation}
        \ce{p} + \ce{p} \rightarrow \ce{^2D} + \ce{e^+} + \nu_e
        \qquad
        \ce{^8B} \rightarrow \ce{^8Be^*} + \ce{e^+} + \nu_e \fullstop
        \label{eq:ppCycle}
    \end{equation}
    Note that only electron neutrinos are produced in the sun. Starting from the 1970s the solar neutrino flux was measured; the first time by the Homestake experiment using the inverse beta decay of \ce{^{37}Cl}. It could detect electron neutrinos with an energy threshold of \SI{813}{keV}. The flux was one third of the prediction by the standard solar model \cite{Cleveland1998}. The experiments GALLEX/GNO and SAGE confirmed the results, where the later could detect electron neutrinos with an energy threshold of \SI{233}{keV}~\cite{Kirsten1998, Altmann2005, Abdurashitov2009}. Starting from 1999 the SNO experiment measured the neutrino flux of all flavors. It used \SI{1000}{t} of heavy water \ce{D2O} to detect electron neutrinos via charged currents as well as all flavors via neutral currents and neutrino-electron scattering. The measured flux of all flavors of the \ce{^8B} neutrinos \eqref{eq:ppCycle} was in accordance with the electron neutrino flux predicted by the standard solar model \cite{Aharmim2013}. Accounting for neutrino oscillations in free space as well as matter mediated (MSW effect \cite{Wolfenstein1977, Mikheev1986}) explains the flux data. All in all, neutrino oscillations are experimentally verified and proof that neutrinos have mass.
    
    \subsection{Mixing Parameters and Mass Ordering}
    \begin{figure}[t]
        \inputpdftex[0.6\linewidth]{chapter/neutrinos/fig/mass-hierarchy}
    	\xcaption{Mixing parameters and mass ordering}{Mixing parameters and mass ordering.}{The chart shows how the mass eigenstates $\nu_i$ are composed of the flavor eigenstates $\nu_\alpha$ in the normal and inverted mass ordering. The composition depends on the phase $\delta$. The mixing is shown for the two extreme cases $\delta = \SI{0}{\degree}$ (baseline) and $\delta = \SI{180}{\degree}$ (topline). Note that the mass splitting is not to scale. (Adapted from \cite{SeitzM2019}. Numerical values can be found in \cite{Esteban2019}.)}
	    \label{fig:mixingParams}
    \end{figure}
    According to (\ref{eq:nuOsci}) the ratio $L/E$ determines the sensitivity of an experiment to the PMNS matrix $U$ (mixing parameters) and to $\Delta m^2_{ij}$ (mass ordering). $L$ can be tuned by placing the detector in a suitable distance from the neutrino source. $E$ can either be tuned by using e.g. particle accelerators as source or if the source exhibits an energy spectrum like e.g. the sun. According to \cite{zuber2011neutrino} there are four major neutrino sources that can be used to measure the mixing parameters and the mass ordering:
    \begin{center}
        \begin{tabular}{lll}
        \toprule
             source & flavors & sensitive to \\
             \hline
             nuclear power plants & $\bar{\nu}_e$ & $\sin\theta_{13}$ \\
             accelerators & $\nu_e$, $\nu_\mu$, $\bar{\nu}_e$, $\bar{\nu}_\mu$ & $\sin\theta_{23}$, $\Delta m^2_{23}$ \\
             atmosphere & $\nu_e$, $\nu_\mu$, $\bar{\nu}_e$, $\bar{\nu}_\mu$ & $\sin\theta_{23}$, $\Delta m^2_{23}$ \\
             Sun & $\nu_{e}$ & $\sin\theta_{12}$, $\Delta m^2_{21}$ \\
        \bottomrule
        \end{tabular}
    \end{center}
    Furthermore, the MSW resonance of solar neutrinos requires $m_1 < m_2$, which allows for two possible mass orderings:
    \begin{enumerate}
        \item normal ordering $m_1 < m_2 < m_3$ and
        \item inverted ordering $m_3 < m_1 < m_2$.
    \end{enumerate}
    A combination of recent experimental results can be found in \cite{Esteban2019}. Figure \ref{fig:mixingParams} illustrates the mixing parameters and the mass ordering. For normal ordering the best fit for the phase is $\delta=\SI{215}{\degree}$, but CP-conservation ($\delta=\SI{180}{\degree}$) can not be ruled out. Furthermore, $\Delta m^2_{21} \approx 7 \times 10^{-5} \SI{}{eV^2}$ and $\Delta m^2_{23} \approx 2 \times 10^{-3} \SI{}{eV^2}$.
    
    \section{Absolute Neutrino Mass Measurements}
    \label{sec:absoluteNuMassMeasurement}
    Measurements of the absolute neutrino mass fall into one of three categories \cite{Otten:2008zz}:
    \begin{itemize}
        \renewcommand{\labelitemi}{$\bullet$}
        \item cosmology,
        \item neutrinoless double $\upbeta$ decay or
        \item kinematic measurements.
    \end{itemize}
    \paragraph{Cosmology}
    In the early Universe neutral particles such as light neutrinos could escape from areas of high density to areas of low density. As they carry away mass, the larger the neutrino mass $m_\nu$ the stronger is the suppression of density fluctuations on small scales. In a mathematical formulation the so called power spectrum of the density contrast is examined. Corresponding data are e.g. obtained by the Sloan Digital Sky Survey (SDSS). This experiment records the sky's electromagnetic spectrum via telescope \cite{Doroshkevich2004}. Furthermore, the temperature anisotropies in the cosmic microwave background (CMB) encode information on the Universe's structure. The latest and most precise data are recorded by the PLANCK satellite \cite{Aghanim:2018}. Under the assumption that all mass states contribute with the same number density cosmological observations are to first order only sensitive to the sum of all neutrino masses $\sum_{i} m_i$. A combination of the above data sets yields \cite{Yeche:2017upn}
    \begin{equation*}
        \sum_i m_i < \SI{0.14}{eV} \quad (\SI{95}{\percent} \text{ C.L.}) \fullstop
    \end{equation*}
    
    \paragraph{Neutrinoless Double $\boldsymbol{\upbeta}$ Decay}
    Double $\upbeta$ decay ($2\nu\upbeta\upbeta$) is described as a nucleus of element $X(Z,A)$ with $Z$ protons and $A-Z$ neutrons that decays to a daughter isotope $Y(Z+2,A)$ via two simultaneous $\upbeta$ decays 
    \begin{equation}
        X(Z,A) \rightarrow Y(Z+2,A) + 2e^- + 2\bar{\nu}_e \fullstop
    \end{equation}
    According to \cite{zuber2011neutrino} if the neutrino is its own antiparticle, the neutrino emitted in the first decay can be absorbed in the second decay resulting in a neutrinoless double decay ($0\nu\upbeta\upbeta$). This would require the neutrino to have mass and be of Majorana type. Such a decay would manifest itself in a peak in the $\upbeta$ spectrum two neutrino masses above the endpoint of the continuum. Note that this would violate lepton number conservation. The half-life of such a decay encodes the Majorana mass of the electron neutrino as a weighted sum of all neutrino masses using the PMNS matrix $U$ \eqref{eq:PMNSmatrix}
    \begin{equation}
        \label{eq:majoranaMass}
        m_{\upbeta\upbeta} = \abs{\sum_i U_{ei}^2 m_i} \fullstop
    \end{equation}
    Note that $U$ contains two unknown Majorana phases that might cause cancellation in (\ref{eq:majoranaMass}). This makes it difficult to compare $m_{\upbeta\upbeta}$ to masses obtained by other methods. The two most stringent upper limits on $m_{\upbeta\upbeta}$ are given as ranges by:
    \begin{center}
    \begin{tabular}{lr}
        \toprule
        experiment & \SI{90}{\percent} C.L. upper limit on $m_{\upbeta\upbeta}$ [eV]\\
        \hline
        GERDA \cite{Agostini2018} &  0.12–0.26 \\
        KamLAND-Zen \cite{Gando2016} & 0.05–0.16 \\
        \bottomrule
    \end{tabular}
    \end{center}
    
    \paragraph{Kinematic Measurements}
    In decay processes the mass of neutrinos manifests itself in their energy spectrum or the energy spectrum of other decay products. With currently achievable energy resolutions the corresponding observable is a weighted sum of the $N$ neutrino eigenmasses where the weights are the elements of the PMNS matrix \eqref{eq:PMNSmatrix} 
    \begin{equation}
    \label{eq:nuMassSquared}
        m^2_{\nu_{\alpha}} = \sum_{i}^{N}\abs{U_{\alpha i}}^2 m_i^2 \fullstop
    \end{equation}
    An experiment with a sufficiently high energy resolution can be sensitive to terms of the above sum \eqref{eq:nuMassSquared}. This enables the search for sterile neutrinos ($N>3$). A summary of kinematic neutrino mass experiments can be found in \cite{Otten:2008zz, SeitzM2019, zuber2011neutrino}, of which a selection is:
    \begin{itemize}
    \renewcommand{\labelitemi}{$\bullet$}
        \item \textbf{Neutrinos from Supernova 1987A}\\ 
        One would expect higher energetic neutrinos to arrive on Earth earlier than lower energetic ones, which was observed by multiple experiments. A corresponding analysis yielded an upper limit for the mass of the electron neutrino \cite{Loredo2002}
        \begin{equation*}
            m_{\nu_e} < \SI{5.7}{eV} \quad (\SI{95}{\percent} \text{ credible interval}) \fullstop 
        \end{equation*}
        \item \textbf{Muon decay}\\ 
        At the Paul Scherrer Institute the decay of pions to muons and muon neutrinos was examined. An analysis of the momenta yielded an upper limit on the muon neutrino mass \cite{Assamagan1996}
        \begin{equation*}
            m_{\nu_\mu} < \SI{17}{keV} \quad (\SI{90}{\percent} \text{ C.L.}) \fullstop 
        \end{equation*}
        \item \textbf{Tauon decay}\\ 
        At the \gls{lep} the decay of tauons to pions and tau neutrinos was examined. An analysis of the momenta yielded an upper limit on the tauon neutrino mass \cite{Barate:1997zg}
        \begin{equation*}
            m_{\nu_\tau} < \SI{18.2}{MeV} \quad (\SI{95}{\percent} \text{ C.L.}) \fullstop 
        \end{equation*}
        \item \textbf{$\boldsymbol{\upbeta}$ decay}\\ 
        In $\upbeta^-$ decay
        \begin{equation}
            X(Z,A) \rightarrow Y(Z+1,A) + e^- + \bar{\nu}_e
        \end{equation}
        part of the released surplus energy generates the neutrino's mass. This leaves a signature in the $\upbeta$ spectrum. The latest two experiments investigated tritium $\upbeta$ decay in Mainz and Troitsk and yielded a combined result of \cite{Kraus2005, Aseev:2011dq, ReviewOfParticlePhysics}
        \begin{equation*}
            m_{\nu_e} < \SI{2}{eV} \quad (\SI{95}{\percent} \text{ C.L.}) \fullstop 
        \end{equation*}
        Note that KATRIN is a successor of these two experiments and aims for an even better precision of $m_{\nu_e} < \SI{0.2}{eV}$ (\SI{90}{\percent} C.L.) \cite{Angrik:2005ep}.
    \end{itemize}
    
    \newacronym{standardmodel}{SM}{Standard Model of particle physics}
\newacronym{lep}{LEP}{Large Electron Positron Collider}
    \def\currentRootFolder{chapter/energyDependentCrossSec}
\def\currentFigureFolder{\currentRootFolder/fig}
\newcommand{\sigmaInel}{\sigma_\mathrm{inel}}
\newcommand{\sigmaAvg}{\sigma_\mathrm{avg}}

\newcommand{\qUmin}{qU_\mathrm{min}}

\newcommand{\Ekin}{E_\mathrm{kin}}
\newcommand{\nSource}{n_\mathrm{S}}
\newacronym{ssc}{SSC}{source and spectrum calculation}

\chapter{Energy-Dependence of the Cross Section for Inelastic Electron Scattering within the \glsentryshort{wgts}}
\label{sec:eDepScatCrossSec}
The probability of an electron to scatter when traveling through the \gls{wgts} can be characterized by the total scattering cross section $\sigma_\mathrm{tot}$. Two types of scatterings can be distinguished: elastic and inelastic scattering. The cross section for elastic scattering is by smaller than the one for inelastic scattering by one order of magnitude~\cite{Kleesiek2019}. This chapter focuses on inelastic scattering and neglects the other. Within this chapter, the cross section for electrons scattering inelastically off tritium molecules is just denoted as ``cross section'' and with the symbol $\sigma$. For ease of notation and reading, the adjective ``inelastic'' and an index such as ``inel'' is omitted where the context allows it unambiguously.

The cross section depends on the energy of the incident electrons: $\sigma \equiv \sigma(\Ekin)$. This dependence has been neglected in the formal modeling of a KATRIN measurement within the previous chapter~\ref{sec:intSpecModel}. This chapter investigates effects related to the incorporation of the energy dependence. Section~\ref{sec:eDepScatCrossSecSources} lists cross section values from different sources and relates them to each other. Section~\ref{sec:eDepScatCrossSecModel} extends the mathematical formalism for a KATRIN measurement in order to incorporate the energy dependence of the scattering cross section. Section~\ref{sec:eDepScatCrossSecNuMassInf} discusses the energy-dependence within the context of neutrino mass inference. In the end, section~\ref{sec:eDepScatCrossSecConclusion} concludes and offers an outlook.


\section{Cross Section for Electrons Scattering off Molecules of Hydrogen Isotopologues}
\label{sec:eDepScatCrossSecSources}
There exist several sources for cross-section formulae and values. This section gives an overview about the sources considered in this thesis and how they relate to each other. First, the cross section for electrons with an energy of \SI{18600}{eV} scattering off tritium was measured at the Troitsk experiment to be~\cite{Aseev2000}
\begin{equation}
	\sigma(\SI{18600}{eV}) = \SI{3.40\pm0.07e-22}{m^{-22}} \fullstop
\end{equation}
Also, the KATRIN Design Report lists a reference value~\cite{Angrik:2005ep}
\begin{equation}
	\sigma_\mathrm{TDR} = \SI{3.456e-22}{m^{-2}} \fullstop
\end{equation}
Furthermore, there exist theoretical calculations of the cross section for electrons scattering of hydrogen molecules. In section~\ref{sec:eDepScatCrossSecSourcesTheory} the theoretical formulae are reviewed. How the different sources relate to each other and what approach is chosen within the scope of this thesis is explained in section~\ref{sec:eDepScatCrossSecSourcesChoice}.
\subsection{Theoretical Formulae}
\label{sec:eDepScatCrossSecSourcesTheory}
An expression for the inelastic cross section for electrons scattering from hydrogen molecules can be found in~\cite{Liu1973}. Two expressions are given, one for relativistic incident electrons and one for non-relativistic incident particles. With regard to KATRIN the energies of $\upbeta$ electrons from tritium $\upbeta$ decay are relevant. The maximum relativistic $\beta$ factor of electrons from tritium $\upbeta$ decay is
\begin{align}
\beta(\Ekin, m) &= 
\sqrt{
	1-\frac{1}{
		(\frac{\Ekin}{m}+1)^2
	}
} \label{eq:eDepScatCrossSecSourcesCrossSecBetaFactor} \\
\Rightarrow\beta_\mathrm{max, T} &= 
\beta(\Eeff\approx\SI{18.6}{keV}, m_\elecIndex\approx\SI{511}{keV})\approx0.26 
\fullstop
\end{align}
Traveling at approximately a forth of the speed of light, the $\upbeta$ electrons are assumed to behave non-relativistic. Then, the given expression for the energy dependent cross section is~\cite{Liu1973}
\begin{equation}
\label{eq:eDepScatCrossSecSourcesCrossSecLiu}
\sigma(E) =  
(4 \pi a_0^2) \cdot
\left(\frac{E}{R}\right)^{-1} \cdot
\left[
C_1 \cdot \ln{\left(\frac{E}{R}\right)} + C_2
\right]
\end{equation}
with the Bohr radius\footnote{Bohr radius $a_0=\SI[separate-uncertainty=false]{0.529 177 210 67(12)e-10}{m}$~\cite{ReviewOfParticlePhysics}} $a_0$, 
the Rydberg energy\footnote{Rydberg energy $R=\SI[separate-uncertainty=false]{13.605 693 009(84)}{eV}$~\cite{ReviewOfParticlePhysics}} $R$ and two constants $C_1$ and $C_2$. The later two depend on the hydrogen isotopologue. Different values a stated in different works for isotopic hydrogen
\begin{subequations}
\label{eq:eDepScatCrossSecSourcesCrossSecLiuConstants}
\begin{align}
C_1 &= 1.5487 &&\text{\cite{Liu1973}}
\label{eq:eDepScatCrossSecSourcesCrossSecLiuConstantsC1}\\[10pt]
C_2 &= 2.2212\pm0.0434 &&\text{\cite{Liu1973}}
\label{eq:eDepScatCrossSecSourcesCrossSecLiuConstantsC2Uncert}\\
C_2 &= 1.53 &&\text{\cite{Gerhart1975}} \\
C_2 &= 2.4036 &&\text{\cite{Liu1987}}
\label{eq:eDepScatCrossSecSourcesCrossSecLiuConstantsC2}
\fullstop
\end{align}
\end{subequations}
The latest of these references~\cite{Liu1987} acknowledges that the listed values for $C_2$ are not compatible. Within this work, the value from~\cite{Liu1987} is chosen as it is the most up-to-date of the listed ones. 

In equation~\eqref{eq:eDepScatCrossSecSourcesCrossSecLiu}, $E$ denotes\footnote{I would like to thank F. Glück for pointing this out.}
\begin{equation}
	\label{eq:eDepScatCrossSecSourcesCrossSecNonRelEnergy}
	E \equiv E(\Ekin) = \frac{1}{2} m_\elecIndex \beta^2(\Ekin, m_\elecIndex)
\end{equation}
with $\beta$ as in equation~\eqref{eq:eDepScatCrossSecSourcesCrossSecBetaFactor}. Figure~\ref{fig:eDepScatCrossSecSourcesValues} shows the theoretical cross-section formula along with the measured value by the Troitsk experiment and the value from the KATRIN Design Report.
\begin{figure}[t]
	\centering
	\includegraphics[width=\textwidth]{\currentFigureFolder/crossSecNoZoom.pdf}
	\xcaption{Inelastic cross section for non-relativistic incident electrons scattering off molecular hydrogen isotopologues}{Inelastic cross section for non-relativistic incident electrons scattering off molecular hydrogen isotopologues.}{Shown is the theoretical calculation according to equation~\eqref{eq:eDepScatCrossSecSourcesCrossSecLiu} with constants from equation~\eqref{eq:eDepScatCrossSecSourcesCrossSecLiuConstantsC1} and~\eqref{eq:eDepScatCrossSecSourcesCrossSecLiuConstantsC2} where the later is assumed to have an uncertainty according to equation~\eqref{eq:eDepScatCrossSecSourcesCrossSecLiuConstantsC2Uncert}. Also shown is the measurement by~\cite{Aseev2000} at the Troitsk experiment and the value stated in the KATRIN Design Report~\cite{Angrik:2005ep}. The shown energy interval is chosen according to the \gls{mtd} of the \gls{ft} measurement campaign.}
	\label{fig:eDepScatCrossSecSourcesValues}
\end{figure}

\subsection{Relation to Former Works}
\label{sec:eDepScatCrossSecSourcesChoice}
As can be seen in figure~\ref{fig:eDepScatCrossSecSourcesValues}, the cross section from the KATRIN Design Report does not match the theoretical calculations used in this thesis. However, the value stated in the KATRIN Design Report can be recovered from equation~\eqref{eq:eDepScatCrossSecSourcesCrossSecLiu}. If instead of the energy interpretation
of equation~\ref{eq:eDepScatCrossSecSourcesCrossSecNonRelEnergy}, one applies the interpretation
\begin{equation}
	\label{eq:eDepScatCrossSecSourcesCrossSecTDREngeryInterpretation}
	E \equiv \Ekin
	\fullstop
\end{equation}
The obtained cross section is $\sigma(\Ekin \approx \SI{18564.4}{eV}) = \SI{3.456e-22}{m^2}$ as stated in the KATRIN Design Report where the energy \SI{18564.4}{eV} is within the KATRIN design analysis interval (see section~\ref{sec:intSpecModelMTD}). This work applies the energy interpretation~\eqref{eq:eDepScatCrossSecSourcesCrossSecTDREngeryInterpretation} when comparability to former work is of importance. Otherwise, interpretation~\eqref{eq:eDepScatCrossSecSourcesCrossSecNonRelEnergy} is used\footnote{The cross sections obtained from the theoretical formula~\eqref{eq:eDepScatCrossSecSourcesCrossSecLiu} applying the energy interpretation via equation~\eqref{eq:eDepScatCrossSecSourcesCrossSecNonRelEnergy} are in better agreement with recently taken data at KATRIN according to preliminary analysis results by dedicated subgroups of the KATRIN collaboration at the time of writing this thesis.}. Corresponding indications are given. The quantitative difference of these two interpretations can be assessed by expanding the $\beta$ factor~\eqref{eq:eDepScatCrossSecSourcesCrossSecBetaFactor} in the ratio $\Ekin/m_\elecIndex \approx 18.575/511 \approx 0.036 \ll 1$
\begin{equation}
	\beta^2 \approx 
	2 \frac{\Ekin}{m_\elecIndex} - 
	3 \left(\frac{\Ekin}{m_\elecIndex}\right)^2
\end{equation}
The energy interpretation of equation~\eqref{eq:eDepScatCrossSecSourcesCrossSecNonRelEnergy} then becomes
\begin{equation}
	E(\Ekin) \approx 0.95 \Ekin
\end{equation}
which is a shift in energy and hence in the cross section of about \SI{5}{\percent} compared to the interpretation in~\eqref{eq:eDepScatCrossSecSourcesCrossSecTDREngeryInterpretation}. Exact calculations are given in the subsequent sections.

\section{An Energy-Dependent Scattering Model within the KATRIN Formalism}
\label{sec:eDepScatCrossSecModel}
The energy-dependence of the cross section enters into the calculation of the scattering probabilities~\eqref{eq:intSpecModelNonAveragedScatProbs}. In the derivation, that is given in the previous section~\ref{sec:intSpecModelResponseScattering}, the dependence on the starting energy $\Esource$ of electrons is neglected. Instead an average starting energy and hence, an average scattering cross section
\begin{equation}
	\label{eq:eDepScatCrossSecModelTRDCrossSec}
	\sigma(\SI{18564.37463}{eV})=\SI{3.456e-22}{m^2}
\end{equation}  (energy interpretation as per equation~\ref{eq:eDepScatCrossSecSourcesCrossSecTDREngeryInterpretation}) is assumed. Table~\ref{tab:eDepScatCrossSecModelScatProbs} lists the corresponding scattering probabilities averaged over all starting positions and pitch angles of electrons. Additionally, the results of a Monte-Carlo simulation by~\cite{Groh2015} and the values using the energy interpretation of equation~\eqref{eq:eDepScatCrossSecSourcesCrossSecNonRelEnergy} are given. How the energy-dependence of the scattering probabilities can be modeled is shown in section~\ref{sec:eDepScatCrossSecModelDescription}. Section~\ref{sec:eDepScatCrossSecModelDiscussion} discusses the given model from a physical perspective and section~\ref{sec:eDepScatCrossSecModelPerformanceAndAccuracy} considers its computational performance and accuracy. \todo{update and complete!}

\begin{table}[t]
	\centering
	\xcaption{Probabilities for severalfold scattering of electrons in the \gls{wgts}}{Probabilities for severalfold scattering of electrons in the \gls{wgts}}{averaged over all starting positions and starting pitch angles. Both, the values from a Monte Carlo (MC) simulation and the values according to equation~\eqref{eq:intSpecModelAveragedScatProbs} are given. The  cross section was evaluated at an energy of $E=\SI{18564.37463}{eV}$ 
	for the two energy interpretations described by equation~\eqref{eq:eDepScatCrossSecSourcesCrossSecTDREngeryInterpretation} and~\eqref{eq:eDepScatCrossSecSourcesCrossSecNonRelEnergy}. Further input parameters to the calculations are a constant gas column density $\rho d = \SI{5e17}{cm^{-2}}$, 
	a \gls{wgts} length of $d=\SI{10.0820}{m}$
	and a maximum acceptance angle of $\thetaMax=\SI{50.7685}{\degree}$.}
	\begin{tabular}{lllr}
		\toprule
		\makecell[tl]{cross section (\SI{e-22}{m^{-2}}) $\rightarrow$} &
		3.456 &
		3.456 &
		3.673 \\
		\hline
		\makecell[tl]{source $\rightarrow$} & 
		\makecell[tl]{MC particle tracking\\ \cite{Groh2015}} & 
		\makecell[tl]{eq. \eqref{eq:intSpecModelAveragedScatProbs} \\ \cite{Groh2015, Kleesiek2014}} &
		\makecell[tr]{eq. \eqref{eq:intSpecModelAveragedScatProbs}}
		\\
		\hline
		\makecell[cl]{scattering count $\downarrow$} & 
		& 
	    & \\
		\hline
		\makecell{0} & $0.415 \pm 0.002$ & 0.41334 & 0.39564 \\
		\makecell{1} & $0.292 \pm 0.002$ & 0.29266 & 0.28967 \\
		\makecell{2} & $0.166 \pm 0.001$ & 0.16733 & 0.17298 \\
		\makecell{3} & $0.079 \pm 0.001$ & 0.07913 & 0.08590 \\
		\makecell{4} & $0.031 \pm 0.001$ & 0.03178 & 0.03634 \\
		\bottomrule
	\end{tabular}
	\label{tab:eDepScatCrossSecModelScatProbs}
\end{table}


\subsection{Model Description}
\label{sec:eDepScatCrossSecModelDescription}
\begin{figure}[th]
\includegraphics[width=\textwidth]{\currentFigureFolder/scatProbs012.pdf}
        \xcaption{Energy-dependent probabilities for electron scattering within the \gls{wgts}}{Energy-dependent probabilities for electron scattering within the \gls{wgts}.}{Shown are from top to bottom the probability for no, one-fold and two-fold scattering averaged over all starting positions and starting pitch angles of electrons within the \gls{wgts}. The lines show the Poisson model and the markers the extended model (see main text for a description of the models). The extended model is only shown for one-fold scattering because for no scattering, it equals the Poisson model and for two-fold scattering it was not calculated (see appendix~\ref{sec:appendixEDepScatCrossSecExtendedModelNumEval} for an explanation). The numerical evaluation of the extended model is subject to an uncertainty of $\sim10^{-5}$ (also see appendix~\ref{sec:appendixEDepScatCrossSecExtendedModelNumEval}), that is depicted as uncertainty bars. The shown energy range matches the measurement range of the \glsentryfull{ft} measurement campaign and the inset shows an energy span around the endpoint of the tritium $\upbeta$ spectrum.}
        \label{fig:eDepScatCrossSecModel}
\end{figure}
Within this section, two models are presented in order to incorporate the energy-dependence of the scattering cross section into the mathematical formalism of a KATRIN measurement.

\paragraph{Poisson Model}
An expression for the probability of $l$-fold scattering of electrons within the \gls{wgts} is derived in the previous section~\ref{sec:intSpecModelResponseScattering}. The given model is independent of the energy of the electrons. Instead of using a constant cross section, the energy dependence can be respected. The corresponding formulae from section~\ref{sec:intSpecModelResponseScattering} are repeated below, with the energy-dependence made explicit
\begin{subequations}
\label{eq:eDepScatCrossSecModelPoisson}
\begin{align}
    \mu(\Esource,\zSource,\thetaSource) =&
    \frac{\sigma(\Esource)}{\cos\thetaSource}
    \int_{\zSource}^{d/2} \rho(z)\d z \label{eq:energydepScatProbsPoissonExpectedScatCount} \\
    P_l(\Esource,\zSource,\thetaSource) =&
    \mathrm{Poisson}(\mu(\Esource,\zSource,\thetaSource),l) \label{eq:eDepScatCrossSecModelPoissonNonAveragedScatProbs} \\
    \bar{P}_l(\Esource) =&
    \frac{1}{d \cdot (1-\cos(\thetaMax))} 
      \int_{-d/2}^{d/2}  
          \int_{0}^{\theta_{max}} 
            \sin(\thetaSource)
            \mathrm{Poisson}(\mu(\Esource,\zSource,\thetaSource),l)
          \d\thetaSource
      \d\zSource
      \label{eq:eDepScatCrossSecModelPoissonAveragedScatProbs}
    \fullstop
\end{align}
\end{subequations}
As a reminder, $\bar{P}_l(\Esource)$ in the final equation \eqref{eq:energydepScatProbsPoisson} denotes the probability of $l$-fold scattering for a $\upbeta$ electron with a starting energy $\Esource$ averaged over all starting positions and pitch angles. In the following, this model is denoted ``Poisson model''. It is expected to be accurate for the probability of no scattering $\bar{P}_0(\Esource)$. But, depending on the required accuracy, for one or more scatterings the Poisson model does not necessarily hold as explained in the following paragraph.

\paragraph{Extended Model}
A scattering electron loses energy. The scattering cross section increases with decreasing energies and the electron becomes more likely to scatter again. In other words, the probabilities of individual scattering processes are no longer independent. This violates one of the preconditions to model the scattering probabilities via a Poisson distribution. Another model is suggested that partly accounts for this fact (inspired by a model in~\cite{Groh2015} that incorporates changes of the electron pitch angle due to scattering). It assumes a fixed energy loss per scattering. A descriptive derivation is given in appendix~\ref{sec:appendixEDepScatCrossSecExtendedModel}. In this section it is labeled by
\begin{equation}
	\label{eq:eDepScatCrossSecModelExtended}
	\bar{P}^{\star}_l(\Esource)
\end{equation}
and it denotes the probability of $l$-fold scattering for a $\upbeta$ electron with a starting energy $\Esource$ averaged over all starting positions and pitch angles assuming a fixed energy loss $\epsilon$ per scattering. In the following, this model is denoted the ``extended model''. The value $\epsilon=\SI{12.6}{eV}$ was chosen as it is the most probable energy loss for electrons traveling through tritium gas (see figure~\ref{fig:intSpecModelAseevEloss}). A more accurate description would incorporate the full energy loss function. This may be the subject of a future study. 

The extended model was evaluated numerically as it includes one limit and two integrals (see appendix~\ref{sec:appendixEDepScatCrossSecExtendedModelFormalism}). The numerical accuracy had to be good enough to decided whether it differs significantly from the Poisson model or not. At the same time, a balance between the numerical accuracy and the evaluation run time had to be found. The probability for one-fold scattering could be calculated with a numerical accuracy on the $10^{-5}$ level (see appendix~\ref{sec:appendixEDepScatCrossSecExtendedModelNumEval} on how this value is derived and how it can be interpreted). For more than one scattering the evaluations are not yet done.

\subsection{Model Discussion}
\label{sec:eDepScatCrossSecModelDiscussion}
Figure~\ref{fig:eDepScatCrossSecModel} shows the Poisson model along with the suggested extended model. The results are discussed in the following paragraphs.

\paragraph{Model compatibility}
For the probability of one-fold scattering the difference between the Poisson and the extended model is below $10^{-4}$. Table \ref{tab:eDepScatCrossSecModelScatProbs} lists the scattering probabilities for an energy independent Poisson model and a reference cross section $\sigma(\Ekin\approx\SI{18564.4}{eV})\approx\SI{3.673e-22}{m}$. The energy dependent Poisson model recovers the energy independent model exactly at the corresponding energy as expected.

\paragraph{Trend of energy-dependence}
For a decreasing starting energy of electrons within the \gls{wgts} the probability for no and one-fold scattering also decreases while the probability for two-fold scattering increases. In the following, an explanation for this is given. Therefore, the expected amount of scatterings $\mu$ (see equation~\ref{eq:energydepScatProbsPoissonExpectedScatCount}) is averaged over all starting positions~$\zSource$ and pitch angles~$\thetaSource$
\begin{equation}
    \bar{\mu}(\Esource) = 
      \frac{1}{d \cdot (1-\cos(\thetaMax))} 
      \int_{-d/2}^{d/2}  
          \int_{0}^{\theta_{max}} 
            \sin(\thetaSource)
            \mu(\Esource,\zSource,\thetaSource)
          \d\thetaSource
      \d\zSource
      \fullstop
\end{equation}
One can see, that this expression is monotonous in the starting energy $\Esource$ because all appearing quantities are positive and $\mu(\Esource,\zSource,\thetaSource)$ is monotonous in $\Esource$. This monotony means, faster electrons are less likely to scatter. It also means, for the depicted energy range $\Esource \in [\SI{17}{keV}, \SI{18.6}{keV}]$ in figure~\ref{fig:eDepScatCrossSecModel} an upper and lower limit for $\bar{\mu}(\Esource)$ can be given
\begin{equation}
	\bar{\mu}(\SI{18.6}{eV}) \approx 1.14 < 
	\bar{\mu}(\Esource) < 
	1.23 \approx \bar{\mu}(\SI{17}{keV})
	\fullstop
\end{equation}
In other words, electrons with a starting energy between \SI{17}{keV} and \SI{18.6}{keV} are expected to scatter between 1.14 and 1.23 times when traveling through the \gls{wgts}. If the starting energy decreases and hence scattering becomes more likely, it also becomes more likely to scatter more than 1.14 to 1.23 times and less likely to scatter less than this amount of times. Thus, the probability for no and one-fold scattering decreases along with the starting energy of the electrons. At the same time the probability for more than one scatterings increases for lower starting energies. 

\FloatBarrier

\subsection{Model Implementation and Performance}
\label{sec:eDepScatCrossSecModelPerformanceAndAccuracy}

Within the scope of this thesis, the energy dependent Poisson model of equation~\eqref{eq:eDepScatCrossSecModelPoisson} was implemented into the \gls{ssc} software framework. The extended model of equation~\eqref{eq:eDepScatCrossSecModelExtended} was not further investigated in this work. The energy dependence of the scattering cross section can not be neglected in neutrino mass inference as is explained in the subsequent section~\ref{sec:eDepScatCrossSecNuMassInf}. For that reason, the impact on the fitting run time by using an energy dependent cross section was probed. Depending on the \gls{mtd}, a fit might become slower by a factor of 40 to 120. This is due to the fact, that, when integrating over the energy loss in the response function in equation~\eqref{eq:intSpecModelResponse}, the scattering probabilities have to be recomputed in every step of the numerical integration. It might be beneficial to investigate whether the evaluation can be speed up in the future.


\section{Effect on the Inferred Neutrino Mass}
\label{sec:eDepScatCrossSecNuMassInf}
\begin{figure}[t]
\includegraphics[width=\textwidth]{\currentFigureFolder/nuMassShiftForEDepCrossSec.pdf}
        \xcaption{Neutrino mass shift due to an energy dependent inelastic scattering cross section}{Neutrino mass shift due to an energy dependent inelastic scattering cross section}{as calculated by SSC for 5 measurement intervals. The configuration for the calculation, especially the \glsentryfull{mtd}, follows the KATRIN Design Report \cite{Angrik:2005ep}. For comparison the statistical uncertainty is plotted as well. It is derived from the profile likelihood. The lines between the 5 markers are linear interpolations.}
        \label{fig:eDepScatCrossSecNuMassInfShifts}
\end{figure}
It was investigated how much the squared neutrino mass that is inferred from a KATRIN measurement would be shifted if the energy-dependence of the scattering cross section is neglected in the corresponding fitting procedure. The comparability to former results is of importance within this section. For that reason, the energy interpretation of equation~\ref{eq:eDepScatCrossSecSourcesCrossSecTDREngeryInterpretation} is used, which yields a cross section of $\sigma=\SI{3.456e-22}{m^{-2}}$~\cite{Angrik:2005ep} within the KATRIN design analysis interval of \SI{30}{eV} below the endpoint of the tritium $\upbeta$ spectrum.

First, the results from a similar study in~\cite{Groh2015} are reviewed because they might intuitively contradict the results presented in this thesis. Then, the results of this thesis are listed and an explanation is given, why both sets of results are in accordance.

In~\cite{Groh2015} was investigated how much a constant offset of the cross section would shift the inferred squared neutrino mass if it were neglected in the analysis. A rule of thumb for the neutrino mass shift in dependence on the offset of the cross section $\Delta\sigma$ is given~\cite{Groh2015}
\begin{equation}
	\label{eq:eDepScatCrossSecNuMassInfShiftRuleOfThumb}
	\frac{\Delta\nuMass^2(\sigma)}{\SI{e-3}{eV^2}} =
	-0.45
	-1204\cdot
	\frac{
		\Delta\sigma
	}{
		\sigma_\mathrm{TDR}
	}
	\qquad \text{ with } \quad 
	\sigma_\mathrm{TDR} = \SI{3.456e-22}{m^{-2}}
	\fullstop
\end{equation}
The difference between the cross section $\sigma_\mathrm{TDR}$ in the middle of the KATRIN analysis interval and the ones at its boundary energies can be determined using the energy dependent formula for the cross section~\eqref{eq:eDepScatCrossSecSourcesCrossSecLiu} with the energy interpretation of equation~\eqref{eq:eDepScatCrossSecSourcesCrossSecTDREngeryInterpretation}. Together with the rule of thumb~\eqref{eq:eDepScatCrossSecNuMassInfShiftRuleOfThumb} one obtains
\begin{align*}
	\label{eq:eDepScatCrossSecNuMassInfShiftsByGroh}
	\abs{\Delta\nuMass^2(\sigma(\SI{18575}{eV}))} &< \SI{3e-3}{eV^2} \\
	\abs{\Delta\nuMass^2(\sigma(\SI{18545}{eV}))} &< \SI{3e-3}{eV^2} 
	\fullstop
\end{align*}

In contrast to the constant offset of the cross section investigated in~\cite{Groh2015}, in the scope of this thesis, an energy-dependent offset was investigated. A KATRIN neutrino mass measurement for a neutrino mass of \SI{0}{eV} was simulated using an energy dependent cross section. A model that uses a constant cross section was fitted to the simulated spectrum. This procedure was repeated for the five different \gls{mtd}s given in the KATRIN Design Report. Figure~\ref{fig:eDepScatCrossSecNuMassInfShifts} shows the results for the obtained squared neutrino masses. The difference from \SI{0}{eV} respectively the shift is
\begin{equation}
	\Delta\nuMass^2 = \SI{1.09e-2}{eV^2}
\end{equation}
for the \SI{30}{eV} range. This offset is larger than the one listed above under equation~\eqref{eq:eDepScatCrossSecNuMassInfShiftsByGroh}. This might be contra intuitive, but can be explained. The reason is that the fit parameter for the signal amplitude $\sigAmp$ in a nominal KATRIN fit (see section~\ref{sec:statMethodsStandardFit}) can compensate for a constant offset of the cross section. However, it can not compensate for an energy-dependent one. In order to verify this statement, two arguments are presented below.
First, the study from~\cite{Groh2015} was reproduced - once with a free fit parameter for $\sigAmp$ and once with fixed $\sigAmp=1$. The results are shown in figure.





\begin{figure}[t]
	\includegraphics[width=\textwidth]{\currentFigureFolder/nuMassShiftForConstCrossSec.pdf}
	\xcaption{}{}{}
	\label{fig:eDepScatCrossSecNuMassInfShiftsForConstCrossSec}
\end{figure}

\section{Conclusion and Outlook}
\label{sec:eDepScatCrossSecConclusion}
The energy dependence of the scattering cross section enters into the calculation of the scattering probabilities. An accurate modelling of the energy dependent scattering probabilities is challenging due to performance reasons, but modelling them according to a Poisson distribution is possible. It was shown that the difference between the Poisson model and a more accurate model for 1-fold scattering is on the $10^{-4}$ level. The cases for more than 1 scatterings need further investigation. Also a fixed energy loss per scattering was assumed instead of a energy loss probability distribution. Future work might consider these aspects and what influence a more accurate modeling on the scattering probabilities has on the neutrino mass determination.

Given the KATRIN uncertainty budget, when modeling the energy dependent scattering probabilities via a Poisson distribution, the energy dependence of the scattering cross section is not negligible for measurement intervals that extend more than \SI{35}{eV} below the endpoint of the tritium $\upbeta$ spectrum.

Including the energy dependence in the analysis increases the run time of the fitting procedure significantly. Future work might consider to precalculate the scattering probabilities for different fixed energies and use interpolation techniques for energies in-between the fixed ones.





The total scattering cross section is a sum of the cross section for elastic and inelastic scattering.
\begin{equation}
\sigma(\Ekin) = \sigma_\mathrm{el} + \sigmaInel(\Ekin)
\fullstop
\end{equation}
Only the energy dependence of the inelastic scattering cross section will be considered. This is justified for now as according to \cite{Angrik:2005ep} the cross section for elastic scattering is about an order of magnitude smaller than the one for inelastic scattering. An expression for the inelastic cross section for electrons scattering from hydrogen molecules can be found in \cite{Liu1973}. Two expressions are given, one for relativistic incident particles and one for non-relativistic incident particles. For the maximum relativistic $\beta$ factor of $\upbeta$ electrons from tritium decay one finds
\begin{equation}
\begin{split}
\beta(\Ekin, m) &= 
\sqrt{
	1-\frac{1}{
		(\frac{\Ekin}{m}+1)^2
	}
} \\
\beta_\mathrm{max} &= 
\beta(\Eeff\approx\SI{18.6}{keV}, m_\elecIndex\approx\SI{511}{keV})\approx0.26
\end{split}
\end{equation}
Traveling at approximately a forth of the speed of light, the $\upbeta$ electrons are assumed to be non-relativistic. Then, the given expression for the energy dependent cross section is
\begin{equation}
\label{eq:crossSecLiu}
\sigma(\Ekin) =  
(4 \pi a_0^2) \cdot
\left(\frac{\Ekin}{R}\right)^{-1} \cdot
\left[
C_1 \cdot \ln{\left(\frac{\Ekin}{R}\right)} + C_2
\right]
\end{equation}
with the Bohr radius $a_0$, the Rydberg energy $R$ and two constants given as
\begin{equation}
\label{eq:crossSecLiuConstants}
C_1 = 1.5487 
\quad \text{and} \quad 
C_2 = 2.2212\pm0.0434
\fullstop
\end{equation}
Note that in other works $C_2=2.4036$ \cite{Liu1987} and $C_2=1.53$ \cite{Gerhart1975} are given. The value in \eqref{eq:crossSecLiuConstants} from \cite{Liu1973} was chosen to enable comparability with the KATRIN Design Report~\cite{Angrik:2005ep}.

\begin{table}[t]
	\centering
	\xcaption{Inelastic cross section for electrons scattering off molecular hydrogen isotopologues for different incident energies}{Inelastic cross section for electrons scattering off molecular hydrogen isotopologues for different incident energies.}{Listed are important values with reference to KATRIN. Except for the measured value, they are obtained using \eqref{eq:crossSecLiu}. The values are given relative to an assumed endpoint of the measured spectrum $\Eeff=\SI{18575}{eV}$.}
	\begin{tabular}{lll}
		\toprule
		\makecell[tl]{kin. energy} & 
		\makecell[tl]{cross section \\ (\SI{e-22}{m})} & 
		\makecell[tl]{Note} \\
		\hline
		$\Eeff-\SI{1600}{eV}$ & 
		3.740 & 
		largest range in \gls{ft} campaign \\
		$\Eeff-\SI{90}{eV}$ & 
		3.469 & 
		largest range in \gls{knm1} campaign \\
		$\Eeff-\SI{30}{eV}$ & 
		3.459 & 
		KATRIN reference measurement interval \cite{Angrik:2005ep} \\
		$\Eeff-\SI{10}{eV}$ & 
		3.456 & 
		KATRIN reference value \cite{Angrik:2005ep} \\
		$\Eeff$ & 
		3.454 & 
		endpoint of tritium $\upbeta$ spectrum \\
		\SI{18600}{eV} & 
		$3.40\pm0.07$ & 
		measured at the Troitsk experiment \cite{Aseev2000} \\
		\bottomrule
	\end{tabular}
	\label{tab:crossSections}
\end{table}

Furthermore, the total inelastic scattering cross section was measured at the Troitsk experiment and the KATRIN Design Report states a reference value. The values are listed in table \ref{tab:crossSections}. The reference value matches the theoretical calculation using \eqref{eq:crossSecLiu} at a kinetic energy of $\Ekin\approx\SI{18564.4}{eV}$ which would be the center of a $\Delta\Ekin=\SI{20}{eV}$ KATRIN measurement interval. Note that in the KATRIN reference measurement interval $[\Eeff-\SI{30}{eV}, \Eeff]$ the cross section varies  $\sim\SI{0.14}{\percent}$. This variation is below its theoretical uncertainty given by \eqref{eq:crossSecLiuConstants}. In the measurement interval of the \gls{ft} campaign it varies $\sim\SI{8}{\percent}$.



In a KATRIN measurement there exists a minimum retarding energy $\qUmin$ and only electrons with a kinetic energy greater than $\qUmin$ can reach the detector. A scattering cross section averaged over the kinetic energy of the electrons reaching the detector was assumed
\begin{equation}
\sigma = \sigmaAvg = 
\frac{1}{\Delta\Ekin} 
\int_{\qUmin}^{\qUmin+\Delta\Ekin} \sigma(\Ekin) \d \Ekin 
\quad \text{with} \quad
\Delta\Ekin = \Eeff-\qUmin
\fullstop
\end{equation}
Here, the ``effective endpoint'' $\Eeff$ denotes the highest kinetic energy of $\upbeta$ electrons reaching the detector. It does not necessarily match the endpoint of the tritium $\upbeta$ spectrum as experiment specific effects might shift it. $\Eeff=\SI{18575}{eV}$ is assumed. Instead of an average cross section an energy dependent formula can be used. Incorporating the energy dependence makes the model more complicated and slower to compute; neglecting it may lead to wrong results. More light will be shed on both aspects within this chapter.

\section{Motivation and Significance for the KATRIN Experiment}
The motivation to implement an energy dependent cross section into the \gls{ssc} framework is two-fold:
\par{\textbf{Deep scans:} According to \cite{Groh2015} for a neutrino mass determination with the precision goals of KATRIN, the inelastic scattering cross section has to be known with an upper uncertainty of $\SI{0.0055e-22}{m}$ (\SI{0.16}{\percent}). This requirement might be surpassed by neglecting that the scattering cross section is not constant, but varies with energy. According to table \ref{tab:crossSections} the requirement is fulfilled in a measurement interval of $[\Eeff-\SI{30}{eV}, \Eeff]$. According to \eqref{eq:crossSecLiu} it is violated if the lower bound of the measurement interval is below \SI{18531}{eV}. Scans deeper into the tritium $\upbeta$ spectrum increase the count rates and hence, improve the statistic uncertainty on the neutrino mass. Also, for the search of sterile neutrinos deeper scans are necessary. On top of that, deeper scans have already been performed e.g. in the \gls{ft} campaign and help to establish an even better understanding of the KATRIN apparatus. In these cases the energy dependence is not negligible.}
\par{\textbf{Comparability:} A possible cross-check for software is its comparison to other software that was developed independently. \gls{ssc} is part of a data fitting suit. Other fitting software exists within the KATRIN collaboration that uses an energy dependent scattering cross section, e.g. FITRIUM \cite{Fitrium}. Furthermore, \gls{ssc} is commonly cross-checked against a Monte-Carlo particle tracking software called KASSIOPEIA \cite{KATRINCOL2019} which also implements an energy dependent scattering cross section.}


\begin{subequations}
	\label{eq:energydepScatProbsModel}
	\begin{align}
	\mu(E,\thetaSource) =&
	\frac{\sigma(E)\rho L}{\cos\thetaSource} \\
	p_0(E,\thetaSource,n) =&
	\left(
	1-\frac{\mu(E,\thetaSource)}{N}
	\right)^n \\
	p_l(E,\thetaSource,n) =&
	\sum_{k=l}^{n}
	p_{l-1}(E,\thetaSource,k-1)
	\left(
	1-p_0(E-(l-1)\epsilon,\thetaSource,1)
	\right)
	p_0(E-l\epsilon,\thetaSource,n-k) \\
	\bar{p}_l(E,\thetaSource) =& 
	\frac{1}{L}
	\int_{0}^{L}
	p_l(E,\thetaSource,
	\left\lceil N \frac{\zSource}{L}\right\rceil
	)
	\d \zSource 
	\label{eq:energydepScatProbsZAverage} \\
	P^{\star}_l(\Esource,\thetaSource) =& 
	\lim_{N\rightarrow\infty} \bar{p}_l(\Esource,\thetaSource) 
	\label{eq:energydepScatProbsLimit} \\ 
	\bar{P}^{\star}_l(\Esource) =& 
	\frac{1}{1-\cos\thetaMax}
	\int_0^{\thetaMax}
	P^{\star}_l(\Esource,\thetaSource) 
	\d \thetaSource
	\label{eq:energydepScatProbs}
	\fullstop
	\end{align}
\end{subequations}


This reasoning is also reflected by deriving the scattering probabilities of the Poisson model~\eqref{eq:eDepScatCrossSecModelPoissonScatProbs} for the starting energy where $\mu(\Esource,\zSource,\thetaSource)$ denotes the expected scattering count
\begin{equation}
\frac{\d P_l(\Esource,\zSource,\thetaSource)}
{\d \Esource} =
\underbrace{
	\mathrm{Poisson}(\mu(\Esource,\zSource,\thetaSource),l)
	\vphantom{\frac{\d \sigma(\Esource)}{\d \Esource}}
}_{\geq 0} \cdot
\underbrace{
	\left(
	\frac{l}{\mu(\Esource,\zSource,\thetaSource)} -1
	\right)
	\vphantom{\frac{\d \sigma(\Esource)}{\d \Esource}}
}_{(*)} \cdot
\underbrace{
	\frac{\int_{\zSource}^{d} \rho(z) \d z}
	{d \cdot \cos\thetaSource}
	\vphantom{\frac{\d \sigma(\Esource)}{\d \Esource}}
}_{>0} \cdot
\underbrace{
	\frac{\d \sigma(\Esource)}{\d \Esource}
}_{<0} \fullstop
\end{equation}
The sign of the derivative is determined by the sign of $(*)$. 


Albeit,  $l<\bar{\mu}(\Esource)$ and positive for $l>\bar{\mu}(\Esource)$.



in the on the run time was probed for 5 different \gls{mtd}s. They differ in the amount and range of retarding potentials. Both aspects should influence the run time. The run time should be approximately linear in the amount of retarding potentials. The run time should get longer the wider the range of the retarding potentials is as the numerically evaluated integral over energies in \eqref{eq:countsDependingOnPositionAndPitchAngle} then stretches over a wider range. Figure \ref{fig:scatCrossSecRunTimes} shows that the run times increase by a factor of approximately $40-120$ between assuming a constant cross section and an energy dependent one.


Note that a similar model has been derived for a fixed change of the pitch angle $\theta$ per scattering in \cite{Groh2015}. This model for angular changes is implemented in the \gls{ssc} software and can be used in fitting procedures. Though, an important difference is that the determination of the count rate requires an integral over the starting energy $\Esource$ in \eqref{eq:countsDependingOnPositionAndPitchAngle}. Hence, the energy dependent scattering probabilities have to be recomputed in every step of the numerical integration. This is not the case for the model considering the angular changes.

Albeit the energy dependent Poisson model might not fully hold in the case of an energy dependent cross section, figure \ref{fig:energyDependentScatProbs} shows that for larger measurement intervals it is more accurate than assuming energy independent scattering probabilities. Hence, it is a reasonable choice to implement the energy dependent Poisson model into the \gls{ssc} software. This was done. As mentioned the energy dependent scattering probabilities have to be recomputed in every step of the numerical integration in \eqref{eq:countsDependingOnPositionAndPitchAngle} over the $\upbeta$ electron energy. 


    
    \chapter{(Theoretical Corrections to the \texorpdfstring{$\upbeta$}{Beta} Decay Spectrum)}
    \section{Motivation}
    What are theoretical corrections?
    
    \section{Application to Data}
    What is the effect on the parameters of interest? How can it be explained? 
    
    
    
    \chapter{Parameter Inference at KATRIN with KaFit}
    
    \section{The KATRIN Likelihood}
    What does parameter inference mean? What does sensitivity on the neutrino mass mean? What is the likelihood? What is a Frequentist approach? What is a Bayesian approach? 
    \section{Overview of Analysis Methods}
    What are the parameters of interest? What is run stacking? What is a uniform fit? Was the neutrino mass fixed? Which energy loss model/fsd binning/...? More general, which analysis configuration was used and why?
    
    \section{Inference Algorithms}
    \subsection{Classical Minimizer}
    What algorithm is implemented in/interfaced to KaFit? What is MINUIT and MINOS? What are the pros and cons?
    \subsection{Markov-Chain-Monte-Carlo}
    What is a Markov-Chain-Monte-Carlo?  What algorithm is implemented in/interfaced to KaFit? What are the pros and cons of the method? What are the pros and cons of the different implementations?
    
    \section{Treatment of Uncertainties}
    \subsection{Nuisance Parameters}
    What are nuisance parameters? How can they be included in an analysis? What are the difficulties? How can these difficulties be circumvented?
    \subsection{Penalized Likelihood and Priors}
    What does it mean to constrain a nuisance parameter? How can penalty terms and priors be described as constraints? How do penalty terms priors compare? How does this fit in a Frequentist and Bayesian framework?
    \subsection{Monte Carlo Propagation and Covariance Matrix Approach}
    What is the Monte Carlo Propagation and the Covariance Matrix Approach. Why are they similar? Where do they differ? What is the motivation behind using them? What are the pros and cons? What is the convergence criteria for the sampling?
    \subsection{Shift Method}
    What is the shift method? What are its limitations and when is it needed?
    
    
    
    \chapter{(Energy Loss Normalization on First Tritium)}
    
    
    
    \chapter{Studies using the Preliminary KATRIN Energy Loss Model}
    \section{The Preliminary KATRIN Energy Loss Model}
     Which systematic effects were considered? What are the quantified uncertainties on them and where do these numbers come from? Why this selection?
    \section{Neutrino Mass Sensitivity}
    \subsection{Motivation}
    Why combining systematics? What was possible in the scope of this thesis? What does this thesis try to show?
    \subsection{Methodology}
    What is the sensitivity on the parameters of interest? What is the z score and shrinkage for all parameters?
    \subsection{Model Configuration}
    Which model configuration (MTD, Slicing, Eloss-Model, FSD Binning, ...) was chosen and why?
    \subsection{Results}

    \section{Application to Data}
    
    \chapter{Conclusions}

    \Appendix
    \chapter*{\appendixname} \addcontentsline{toc}{chapter}{\appendixname}
    \section{Energy-Dependent Model of the Probability for Electron-Scattering in the \glsentryshort{wgts}}
\label{sec:appendixEDepScatCrossSecExtendedModel}
Section~\ref{sec:eDepScatCrossSecModel} introduces a model for the probability of electron scattering in the \gls{wgts} using the Poisson distribution. As explained, depending on the required accuracy, the conditions for using a Poisson distribution might be violated. Section~\ref{sec:appendixEDepScatCrossSecExtendedModelFormalism} introduces a model beyond the description via a Poisson distribution. Its evaluation was done numerically which demanded a trade-off between accuracy and run time. The latter is assessed in section~\ref{sec:appendixEDepScatCrossSecExtendedModelNumEval}.


\section{Proof that the Poisson Model for Electron-Scattering is a Probability Density}
\label{sec:appendixEDepScatCrossSecPoissonModelProbDensityProof}
In the following, it is shown that the energy-dependent Poisson model $\bar{P}_l(\Esource)$ (see equation~\ref{eq:eDepScatCrossSecModelPoisson}) resembles a probability density in $l$ independent of a constant starting energy~$\Esource$ of electrons. Accordingly, the two properties that have to be verified are
\begin{align}
\forall \Esource > 0: \quad
&\bar{P}_l(\Esource) > 0 \\
&\sum_{l}^{\infty} \bar{P}_l(\Esource) = 1
\fullstop
\end{align}
The fist condition holds because all quantities in the calculation of $\bar{P}_l(\Esource)$ are positive. For the second condition, one can use that the Poisson distribution is a probability density that sums to one
\begin{align*}
\sum_{l}^{\infty} \bar{P}_l(\Esource) &=
\frac{1}{d \cdot (1-\cos(\thetaMax))} 
\int_{-d/2}^{d/2}  
\int_{0}^{\theta_{max}} 
\sin(\thetaSource)
\sum_{l}^{\infty}
\mathrm{Poisson}(\mu(\Esource,\zSource,\thetaSource),l)
\d\thetaSource
\d\zSource \\  &=
\frac{1}{d \cdot (1-\cos(\thetaMax))} 
\int_{-d/2}^{d/2}  
\int_{0}^{\theta_{max}} 
\sin(\thetaSource)
\cdot 1
\d\thetaSource
\d\zSource \\ &= 1
\fullstop
\end{align*}
This verifies that the Poisson model is a probability density in the amount of scatterings $l$.
\clearpage

\begin{samepage}
\section{Measurement Time Distributions}
\label{sec:appendixEDepScatCrossSecMTDs}
Below, the five different \gls{mtd}s of three years total measurement time used in the study in section~\ref{sec:eDepScatCrossSecNuMassInfThisWork} are given. The retarding energies $qU_\mathrm{rel}$ are denoted relative to $E_0=\SI{18575}{eV}$.\\[10pt]
\begin{center}
{\scriptsize
\centering
\begin{tabular}{rrrrrr}
	\toprule
	$qU_\mathrm{rel} (eV)$ & \multicolumn{5}{c}{measurement time (s)} \\
	\hline
	-50.0 &        0 &        0 &        0 &        0 &   550000 \\
	-49.0 &        0 &        0 &        0 &        0 &   550000 \\
	-48.0 &        0 &        0 &        0 &        0 &   550000 \\
	-47.0 &        0 &        0 &        0 &        0 &   550000 \\
	-46.0 &        0 &        0 &        0 &        0 &   550000 \\
	-45.0 &        0 &        0 &        0 &        0 &   550000 \\
	-44.0 &        0 &        0 &        0 &        0 &   550000 \\
	-43.0 &        0 &        0 &        0 &        0 &   550000 \\
	-42.0 &        0 &        0 &        0 &        0 &   550000 \\
	-41.0 &        0 &        0 &        0 &        0 &   550000 \\
	-40.0 &        0 &        0 &        0 &   720000 &   550000 \\
	-39.0 &        0 &        0 &        0 &   720000 &   550000 \\
	-38.0 &        0 &        0 &        0 &   720000 &   550000 \\
	-37.0 &        0 &        0 &        0 &   720000 &   550000 \\
	-36.0 &        0 &        0 &        0 &   720000 &   550000 \\
	-35.0 &        0 &        0 &        0 &   720000 &   550000 \\
	-34.0 &        0 &        0 &        0 &   720000 &   550000 \\
	-33.0 &        0 &        0 &        0 &   720000 &   550000 \\
	-32.0 &        0 &        0 &        0 &   720000 &   550000 \\
	-31.0 &        0 &        0 &        0 &   720000 &   550000 \\
	-30.0 &        0 &        0 &  1050000 &   720000 &   550000 \\
	-29.0 &        0 &        0 &  1050000 &   720000 &   550000 \\
	-28.0 &        0 &        0 &  1050000 &   720000 &   550000 \\
	-27.0 &        0 &        0 &  1050000 &   720000 &   550000 \\
	-26.0 &        0 &        0 &  1050000 &   720000 &   550000 \\
	-25.0 &        0 &  1350000 &  1050000 &   720000 &   550000 \\
	-24.0 &        0 &  1350000 &  1050000 &   720000 &   550000 \\
	-23.0 &        0 &  1350000 &  1050000 &   720000 &   550000 \\
	-22.0 &        0 &  1350000 &  1050000 &   720000 &   550000 \\
	-21.0 &        0 &  1350000 &  1050000 &   720000 &   550000 \\
	-20.0 &  1910000 &  1350000 &  1050000 &   720000 &   550000 \\
	-19.0 &  1910000 &  1350000 &  1050000 &   720000 &   550000 \\
	-18.0 &  1910000 &  1350000 &  1050000 &   720000 &   550000 \\
	-17.0 &  1910000 &  1350000 &  1050000 &   720000 &   550000 \\
	-16.0 &  1910000 &  1350000 &  1050000 &   720000 &   550000 \\
	-15.0 &  1910000 &  1350000 &  1050000 &   720000 &   550000 \\
	-14.0 &  1910000 &  1350000 &  1050000 &   720000 &   550000 \\
	-13.0 &  1910000 &  1350000 &  1050000 &   720000 &   550000 \\
	-12.0 &  1910000 &  1350000 &  1050000 &   720000 &   550000 \\
	-11.0 &  1910000 &  1350000 &  1050000 &   720000 &   550000 \\
	-10.0 &  1910000 &  1350000 &  1050000 &   720000 &   550000 \\
	-9.0 &  1910000 &  1350000 &  1050000 &   720000 &   550000 \\
	\hline
	-8.0 &   908164 &   908164 &   908164 &   908164 &   908164 \\
	-7.5 &   908164 &   908164 &   908164 &   908164 &   908164 \\
	-7.0 &   908164 &   908164 &   908164 &   908164 &   908164 \\
	-6.5 &   908164 &   908164 &   908164 &   908164 &   908164 \\
	-6.0 &   908164 &   908164 &   908164 &   908164 &   908164 \\
	-5.5 &   908164 &   908164 &   908164 &   908164 &   908164 \\
	\hline
	-5.0 &  1975257 &  1975257 &  1975257 &  1975257 &  1975257 \\
	-4.5 & 12214808 & 12214808 & 12214808 & 12214808 & 12214808 \\
	-4.0 & 20728847 & 20728847 & 20728847 & 20728847 & 20728847 \\
	-3.5 &  2020665 &  2020665 &  2020665 &  2020665 &  2020665 \\
	\hline
	-3.0 &  3300000 &  3300000 &  3300000 &  3300000 &  3300000 \\
	-2.0 &  3300000 &  3300000 &  3300000 &  3300000 &  3300000 \\
	-1.0 &  3300000 &  3300000 &  3300000 &  3300000 &  3300000 \\
	0.0 &  3300000 &  3300000 &  3300000 &  3300000 &  3300000 \\
	1.0 &  3300000 &  3300000 &  3300000 &  3300000 &  3300000 \\
	2.0 &  3300000 &  3300000 &  3300000 &  3300000 &  3300000 \\
	3.0 &  3300000 &  3300000 &  3300000 &  3300000 &  3300000 \\
	4.0 &  3300000 &  3300000 &  3300000 &  3300000 &  3300000 \\
	5.0 &  3300000 &  3300000 &  3300000 &  3300000 &  3300000 \\
	\bottomrule
\end{tabular}
}
\end{center}
\end{samepage}
\clearpage
    
    
    \section{First Tritium Data}
    What is First Tritium? Which data was selected?
    
    \glsaddall
    \printglossary[type=\acronymtype]
    \TheBibliography
    \printbibliography
    % \nocite{*} or maybe \nocite{Kon64,And59} for specific entries
    %\nocite{*}
    %\bibliographystyle{babalpha}
\end{document}
