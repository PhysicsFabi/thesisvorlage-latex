\def\currentRootFolder{chapter/katrinExperiment/neutrinoMassMeasurement}
\def\currentFigureFolder{\currentRootFolder/fig}
\newcommand{\sigmaInel}{\sigma_\mathrm{inel}}
\newcommand{\sigmaAvg}{\sigma_\mathrm{avg}}

\newcommand{\qUmin}{qU_\mathrm{min}}

\newcommand{\Ekin}{E_\mathrm{kin}}
\newcommand{\nSource}{n_\mathrm{S}}
\newacronym{ssc}{SSC}{source and spectrum calculation}


\section{A KATRIN Neutrino Mass Measurement}
\label{sec:katrinExpNuMassMeasurement}
KATRIN measures electron counts as described in \eqref{eq:countsSCCFinal} at a set of retarding energies ${qU_i}$. How much measurement time $t(qU_i)$ is attributed to a certain retarding energy is called a \gls{mtd}. The \gls{mtd} influences the experiment's sensitivity to the neutrino mass. An optimal \gls{mtd} balances the following aspects:
\begin{enumerate}
	\item Some measurement time has to be attributed to retarding energies beyond the endpoint of the spectrum to determine the background rate. The optimal duration depends on the background rate.
	\item The shape of the integral tritium $\upbeta$ spectrum depends the strongest on the neutrino mass near its endpoint $E_0$ \eqref{eq:endpoint}. \todo{Ask Hendrik for plot.}
	\item Measurements deeper into the spectrum increase the count rate and hence, lower the statistical uncertainty due to Poisson statistics.
	\item The theoretical description of the integral tritium $\upbeta$ spectrum is optimized for the endpoint region. Deeper scans introduce modeling uncertainties.
\end{enumerate}
The KATRIN Design Report \cite{Angrik:2005ep} suggests 5 \gls{mtd}s for different measurement ranges $[E_0-\alpha\;\SI{}{eV}, E_0 + \SI{5}{eV}]$ with $\alpha \in \{20, 25, 30, 40, 50\}$ and the conclusion that $\alpha=30$ works best for the KATRIN experiment.

Furthermore, searches for sterile neutrinos at the keV-scale would require deeper scans. Also, during commissioning measurement it makes sense to vary the \gls{mtd}, especially to perform deeper scans, to ensure a comprehensive understanding of the KATRIN apparatus.

Several measurement campaigns were already conducted. The \gls{ft} commissioning campaign successfully proved the apparatus functioning. The corresponding \gls{mtd} covered a range starting at $\sim E_0-\SI{1.6}{keV}$. The \gls{knm1} campaign is evaluated during the writing of this thesis. It set out to establish an unprecedented limit on the neutrino mass by $\upbeta$ decay measurements. Its \gls{mtd} starts at $\sim E_0-\SI{90}{eV}$.