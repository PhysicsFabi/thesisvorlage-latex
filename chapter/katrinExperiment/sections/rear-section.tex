\section{Rear Section}
\label{sec:katrinExpSetupRearSection}
\begin{figure}[t]
    \inputpdftex{\currentFigureFolder/rear-section}
   	\xcaption{KATRIN rear section}{The rear section}{terminates the KATRIN beam line and houses several monitoring and calibration devices that are described in the main text. (Adapted from \cite{SeitzM2019}.)}
 \label{fig:rearSection}
\end{figure}

The rear section terminates the beam line in the upstream direction and houses monitoring, calibration and control devices. It is depicted in figure \ref{fig:rearSection} and a detailed description can e.\,g.~be found in \cite{Babutzka2014}. In the following the major features of the rear section are reviewed:

{\par\textbf{Electron gun:}
The rear section houses an electron gun in order to measure the response function of the experiment (see section \ref{sec:response}) via a electron source with a well-defined energy resolution of $\sim \SI{0.2}{eV}$ and angular resolution of $\sim \SI{4}{\degree}$. The electrons are guided towards and through the rear wall by a designated electromagnetic guidance system. Furthermore, their flight path can be adjusted by dipole magnets mounted in the \gls{wgts} which enables a scanning of the full flux tube~\cite{Babutzka2014}.}

{\par \textbf{Rear wall and plasma control:}
The so-called rear wall is a gold-coated stainless-steel disc with a diameter of 6 inches that terminates the beam tube. It has a hole in the center to let electrons from the electron gun pass through. Its main purpose is the control of plasma effects: Space charges, respectively a plasma, forms within the \gls{wgts} due to the tritium decay. Therefore, $\upbeta$ electrons may start at different potentials which adds uncertainty to the measured $\upbeta$ spectrum. Simulations show that the plasma can be influenced by the rear wall potential which can be controlled via a voltage supply in the range of $\pm \SI{10}{V}$. Moreover, a UV light illumination of the rear wall can extract electrons via the photoelectric effect that can compensate space charges. Therefore, a homogeneous work function of the rear wall with fluctuations less than \SI{20}{meV} is required~\cite{Kuckert2018, Kuckert2016}.}

{\par\textbf{Activity monitoring:}
A super conducting coil designed to create a magnetic field of \SI{4.7}{T} in the rear section ensures that the magnetic flux tube terminates at the rear wall. Hence, per design of the magnetic guidance, $\upbeta$ electrons either arrive at the detector or hit the rear wall. On hitting the rear wall they emit bremsstrahlung. Two dedicated \gls{bixs} systems measure the corresponding X-ray spectrum to determine the source strength respectively the gas column density, equation~\eqref{eq:columnDensity}~\cite{Roellig2015}.}
