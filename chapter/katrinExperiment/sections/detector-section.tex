\section{Detector Section}
\label{sec:katrinExpSetupDetector}
\begin{figure}[t]
    \inputpdftex{\currentFigureFolder/detector-section}
   	\xcaption{KATRIN detector section}{The detector section}{terminates the KATRIN beam line. Among other instruments it houses the \glsentryfull{fpd} for $\upbeta$ electrons with the detector wafer at its core. For an explanation of the other components the reader is referred to the main text. (Adapted from \cite{SeitzM2019}.)}
 \label{fig:katrinExpSetupDetector}
\end{figure}

The detector section terminates the beam line in downstream direction. It can be separated from the spectrometer section by closing a gate valve. The detector section is depicted in figure \ref{fig:katrinExpSetupDetector} and a detailed description can e.g. be found in \cite{Amsbaugh2015}. In the following the major features of the detector section are reviewed:

{\par \textbf{\Gls{fpd}}: The \gls{fpd} counts the $\upbeta$ electrons that pass the spectrometer section. It is a \textit{pin}-silicon detector with a sensitive area of \SI{9}{cm} diameter. It is subdivided in 148 pixels of the same area arranged in 12 rings of 12 pixels each and the so called bull's eye of 4 pixels in the center. This arrangement allows later correction for radial electrical, magnetic and gas dynamical inhomogeneities in the beam line~\cite{Amsbaugh2015}.}

{\par \textbf{Shield and veto system:} The radiation shield of the \gls{fpd} system consists of two nested cylindrical shells: an outer lead shell of \SI{3}{cm} that reduces photon background and an inner copper shell of \SI{1.27}{cm} that blocks X-rays originating from the outer lead shell. The shield is surrounded by a veto system to tag incoming muons. Such a system is necessary to keep the strict background requirements~\cite{Amsbaugh2015}.}

{\par \textbf{Calibration:} Photoelectron sources can be lowered in the line of sight of the detector. The corresponding photocurrent can be measured with the \gls{pulcinella} system. A comparison of \gls{pulcinella} and the \gls{fpd} yields the \gls{fpd}'s detection efficiency. It was determined to be $\epsilon_\mathrm{det}=95\pm1.8\pm2.2\,\text{\%}$~\cite{Amsbaugh2015}.}

{\par \textbf{Detector magnet:} The detector magnet ($B_\mathrm{D}=\SI{3.6}{T}$) allows to form the flux tube near the detector independently of the main spectrometer magnetic field setting. It especially allows its mapping on the the detector~\cite{Amsbaugh2015}.}

{\par \textbf{Post-acceleration electrode:} The post-acceleration potential shifts the electrons arriving from the main spectrometer to a more favorable energy region. This increases the detector efficiency and, additionally, $\upbeta$ electrons can be distinguished from noise originating in the detector by an energy region of interest cut. An appropriate setting was found to be~$\sim\SI{10}{keV}$~\cite{Amsbaugh2015}.}