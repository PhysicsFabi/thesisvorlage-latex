\section{Differential Pumping Section}
\label{sec:katrinExpSetupDiffPumpingSection}
\begin{figure}[t]
    \inputpdftex{\currentFigureFolder/dps}
   	\xcaption{KATRIN \glsentryfull{dps}}{The \glsentryfull{dps}}{reduces the gas flow by five orders of magnitude and blocks tritium ions. Its five elements, each with a separate magnet (M1-M5), and connected by pump ports, are shown. The outer loop connects the turbo molecular pumps to the infrastructure of the \gls{tlk}. (Adapted from \cite{SeitzM2019}.)}
 \label{fig:katrinExpSetupDiffPumpingSection}
\end{figure}
The \glsentryfull{dps} is composed of five elements. It is depicted in figure \ref{fig:katrinExpSetupDiffPumpingSection} and a detailed description can e.\,g.~be found in~\cite{Kosmider2012}. For orientation, in this section, the elements are labeled 1 to 5 from \gls{wgts} to \gls{cps}. The major features of the \gls{dps} are reviewed in the following:

{\par\textbf{Reduction of tritium flow:}
The five beam tube elements of the \glsentryfull{dps} form a \SI{20}{\degree} angle to each other and are arranged in a chicane. $\upbeta$~electrons are magnetically guided along the chicane by a magnetic field of up to \SI{5.5}{T} created by five superconducting solenoids. By contrast, the neutral gas molecules scatter off the walls. This reduces the molecular beaming effect and enhances the pumping probability~\cite{ZHANG2012}. Four turbo molecular pumps mounted between the beam tube elements then reduce the gas flow by approximately five orders of magnitude and feed the gas into the so-called outer loop where it is reprocessed~\cite{Kosmider2012}.}

{\par\textbf{Ion blocking:}
In the \gls{wgts}, ions such as \ce{HeT^+}, \ce{T_2+}, \ce{T_3+}, \ce{T_5+} can form. If they were not blocked, they would reach the spectrometer section together with the $\upbeta$~electrons and would be even accelerated by the retarding voltage (see section~\ref{sec:katrinExpSetupSpectrometer}). This would eventually lead to an increased background rate. A potential barrier created by two ring electrodes in element 5 and the pump port between \gls{dps} and \gls{cps} set to $+\SI{100}{V}$ avoids such a scenario. The positive ions are deflected, and dipole electrodes in the elements 1 to 4 make them drift out of the flux tube. They hit the wall and get neutralized~\cite{Klein2019}.}

{\par\textbf{Ion monitoring:}
Downstream of the blocking electrodes, the remaining ion flux is measured by a Fourier transform ion cyclotron resonance device (FT-ICR)~\cite{Ubieto2009}.}
