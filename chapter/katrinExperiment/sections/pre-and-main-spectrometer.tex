\section{Pre and Main Spectrometer}
\label{sec:katrinExpSetupSpectrometer}
The pre and main spectrometer are vacuum vessels designed to filter passing electrons according to their kinetic energy. The pre spectrometer has a length of \SI{3.4}{m} and a diameter of \SI{1.7}{m}. Details on its design can e.\,g.~be found in \cite{Valerius2009,Fraenkle2010}. The main spectrometer has a length of \SI{23}{m} and a diameter of \SI{10}{m}. Details on its design can e.\,g.~be found in \cite{Valerius2009, Valerius2004}. The functionality and purpose of spectrometer related aspects are reviewed in this section. The content is divided into two parts: Section \ref{sec:katrinExpSetupSpectrometerMACE} explains the so-called MAC-E filter principle and section \ref{sec:katrinExpSetupSpectrometerBGCounterMeasures} list several measures to keep the strict KATRIN background requirements.

\subsection{MAC-E-filter principle}
\label{sec:katrinExpSetupSpectrometerMACE}
\begin{figure}[t]
	\inputpdftex{\currentFigureFolder/mace}
	\xcaption{Scheme of the KATRIN main spectrometer and the \glsentryshort{mace} filter principle.}{
		Scheme of the KATRIN main spectrometer and the \glsentryfull{mace} filter principle.}{
		The {KATRIN} design magnetic field settings are
		$\Bps=\SI{4.5}{T}$, 
		$\Bsource=\SI{3.6}{T}$, 
		$\Bmax=\SI{6.0}{T}$, 
		$B_\mathrm{D}=\SI{3.6}{T}$, 
		$\Bana\approx\SI{3e-4}{T}$. $\Vec{E}$ denotes the magnetic field regulated by the retarding potential $U$ that reaches its maximum $U_\mathrm{a} = U$ at the analyzing plane. (Adapted from \cite{SeitzM2019}.)
	}
	\label{fig:katrinExpSetupSpectrometer}
\end{figure}
The pre and main spectrometer are based on the principle of the so-called \gls{mace}~\cite{Beamson1980}. It enables the filtering of electrons according to their kinetic energy. Figure~\ref{fig:katrinExpSetupSpectrometer} sketches the \gls{mace} filter of KATRIN. The following paragraphs outline the basic concepts and their experimental implementation. As the principle is of key importance for the KATRIN experiment, it is additionally treated in a mathematical way in the subsequent section~\ref{sec:intSpecModelResponseTransmission}.

{\par \textbf{Electrostatic filtering:}
A retarding voltage barrier is applied along the beam axis within the spectrometer, reaching its maximum $U$ at the so-called analyzing plane in the center and dropping off towards the source and the detector. The retarding voltage barrier deflects electrons with kinetic energies below $eU$. For higher energies it depends on the electron's angle with respect to the beam line axis whether it can pass the spectrometer or not, see equation~\eqref{eq:intSpecModelTransmission}. 

{\par \textbf{Magnetic collimation}: 
The electric field gradient of the retarding voltage barrier is parallel to the beam line, but $\upbeta$ electrons are emitted in an arbitrary angle with respect to the magnetic field lines. In order to analyze their full kinetic energy, they have to be collimated. This is achieved by a magnetic field gradient that drops from $\Bsource=\SI{3.6}{T}$ in the \gls{sts} to $\Bana\approx\SI{3e-4}{T}$ in the analyzing plane. In the following a plausibility argument for the momentum collimation due to the field gradient, according to~\cite{Angrik:2005ep}, is given: Electrons entering the spectrometer vessel perform cyclotron motions around the magnetic field lines. Their total kinetic energy~$\Esource$ is split into a longitudinal component~$E_\parallel$ along the beam axis and a transverse component~$E_\bot$
\begin{equation}
\label{eq:totalKinElecEnergy}
\Esource = E_\parallel + E_\bot \fullstop
\end{equation}
In the non-relativistic and adiabatic approximation the transverse component can be expressed by the magnetic field strength~$B$ and the electron's magnetic moment~$\mu$ respectively its charge~$q=e$, its mass~$m_\elecIndex$ and angular momentum~$L$~\cite{jackson1975classical}
\begin{equation}
E_\bot = - \mu B = \frac{e}{2 m_\elecIndex}LB \fullstop
\end{equation}
Adiabaticity conserves angular momentum $L$ and the total energy of the electron $\Esource$ along its trajectory. Hence, when the magnetic field strength $B$ decreases to $\Bana=\Bmin$ in the analyzing plane, the transverse component of the electron's energy $E_\bot$ decreases likewise and transforms to longitudinal energy $E_\parallel$.}

{\par \textbf{Magnetic Bottle effect:}
As the source is placed in a lower magnetic field $\Bsource$ compared to the maximum field strength along the beam line $\Bmax$ at the detector side, $\upbeta$ electrons traveling downstream to the detector are subject to the magnetic bottle effect~\cite{Angrik:2005ep}. They get reflected and travel upstream to the rear wall if their starting angle $\thetaSource$ with respect to the beam line axis surpasses $\thetaMax$ with
\begin{equation}
\label{eq:thetaMax}
\sin\thetaMax = \sqrt{\frac{\Bsource}{\Bmax}} 
\fullstop
\end{equation}
For the KATRIN design values $\Bmax=\SI{6}{T}$ and $\Bsource=\SI{3.6}{T}$ one obtains $\thetaMax\approx\SI{51}{\degree}$.
A cutting angle $\thetaMax$ is beneficial because the greater the emission angle of a $\upbeta$ electron the larger the distance it travels in the \gls{wgts} and the more it is subject to energy losses such as scattering or synchrotron radiation~\cite{Angrik:2005ep}.}

{\par \textbf{\gls{mace}-filter width:} 
Electrons with a kinetic energy below $qU$ cannot pass the spectrometer. Electrons with a kinetic energy above $qU+\Delta E$ do pass the spectrometer. Here, $\Delta E$ denotes the filter width \cite{Angrik:2005ep}
\begin{equation}
\label{eq:katrinExpSetupFilterWidth}
\Delta E = \frac{\Bana}{\Bmax} E
\fullstop
\end{equation}
Electrons with an energy between $eU$ and $eU+\Delta E$ pass the potential barrier only with a certain probability. A quantitative description of this so-called transmission probability is given in the subsequent section~\ref{sec:intSpecModelResponseTransmission}. However, it can already be deduced, that a larger $\Delta E$ adds a greater uncertainty to the measurement and thus it should be kept as low as possible. $\Delta E$ depends on the maximum magnetic field strength along the beam line $\Bmax=\SI{6}{T}$, the kinetic energy of $\upbeta$ electrons $E\approx\SI{18.6}{keV}$, and the field in the magnetic field in the analyzing plane $\Bana\approx\SI{3e-4}{T}$. Hence, its KATRIN design value is $\Delta E\approx\SI{0.93}{eV}$. }

{\par \textbf{Dimensions of the KATRIN main spectrometer:} This paragraph outlines, why the diameter of the KATRIN main spectrometer is \SI{10}{m}, while the one of its predecessor experiment in Mainz was only \SI{1}{m}~\cite{Kraus2005}. KATRIN's envisaged sensitivity requires a relative \gls{mace}-filter width of at least $\Delta E/E = 1/20000$, which directly corresponds the ratio of the magnetic fields $\Bmax/\Bana$~(see equation~\eqref{eq:katrinExpSetupFilterWidth}). For a smaller $\Delta E$, $\Bana$ should be chosen as low as possible. However, the lower $\Bana$, the wider the flux tube that must be governed by the spectrometer vessel. Also, the magnetic field must decrease at a sufficiently slow rate from the spectrometer's entrance to the analyzing plane in order to guarantee adiabaticity, which requires a certain spectrometer length. Dimensions that meet the demands and are feasible for the main spectrometer were found to be a radius of \SI{10}{m} and a length of \SI{23}{m}~\cite{Angrik:2005ep, Valerius2004}.}

Now, that the requirements on the magnetic and electrostatic fields are outlined, the following two paragraphs review their technical implementation:

{\par \textbf{Magnetic field:} The main spectrometer is surrounded by a system of coils that shapes the \gls{mace} filter's  magnetic field. Upstream, there is the PS2 magnet ($\Bmax=\SI{4.5}{T}$); downstream the pinch ($\Bpinch=\Bmax=\SI{6.0}{T}$) as well as the detector magnet ($B_\mathrm{D}=\SI{3.6}{T}$), which are superconducting solenoids. The field is fine-tuned by a system of air coils around the spectrometer hull: There is the \gls{emcs} with 26 current loops parallel to the beam line axis. Furthermore, there is the \gls{lfcs} with 14 air coils perpendicular to the beam line axis. The combined system constrains the electrons' flux tube to the spectrometer vessel and compensates the Earth's magnetic field as well as effects from ferromagnetic materials in the spectrometer's surroundings~\cite{Erhard2018}. Additionally, a vertical and radial magnetic measuring system (\glsentryshort{vmms} and \glsentryshort{rmms}) are installed outside the spectrometer vessel. The field inside the spectrometer vessel is assessed via samples of these measuring systems combined with simulations~\cite{Letnev2018}.}

{\par \textbf{Electrostatic field:} A high-voltage system establishes the \gls{mace} filter's retarding potential. The fluctuation of the retarding voltages must have a standard deviation smaller than \SI{60}{mV} for the envisaged sensitivity on the neutrino mass~\cite{Angrik:2005ep}. The antenna-like beam line setup is sensitive to electromagnetic fluctuations of any source, which is why an active post-regulation system is deployed. It monitors the retarding potential and regulates it with the required precision. For the monitoring the monitor spectrometer and a voltage divider are deployed. For details on the later systems the reader is referred to~\cite{Thuemmler2009,Erhard2014,Zboril2011}.}

\subsection{Background Mitigation Strategies}
\label{sec:katrinExpSetupSpectrometerBGCounterMeasures}
The KATRIN sensitivity goal requires a background rate of less than \SI{10}{mcps}~\cite{Angrik:2005ep}. Several background-related aspects with respect to the spectrometer tanks are:

{\par \textbf{Vacuum:} The spectrometers are operated at a pressure on the order of \SI{1e-11}{mbar}. This prevents electron scattering on residual gas and minimizes background effects by ionization. Correspondingly, turbo molecular and getter pumps are installed at three pump ports of the spectrometer vessels. Furthermore, the spectrometers can be baked out at up to \SI{350}{\celsius}~\cite{Arenz2016}.}
	
{\par \textbf{Wire electrodes:} The inner walls of the spectrometer vessels are lined by wire electrodes. Their potential is at a few hundred volts more negative than the spectrometer hull reflecting electrons coming from the vessel walls. Such electrons may be induced by cosmic rays~\cite{Valerius2009}.}
	
{\par \textbf{Ion blocking:} Analogously to the ones in the \gls{cps} (section \ref{sec:katrinExpSetupCryoPumpingSection}), three blocking electrodes are installed; one between the \gls{cps} and the pre spectrometer, one between the pre and main spectrometer; and one between the main spectrometer and the detector~\cite{Klein2019}.}
	
{\par \textbf{Tandem setup:} $\upbeta$ electrons may scatter on residual gas. This can either directly lead to secondary electrons or create positive ions that travel down the beam line. The positive ions in turn may again yield secondary electrons through scattering. The more $\upbeta$ electrons enter the main spectrometer, the higher is the probability to create secondary electrons. In order to reduce the flux of $\upbeta$ electrons into the main spectrometer, the retarding potential of the pre spectrometer is set to a few hundred volts more positive than the one of the main spectrometer. On the one hand this is a countermeasure against background events, but on the other hand, charged particles can be trapped between the two spectrometers due to the electromagnetic setup (Penning trap). A sudden discharge may harm the hardware, especially the detector. Therefore, it is possible to sweep a charged wire through the volume in order to collect the trapped particles and avoid this ``Penning discharges''~\cite{Valerius2009}.}