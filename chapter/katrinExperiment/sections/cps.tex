\section{Cryogenic Pumping Section}
\label{sec:katrinExpSetupCryoPumpingSection}
\begin{figure}[t]
    \inputpdftex{\currentFigureFolder/cps}
   	\xcaption{KATRIN \glsentryfull{cps}}{The \glsentryfull{cps}}{is the coldest part of the KATRIN experiment. It consists of seven elements, labeled by 1 to 7 from the \gls{dps} to the pre spectrometer. Elements 2 to 5 are covered by a frozen argon layer at~\SI{3}{K} in order to cold-trap tritium molecules. The low temperatures are established using liquid helium (\ce{LHe}) and an insulation of liquid nitrogen (\ce{LN2}). Each element is enclosed by a super conducting coil (M1 to M7) for magnetic guidance of the $\upbeta$~electrons. For the \glsentryshort{fbm} and the \glsentryshort{ckrs} the reader is referred to the main text. (Adapted from \cite{SeitzM2019}.)}
 \label{fig:katrinExpSetupCryoPumpingSection}
\end{figure}
The \glsentryfull{cps} is an approximately 7-m-long cryostat. It is depicted in figure \ref{fig:katrinExpSetupCryoPumpingSection} and a detailed description can e.\,g.~be found in \cite{Jansen2015}. For orientation, in this section, its seven elements are labeled by 1 to 7 from \gls{wgts} to \gls{cps}. The major features of the \gls{cps} are reviewed in the following:

{\par\textbf{Reduction of tritium flow:}
The \gls{cps} consists of seven beam tube elements, of which the first five are arranged in a chicane forming~\SI{15}{\degree} angles, in a similar manner as the beam tube elements of the \gls{dps}. Charged particles are guided along the chicane by a magnetic field of up to~\SI{5.6}{T} created by seven superconducting coils. Neutral molecules hit the walls, that are covered by a frozen argon layer cooled down to~\SI{3}{K} in order to cold-trap particles. These low temperatures are achieved via liquid helium cooling and a heat shield of liquid neon. After the accumulation of about~\SI{1}{Ci} of tritium, the argon frost layer has to be renewed. In order to achieve this, the beam tube is warmed up and the argon is pumped off along with the accumulated tritium. Tests and simulations show a reduction of the tritium flow by approximately 10 orders of magnitude from the entrance to the exit of the \gls{cps}~\cite{Jansen2015,Roettele2019}.}

{\par\textbf{The \glsentryfull{fbm}}: 
The \glsentryshort{fbm} can be moved horizontally into the pump port between beam tube element 6 and 7 of the \gls{cps} with a 2-dimensional spatial resolution of \SI{0.1}{mm}. Two \textit{pin}-diodes measure the $\upbeta$-electron flux and thus the stability of the gas column density in the \gls{wgts}. Furthermore, the \gls{fbm} equips a temperature and a hall sensor. A second detector board holding a Faraday cup for ion measurements is also available~\cite{Klein2019}. More information about the \gls{fbm} can e.\,g.~be found in \cite{Ellinger2017,Ellinger2019}.}

{\par\textbf{The \glsentryfull{ckrs}:} The \gls{ckrs} is a sub mono-layer of \kryptonEightyThree{} on a pyrolytic graphite substrate with a diameter of \SI{2}{cm}. It can be lowered in the pump port of the \gls{cps} and moved in a 2-dimensional plane perpendicular to the beam line. This enables the spatial scanning of the properties of the spectrometer using quasi-monoenergetic conversion electron lines of \kryptonEightyThree~\cite{Bauer2014, Dyba2019, Arenz2018Kr}.}
