The KArlsruhe TRItium Neutrino (KATRIN) experiment performs a kinematic measurement of the tritium-$\upbeta$ spectrum in order to determine the effective mass of the electron antineutrino (from here forth labeled $m_\upnu$) as defined by equation (\ref{eq:nuMassSquared}). In case no neutrino mass signal is observed, KATRIN aims to set an upper limit of
\begin{equation*}
m_\upnu < \SI{200}{meV} \quad (\SI{90}{\percent} \text{ C.L.})
\comma
\end{equation*}
which is one order of magnitude lower than the one set by its predecessor experiments.
KATRIN recorded the first $\upbeta$ spectrum in March 2018 and started neutrino mass measurements in March 2019.

This chapter provides an overview of the KATRIN apparatus. However, given KATRIN's complexity, it can by no means be exhaustive and for a comprehensive treatise the reader is referred to the KATRIN Design Report~\cite{Angrik:2005ep} supplemented by an up-to-date hardware overview that is in the making at the time of writing this thesis\footnote{What to I put here?} \todo{Proper Footnote}.