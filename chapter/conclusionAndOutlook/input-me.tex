\chapter{Summary}
\label{sec:conclusion}
In the scope of this thesis, two aspects concerning the inelastic electron scattering off gas molecules in the \gls{wgts} were investigated:

\paragraph{Energy-dependence of the Inelastic Scattering Cross Section}
The main finding was that neglecting the energy-dependence of the scattering cross section in neutrino mass inference could lead to a systematic bias of the estimated squared neutrino mass of
\begin{equation*}
\Delta\nuMass^2 = \SI{1.09e-2}{eV^2}
\end{equation*}
when using the design KATRIN analysis interval of \SI{30}{eV}. For further findings, the reader is referred to the conclusion and outlook section~\ref{sec:eDepScatCrossSecConclusion}.


\paragraph{Statistical Methods and the KATRIN Energy Loss Model}
The main achievement was the transfer of statistical methods into the context of the KATRIN experiment to enable the treatment of a high-dimensional model of a KATRIN measurement in neutrino mass inference. The statistical tools were successfully applied to assess the impact on KATRIN's sensitivity to the neutrino mass from the uncertainties of the 15 additional parameters of the preliminary KATRIN energy loss model. They increase the statistical uncertainty (\SI{68}{\percent} C.L.) of the squared neutrino mass by
\begin{equation*}
\Delta \sigma_\mathrm{tot}(\nuMass^2) = \SI{1.2e-4}{eV^2}
\fullstop 
\end{equation*} 
This result is, however, preliminary as the KATRIN energy loss model is not yet final. For further findings, the reader is referred to the conclusion and outlook section~\ref{sec:katrinElossModelOutlook}.
