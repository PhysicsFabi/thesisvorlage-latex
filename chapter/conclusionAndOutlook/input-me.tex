\chapter{Conclusion}
\label{sec:conclusion}
In the scope of this thesis, two aspects concerning the inelastic scattering of electrons in the \gls{wgts} were investigated:

\paragraph{Energy-dependence of the Inelastic Scattering Cross Section}
In section~\ref{sec:eDepScatCrossSecModel}, models that incorporate the dependence of the inelastic scattering cross section on the energy of incident electrons were presented. An approximation was used when modeling the probability for severalfold scattering via a Poisson distribution. Whether this approximation needs refinement was not ultimately clarified.

Using the approximated model, strong indications were found, that the energy-dependence should not be neglected in neutrino mass inference as this would shift the inferred neutrino mass by 	$\Delta\nuMass^2 = \SI{1.09e-2}{eV^2}$ when using the design KATRIN analysis interval of \SI{30}{eV}. However, an independent cross-check of the suggested models is recommended.

\paragraph{Statistical Methods and the KATRIN Energy Loss Model}
The profile-likelihood-method and the usage of an Asimov data set were reviewed in the context of uncertainty propagation from the KATRIN energy loss model to neutrino mass inference. Both concepts are only applicable in the large sample limit. An ensemble test showed that this limit is given. It was found that the current uncertainties on the KATRIN energy loss model do not influence KATRIN's sensitivity significantly. This result is, however, preliminary as the KATRIN energy loss model is not yet final. Nonetheless, the study may serve as a proof-of-concept and is expected to be repeatable with less effort, as the corresponding statistical tools were implemented in the KaFit software framework. 
