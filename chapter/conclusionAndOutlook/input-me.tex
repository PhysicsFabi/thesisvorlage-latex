\chapter{Summary}
\label{sec:conclusion}
In the scope of this thesis, two aspects concerning the inelastic electron scattering off gas molecules in the \gls{wgts} were investigated:

\paragraph{Energy-dependence of the Inelastic Scattering Cross Section}
A model for electron-scattering in the \gls{wgts} was established in dependence of the incident electron energy. The main finding was that, if the energy-dependence of the scattering cross section in neutrino mass inference were neglected, it could lead to a systematic bias of the estimated squared neutrino mass of $\Delta\nuMass^2 = \SI{1.09e-2}{eV^2}$ when using the design KATRIN analysis interval of~\SI{30}{eV} below the endpoint of the $\upbeta$ spectrum. However, the effect does not have to be neglected, but can be incorporated in the analysis using the model that was established. It should be mentioned, that certain approximations were made in the modeling process. For details and for further findings, the reader is referred to the conclusion and outlook section~\ref{sec:eDepScatCrossSecConclusion}.

\paragraph{Statistical Methods and the KATRIN Energy Loss Model}
The main achievement is the transfer of statistical methods into the context of the KATRIN experiment to enable the treatment of a high-dimensional model of a KATRIN measurement in neutrino mass inference. The statistical tools were applied to assess whether systematic effects stemming from the energy loss function can be reduced by additional free fit parameters. The reduction was on the order of $\SI{e-4}{eV^2}$ or less for the inferred squared neutrino mass, which is not significant given the systematic budget of $\SI{6e-3}{eV^2}$ allocated for the energy loss function by the KATRIN Design Report.
For further findings, the reader is referred to the conclusion section~\ref{sec:katrinElossModelOutlook}.