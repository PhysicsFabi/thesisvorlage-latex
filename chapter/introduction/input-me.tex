\chapter{Introduction}
To our current knowledge, neutrinos are at the same time the most elusive and most abundant massive particles in the Universe. Their detection is challenging and pushes experiments at the edge of technical frontiers. Yet, their understanding might shed light on long standing open questions of modern physics: What is dark matter? Do Majorana particles exist? Why does matter predominate antimatter? What happened in the earliest stages of of our Universe just after the Big Bang? And in all these regards, the yet unknown mass of neutrinos is a key physics parameter.

The KArlsruhe TRItium Neutrino (KATRIN) experiment aims to measure the effective mass of the electron antineutrino with an unprecedented sensitivity of \mbox{\SI{200}{meV} (\SI{90}{\percent} C.L.)} based on tritium-$\upbeta$ decay. In order to provide this outstanding sensitivity KATRIN features i.\,a.~a gaseous tritium source. Its special characteristics must be well controlled and understood. This thesis focuses on selected effects stemming from electrons scattering in said gaseous source. The effects were included in a high level analysis, meaning their impact on KATRIN's neutrino mass sensitivity was studied where possible.

\paragraph{Outline}
Chapter \ref{sec:neutrinoPhysics} is a brief introduction to neutrino physics with special emphasis on the neutrino mass.

Chapter \ref{sec:katrin} focuses on the setup of the KATRIN experiment and introduces a mathematical model of a KATRIN neutrino mass measurement.

Chapter \ref{sec:statMethods} integrates the mathematical model into a statistical framework for a high level analysis, especially for neutrino mass inference.

Chapter \ref{sec:energyDepScat} investigates a refinement of the mathematical model with regard to electrons scattering from gas molecules within KATRIN's gaseous source. Namely, the dependence of the inelastic scattering cross section on the energy of the incident electrons is studied.

Chapter \ref{sec:katrinEloss} focuses on a preliminary model for the energy loss of electrons scattering inelastically from deuterium molecules. The model was established by a dedicated subgroup of the KATRIN collaboration based on KARIN data from October 2018. Its relation to KATRIN's sensitivity on the neutrino mass is investigated.

Chapter \ref{sec:conclusion} summarizes the results, draws conclusions and offers an outlook.