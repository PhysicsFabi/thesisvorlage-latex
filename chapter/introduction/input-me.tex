\chapter{Introduction}
To our current knowledge, neutrinos are at the same time the most elusive and most abundant massive particles in the Universe. Their detection is challenging and pushes experiments to the edge of technical frontiers. Yet, their understanding might shed light on long-standing open questions of modern physics: What is dark matter? Do Majorana particles exist? Why does matter predominate over antimatter? What happened in the earliest stages of our Universe just after the Big Bang? And in all these regards, the yet unknown mass of neutrinos is a key physics parameter.

The KArlsruhe TRItium Neutrino (KATRIN) experiment aims to measure the effective mass of the electron antineutrino with an unprecedented sensitivity of \mbox{\SI{200}{meV} (\SI{90}{\percent} C.L.)} based on tritium-$\upbeta$ decay. 

In order to provide this outstanding sensitivity, KATRIN features i.\,a.~a gaseous tritium source. Its special characteristics must be well controlled and understood. This thesis focuses on two selected models concerning electron-scattering in said gaseous source: 
\begin{dingautolist}{172}
	\item\label{itm:introductionEDepCrossSec} The dependence of the cross section for inelastic electron-scattering off hydrogen isotoplogues on the energy of the incident electrons was investigated. Former works primarily used an energy-independent cross section when simulating KATRIN neutrino mass measurements. An energy-dependent scattering model was established and applied within a simulated neutrino mass inference. It was found, that the energy-dependence should not be neglected, when modeling a KATRIN neutrino mass measurement.
	\item\label{itm:introductionKatrinEloss} A dedicated subgroup of the KATRIN collaboration established a preliminary model of the probability density for the energy an electron loses when scattering in the KATRIN gaseous tritium source. The model is based on data taken at the KATRIN experiment in October 2018. In the scope of this thesis, statistical methods were implemented that enable the general treatment of model uncertainties within the formalism of the profile-likelihood. The methods were then applied to study the impact from the uncertainties of the preliminary energy loss model on KATRIN's sensitivity to the neutrino mass. The preliminary findings are that the uncertainties are of minor importance given the KATRIN systematic budget.
\end{dingautolist}

\paragraph{Outline}
This thesis is structured as follows:

Chapter \ref{sec:neutrinoPhysics} is a brief introduction to neutrino physics with special emphasis on the neutrino mass.

Chapter \ref{sec:katrinExpSetup} focuses on the setup of the KATRIN experiment.

Chapter \ref{sec:intSpecModel} introduces a mathematical model of a KATRIN neutrino mass measurement that can be used in parameter inference.

Chapter \ref{sec:statMethods} integrates the mathematical model into a statistical framework for neutrino mass inference.

Chapter \ref{sec:eDepScatCrossSec} investigates a refinement of the mathematical model by incorporating the energy-dependence of the cross section for inelastic electron scattering in the KATRIN gaseous tritium source as per item~\ref{itm:introductionEDepCrossSec}.

Chapter \ref{sec:katrinEloss} focuses on the propagation of uncertainties from a preliminary model for the energy loss of electrons scattering inelastically off deuterium molecules as per item~\ref{itm:introductionKatrinEloss}.

And chapter \ref{sec:conclusion} summarizes the results.