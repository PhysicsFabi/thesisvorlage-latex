\chapter{Introduction}
To our current knowledge, neutrinos are at the same time the most elusive and most abundant particles in the Universe. Their detection pushes experiments at the edge of technical frontiers. Yet, their understanding might shed light on long standing open questions of modern physics: What is dark matter? Do Majorana particles exist? Why does matter predominate antimatter? On top of that, the formation of the structure of the Universe is highly influenced by neutrinos. In all these regards, the yet unknown mass of neutrinos is a key physics parameter.

The KArlsruhe TRItium Neutrino (KATRIN) experiment aims to measure the mass of the electron antineutrino with an unprecedented sensitivity of \SI{200}{meV} (\SI{90}{\percent} C.L.) based on tritium $\upbeta$ decay. In order to provide this outstanding sensitivity KATRIN features i.\,a.~a gaseous tritium source. Its special characteristics must be well controlled and understood. This thesis focuses on selected effects stemming from $\upbeta$ electrons scattering in the gaseous source. The effects were included in a high level analysis, meaning their impact on KATRINs neutrino mass sensitivity was studied where possible.

\paragraph{Outline}
In section \ref{} a brief introduction to neutrino physics with special emphasis on the neutrino mass is given.

Section \ref{} focuses on the setup of the KATRIN experiment.

Section \ref{} introduces a mathematical model of a KATRIN neutrino mass measurement.

Section \ref{} integrates the model into a statistical framework that can be used in a high level analysis, especially for neutrino mass inference.

Section \ref{} introduces the two main software packages used in the scope of this thesis. First, there is the so-called source and spectrum calculation (SSC) package. It implements the analytical model derived in section \ref{}. Second, there is the so-called KaFit package. It implements the statistical framework from section \ref{}.

Section \ref{} proposes a refinement of the mathematical model from section \ref{} with regard to $\upbeta$ electrons scattering from gas molecules within the said gaseous source. Namely, the dependence of the inelastic scattering cross section on the energy of the incident electrons is investigated.

Section \ref{} revisits the statistical framework of KaFit and proposes an extension that allows e.\,g.~the combination of a KATRIN neutrino mass and a commissioning measurement.

Section \ref{} shows an application of the proposed methods. Based on measurements from October 2018 a preliminary model for the energy loss of electrons scattering from tritium molecules was established by a dedicated KATRIN subgroup. How its precision relates to KATRIN's sensitivity to the neutrino mass is investigated.

Section \ref{} summarizes the results, draws conclusions and offers an outlook.