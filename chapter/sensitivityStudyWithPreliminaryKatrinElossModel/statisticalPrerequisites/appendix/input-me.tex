\section{Documentation of KaFit-Likelihood Extensions}
\label{sec:appendixKatrinElossStatisticsLikelihoodExtKaFitConfig}
Within this thesis the implementation of the likelihood in the KaFit framework was extended as described in section \ref{sec:katrinElossStatisticsCombMeasurements}. The possibility was introduced to multiply the likelihood by different function types that resemble term $(2)$ in equation~\eqref{eq:katrinElossStatisticsPullTerm}. 

KaFit is configured using an \texttt{XML}-like syntax. Example excerpts from KaFit-\texttt{XML}-configurations are given below. The documentation is given to a level that enables the usage by a user already familiar with KaFit. {\color{brown}\texttt{[Double]}} is used as a placeholder for a number; and {\color{brown}\texttt{[Index*]}} for a parameter index. For example, the squared neutrino mass has the parameter index 0. A new \texttt{Penalty}-tag was introduced as sub-tag of the already established \texttt{LoglikelihoodKatrin}-tag:
\begin{lstlisting}[language=XML]
<LoglikelihoodKatrin 
    Name="myKatrinLogL" PDF="Gauss" RunSource="myRunGen" 
    SpectrumSimulator="mySpecSim">
  <Penalty>
      <!-- Penalty Type -->
  </Penalty>
</LoglikelihoodKatrin>
\end{lstlisting}
{\color{gray}\texttt{<!-- Penalty Type -->}} can be substituted by one ore more of the following tags:
\paragraph{Multivariate Normal Distribution}
The {\color{violet}\texttt{Parameter}}-tag represents a parameter the likelihood depends on. Its attribute {\color{magenta}\texttt{Mean}} specifies the mean of a parameter; {\color{magenta}\texttt{Std}} the standard deviation; and one ore more {\color{violet}\texttt{Correlation}}-sub-tags the correlations between the parameters.
\begin{lstlisting}[language=XML]
<MultivarNorm>
  <Parameter Index="[Index1]" Mean="[Double]" Std="[Double]" />
  <Parameter Index="[Index2]" Mean="[Double]" Std="[Double]" />
  <!-- ... -->
  <Correlation Index1="[Index1]" Index2="[Index2]" Value="[Double]" />
  <!-- ... -->
</MultivarNorm>
\end{lstlisting}

\paragraph{One-Dimensional Gaussian Distribution}
Analogously to the multivariate normal distribution, a one-dimensional Gaussian distribution can be used:
\begin{lstlisting}[language=XML]
<Gaussian ParamIndex="[Index]" Mean="[Double]" Std="[Double]" />
\end{lstlisting}

\paragraph{Uniform Neutrino Mass Prior}
A constant prior on the neutrino mass in a Bayesian analysis can be set via the following tag:
\begin{lstlisting}[language=XML]
<ConstInSqrt ParamIndex="0" />
\end{lstlisting}
The former implementation within KaFit only allowed for a constant prior on the squared neutrino mass. The following lines derive the form of a prior on $m_\nu^2$ that resembles a uniform prior on $m_\nu$. Let $f(m_\nu)=C=\mathrm{constant}$ be the prior on $m_\nu$ and $g(m_\nu^2)$ be the prior on $m_\nu^2$. Starting from conservation of probability one derives
\begin{align*}
f(m_\nu)\d m_\nu =& g(m_\nu^2) \d m_\nu^2 \\
\Rightarrow
g(m_\nu^2) =& f(m_\nu) \left( \frac{\d m_\nu^2}{\d m_\nu} \right)^{-1} \\
\Rightarrow
g(m_\nu^2) =& C \frac{1}{2\sqrt{m_\nu^2}}
\fullstop
\end{align*}