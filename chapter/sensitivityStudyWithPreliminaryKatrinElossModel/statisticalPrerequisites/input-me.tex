\def\currentRootFolder{chapter/sensitivityStudyWithPreliminaryKatrinElossModel/statisticalPrerequisites}
\def\currentFigureFolder{\currentRootFolder/fig}
\newcommand{\elecIndex}{\mathrm{e}}

\newcommand{\Bsource}{B^j_\mathrm{S}}
\newcommand{\BsourceAvg}{B_\mathrm{S}}
\newcommand{\zSource}{z_\mathrm{S}}
\newcommand{\thetaSource}{\theta_\mathrm{S}}
\newcommand{\thetaSourceAvg}{\theta_\mathrm{S}}
\newcommand{\Esource}{E_\mathrm{S}}
\newcommand{\Usource}{U^j_\mathrm{S}}
\newcommand{\gammaSource}{\gamma_\mathrm{S}}


\newcommand{\Bps}{B_\mathrm{PS2}}
\newcommand{\Bana}{B_\mathrm{A}}
\newcommand{\Bpinch}{B_\mathrm{P}}
\newcommand{\Bmax}{B_\mathrm{max}}
\newcommand{\Bmin}{B_\mathrm{min}}

\newcommand{\thetaMax}{\theta_\mathrm{max}}
\newcommand{\Esur}{E_\mathrm{sur}}
\newcommand{\detEff}{\epsilon_\mathrm{det}}
\newcommand{\macefilterwidth}{\Delta \mathcal{E}^j(\thetaS^j)}

\newcommand{\EtransPure}{E^j_\mathrm{tr}}
\newcommand{\Etrans}{\EtransPure(qU,\Esource,\thetaSource)}
\newcommand{\thetaTransPure}{\theta^j_\mathrm{tr}}
\newcommand{\thetaTrans}{\thetaTransPure(\Esource,qU)}

\newcommand{\As}{A_\mathrm{S}}
\newcommand{\Rbg}{R_\mathrm{bg}}


\newacronym{standardmodel}{SM}{Standard Model of Particle Physics}
\newacronym{lep}{LEP}{Large Electron Positron Collider}
\newacronym{ssm}{SSM}{standard solar model}

\section{Statistical Prerequisites}
\subsection{Sensitivity from  the Profile Likelihood Method}
In the scope of this thesis, the sensitivity on the neutrino mass is evaluated using the profile likelihood method (section~\ref{sec:statMethodsProfileLikelihood}) in order to account for nuisance parameters and especially their correlations. The profile likelihood method was also used in~\cite{Kleesiek2014} for a KATRIN standard 4-parameter fit (section~\ref{sec:statMethodsProfileLikelihood}). The obtained value for $\statUncert$ is plotted and can be extracted to be between \SIrange[range-phrase=--]{0.0155}{0.0165}{eV^2} which is in agreement with the result $\statUncert=\SI{0.0162}{eV^2}$ from ensemble testing (see table~\ref{tab:statMethodsSensitivityFromEnsembleTests}). ``In case of the standard 4 parameter fit, the confidence intervals calculated from ensemble tests are in very good agreement with alternative methods [...], including likelihood ratio intervals (profile likelihood method)'' Kleesiek, page 160, profile likelihood, optimized mtd, uncertainty on suqred mnu $0.01494 eV^2$.



\subsection{Combination of Commissioning and Neutrino Mass Measurements}
If two measurements share a set of parameters $\paramVecShared$, but have additionally an individual set of parameters $\paramVec_1$ and $\paramVec_2$ and different sets of observations a combined likelihood is given by the product of the single likelihoods $L_1$ and $L_2$
\begin{equation}
-2\ln L(\paramVecShared, \paramVec_1, \paramVec_2) =  
-2\ln L_1(\paramVecShared, \paramVec_1)
-2\ln L_2(\paramVecShared, \paramVec_2)
\fullstop
\end{equation}
In the case of KATRIN the first measurement could be sensitive to the neutrino mass whereas say the second measurement could have been a calibration using the electron gun and be sensitive to parameters of the response function \eqref{eq:SSCresponse}. Combining both likelihoods would incorporate the uncertainties on the parameters of the response function in the neutrino mass determination. Currently, no software framework exists that allows the construction of combined likelihoods of KATRIN neutrino mass and calibration measurements. Instead the following approximation can be made. The calibration measurement is evaluated independently and one obtains estimates $\hat{\paramVec}_\mathrm{s,2}$, and an estimated covariance matrix $\hat{V}_\mathrm{s,2}$ for all components of $\paramVecShared$ that the calibration measurement is sensitive to. These can in turn be used to approximate the likelihood $L_2$ at least in the dimension of $\paramVecShared$. A choice that stands to reason for the approximation of $L_2$ is a multivariate Gaussian distribution. For the purpose of parameter inference through minimization $-2\ln L_2$ needs only to be accurately approximated around its minimum. The choice of a multivariate Gaussian distribution corresponds a symmetric approximation of $-\ln L_2$ around its minimum by a parabola. The KATRIN likelihood for a combination of a neutrino mass and a calibration measurement then reads
\begin{equation}
\begin{split}
\label{eq:penalizedLikelihood}
-2\ln L(\paramVecShared, \paramVec_1, \paramVec_2) &\approx
-2\ln L^\prime(\paramVecShared, \paramVec_1) \\ &=
\underbrace{
	\chi^2(\paramVecShared, \paramVec_1)
	\vphantom{(\paramVecShared - \hat{\paramVec}_\mathrm{s,2})^{\mathsf{T}}}
}_{(1)}
+
\underbrace{
	(\paramVecShared - \hat{\paramVec}_\mathrm{s,2})^{\mathsf{T}}
	\hat{V}_\mathrm{s,2}^{-1}
	(\paramVecShared - \hat{\paramVec}_\mathrm{s,2})
}_{(2)} +\; 
\mathrm{ constants}\\ &=
\chi^2(\paramVecShared, \paramVec_1) 
-2\ln \mathcal{N}(\paramVecShared, \hat{\paramVec}_\mathrm{s,2}, \hat{V}_\mathrm{s,2}^{-1}) +
\mathrm{ constants}
\end{split}
\end{equation}
Here, $(1)$ is the chi-square expression \eqref{eq:katrinChi2} where the $\paramVecShared$ and $\paramVec_1$ can be written as one combined parameter vector $\paramVec$ for a neutrino mass measurement. And $(2)$ resembles the negative log likelihood of the calibration measurement approximated by a multivariate Gaussian distribution. Terms having a form like $(2)$ are also sometimes called ``pull terms'' or ``likelihood penalties''. In the minimization process they ``pull'' the parameters $\paramVecShared$ towards $\hat{\paramVec}_\mathrm{s,2}$ respectively ``penalize''/increase the negative log likelihood if $\paramVecShared$ and $\hat{\paramVec}_\mathrm{s,2}$ differ.


\newcommand{\CombLmax}{-2\ln L(\hat{\paramVec}_\mathrm{s}, \hat{\paramVec}_1)}
The chi-square term $(1)$ is a sum of $n$ standard normal distributed random variables. Hence, as discussed, a likelihood only composed of the chi-square term $(1)$ offers a goodness-of-fit criteria via the the Pearson chi-square statistic. Note that for the combined likelihood this criteria might not hold. Two special cases can be considered where the chi-square characteristics hold approximately: First, the neutrino mass measurement, term $(1)$, is not sensitive to the shared parameters $\d \chi^2(\paramVecShared, \paramVec_1) /\d \paramVecShared \approx 0$. Then the \gls{mle} for the shared parameters will match the \gls{mle} by the calibration measurement $\hat{\paramVec}_\mathrm{s} = \hat{\paramVec}_{\mathrm{s},2}$ and term $(2)$ will be 0. The combined likelihood evaluated at the \gls{mle} $\CombLmax$ then follows a chi-square distribution with $n-\dim\paramVec_1-\dim\paramVecShared$ degrees of freedom. Second, if the neutrino mass measurement is sensitive to some shared parameters $\d \chi^2(\paramVecShared, \paramVec_1) /\d \paramVecShared \neq 0$, then one might argue, that term $(2)$ evaluated at the \gls{mle} $\hat{\paramVec}_\mathrm{s} \neq \hat{\paramVec}_{\mathrm{s},2}$ is a sum of standard normal distributed random variables. If this holds, the combined likelihood evaluated at the \gls{mle} $\CombLmax$ follows a chi-square distribution with $n-\dim\paramVec_1$ degrees of freedom.


For example, a standard KATRIN 3-year neutrino mass measurement is not at all sensitive to parameters of the energy loss function \eqref{eq:nonAveragedResponse}. Hence, adding a corresponding term $(2)$ from a designated energy loss measurement will not influence the chi-square characteristics. However, a standard KATRIN neutrino mass measurement is even after a short measurement time sensitive to the gas column density \eqref{eq:columnDensity}. Adding a corresponding term $(2)$ from (a naturally more sensitive) monitoring measurement would influence the   


\todo{Add plots from ensemble test that proof statements.}

\subsection{Extension of the KaFit Software Framework}
\label{sec:statLikelihoodExtImpl}
The likelihood $L(\paramVec)$ can be multiplied by a function $g(\paramVec)$
\begin{equation}
\label{eq:likelihoodExtension}
-2\ln L^\prime(\paramVec) = -2\ln L(\paramVec) -2\ln g(\paramVec)
\fullstop
\end{equation}
This procedure may have different interpretations and usage scenarios. E.g. a comparison with \eqref{eq:posterior} shows, if $g$ is a prior probability distribution, $L^\prime$ becomes a non-normalized posterior distribution that can be used in a Bayesian analysis. A further interpretation is given in section \ref{sec:combinationOfMeasurements}.
\label{sec:combinationOfMeasurements}

KaFit allowed to choose $g$ in \ref{eq:likelihoodExtension} as a product of one-dimensional Gaussian distributions. Within this thesis the software was extended to allow products of other functions. Three function types were explicitly made available through a configuration file.
\begin{enumerate}
	\item A reimplementation of a one-dimensional Gaussian distribution: The reimplementation was necessary to conveniently enable the combination of function types.
	\item A multivariate Gaussian distribution: This enables the treatment of uncertainties quantified by calibration or monitor measurements as described in section \ref{sec:combinationOfMeasurements}. It can also be used as a prior distribution in a Bayesian analysis. Particularly, correlations can be respected.
	\item A one-dimensional probability density, that is constant in the square root of a parameter, if it is positive and 0 otherwise:
	\begin{equation}
		g(\theta) =
		\begin{cases}
		0 &\text{ if } \theta \leq 0 \\
		\text{constant} \cdot \frac{1}{\sqrt{\theta}} &\text{ if } \theta > 0
		\end{cases}
		\fullstop
	\end{equation}
	 This can be used as a uniform prior on the neutrino mass ($\theta=m_\nu^2$). Formerly, it was only possible to use a uniform prior on the squared neutrino mass. A derivation of the form of $g$ can be found in appendix \ref{sec:appStatisticPriorOnNu2}.
\end{enumerate}
An example on how to configure KaFit using the new feature is given in appendix \todo{Add appendix}.
