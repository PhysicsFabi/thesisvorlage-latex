\def\currentRootFolder{chapter/sensitivityStudyWithPreliminaryKatrinElossModel/statisticalPrerequisites}
\def\currentFigureFolder{\currentRootFolder/fig}
\newcommand{\sigmaInel}{\sigma_\mathrm{inel}}
\newcommand{\sigmaAvg}{\sigma_\mathrm{avg}}

\newcommand{\qUmin}{qU_\mathrm{min}}

\newcommand{\Ekin}{E_\mathrm{kin}}
\newcommand{\nSource}{n_\mathrm{S}}
\newacronym{ssc}{SSC}{source and spectrum calculation}


\section{Statistical Prerequisites}
\label{sec:katrinElossStatistics}
This section develops the statistical tools used in the scope of this thesis in order to evaluate the impact of the KATRIN energy loss model on KATRIN's sensitivity to the neutrino mass. The methods are described in a general manner (and could be applied to study model uncertainties in general) and then related to the KATRIN energy loss model. Section~\ref{sec:katrinElossStatisticsCombMeasurements} presents a concept for the combination of a neutrino mass and a calibration measurement in parameter inference. Section~\ref{sec:katrinElossStatisticsProfileLikelihood} explains the profile-likelihood method for the treatment of nuisance parameters. And section~\ref{sec:katrinElossStatisticsAsimov} introduces the idea behind an Asimov data set and how it relates to ensemble testing.

\subsection{Combination of a Calibration and a Neutrino Mass Measurement}
\label{sec:katrinElossStatisticsCombMeasurements}
If two measurements share a set of parameters $\paramVecShared$ and, additionally, have an individual set of parameters $\paramVec_1$ and $\paramVec_2$ and different sets of observations a combined likelihood $L$ is given by the product of the likelihoods $L_1$ and $L_2$ of each measurement~\cite{ReviewOfParticlePhysics}
\newcommand{\paramVecSOne}{\paramVec_\mathrm{s,1}}
\newcommand{\paramVecSTwo}{\paramVec_\mathrm{s,2}}
\begin{align}
-2\ln L(\paramVecShared, \paramVec_1, \paramVec_2) &=  
-2\ln L_1(\paramVecShared, \paramVec_1)
-2\ln L_2(\paramVecShared, \paramVec_2)
\nonumber \\
&\equiv
-2\ln L_1(\paramVecSOne)
-2\ln L_2(\paramVecSTwo)
\label{eq:katrinElossStatisticsCombinedLikelihood}
\comma
\end{align}
where, for ease of notation, the combined parameter vectors $\paramVec_\mathrm{s,1}\equiv(\paramVec_\mathrm{s},\paramVec_1)$ and 
$\paramVec_\mathrm{s,2}\equiv(\paramVec_\mathrm{s},\paramVec_2)$ are introduced. In the scope of this thesis, it makes sense to identify the first measurement with a KATRIN neutrino mass measurement and the second with the measurement of the KATRIN energy loss model. In order to emphasize generality, this identification is postponed until the end of this section and the second measurement is called a calibration measurement throughout this section. Combining the likelihoods of a KATRIN neutrino mass and a calibration measurement incorporates the uncertainties of the latter into neutrino mass inference. 

\paragraph{Approximated Combination of Likelihoods}
For practicality, in this thesis, an approximation is applied: The calibration measurement is treated as evaluated independently (which is the case for the evaluation of the measurement of the KATRIN energy loss model). Hence, there are estimates $\hat{\paramVec}_\mathrm{s,2}$, and an estimated covariance matrix $\hat{V}_\mathrm{s,2}$ for the parameters of the calibration measurement. These can in turn be used to approximate the likelihood $L_2$. A choice that stands to reason for the approximation of $L_2$ is a multivariate normal distribution $\mathcal{N}$. For the purpose of parameter inference through the maximum likelihood method $-2\ln L_2$ needs only to be accurately approximated within the contour, that is needed to extract confidence intervals. The choice of a multivariate normal distribution corresponds a symmetric approximation in second order of $-\ln L_2$ around its minimum. The combined likelihood~\eqref{eq:katrinElossStatisticsCombinedLikelihood} then reads
\begin{equation}
\begin{split}
\label{eq:katrinElossStatisticsPullTerm}
-2\ln L(\paramVecShared, \paramVec_1, \paramVec_2) &\approx
\chi^2(\paramVecSOne) 
-2\ln \mathcal{N}(\paramVecSTwo, \hat{\paramVec}_\mathrm{s,2}, \hat{V}_\mathrm{s,2}^{-1}) +
\mathrm{ constants}\\ &=
\underbrace{
	\chi^2(\paramVecSOne)
	\vphantom{(\paramVecSTwo - \hat{\paramVec}_\mathrm{s,2})^{\mathsf{T}}}
}_{(1)}
+
\underbrace{
	(\paramVec_\mathrm{s,2} - \hat{\paramVec}_\mathrm{s,2})^{\mathsf{T}}
	\hat{V}_\mathrm{s,2}^{-1}
	(\paramVec_\mathrm{s,2} - \hat{\paramVec}_\mathrm{s,2})
}_{(2)} +\; 
\mathrm{ constants}
\end{split}
\end{equation}
Here, $(1)$ is the chi-square likelihood for a KATRIN neutrino mass measurement (see equation~\ref{eq:statMethodsKatrinChi2}). And $(2)$ resembles the negative log-likelihood of the calibration measurement approximated by a multivariate normal distribution. Terms having a form like $(2)$ are also sometimes called ``pull terms'' because in the minimization of the likelihood they ``pull'' the parameters $\paramVec_\mathrm{s,2}$ towards the corresponding values in $\hat{\paramVec}_\mathrm{s,2}$.

\paragraph{Chi-Square Characteristics}
The chi-square term $(1)$ in~\eqref{eq:katrinElossStatisticsPullTerm} is assumed to be a sum of $n$ standard normal distributed random variables as discussed in section~\ref{sec:statMethodsKATRINLikelihood}. Hence, a likelihood only composed of the chi-square term $(1)$ offers a goodness-of-fit criteria via the the Pearson chi-square statistic. Whether the same criteria can be applied to the combined likelihood has to be investigated individually from case to case.

\paragraph{Implementation in the KaFit Software Framework}
The KaFit software module (see section~\ref{sec:statMethodsKaFitSSC}) had allowed to use one-dimensional Gaussian ``pull terms'' (term $(2)$ in equation~\ref{eq:katrinElossStatisticsPullTerm}). In the scope of this thesis the software was extended to allow for arbitrary dimensions with corresponding correlations by using a multivariate normal distribution. Albeit not of particular interest in this chapter, for completeness, the following shall be mentioned: By comparing equations~\eqref{eq:statMethodsPosterior} and~\eqref{eq:katrinElossStatisticsPullTerm} it becomes apparent that such ``pull terms'' take the same mathematical form as Bayesian priors. For that reason, further term forms apart from the multivariate normal distribution were implemented in order to be used as priors in a Bayesian analysis. For a documentation of the software features see appendix~\ref{sec:appendixKatrinElossStatisticsLikelihoodExtKaFitConfig}.

\paragraph{Application to the Energy Loss Model}
With regard to the study presented in this chapter, the following identification can be made: $\paramVec_1$ comprises the parameter of a nominal four-parameter KATRIN neutrino mass fit (see section~\ref{sec:statMethodsStandardFit}). Furthermore, the calibration measurement can be identified with the measurement of the KATRIN energy loss model $\paramVecShared=\nuisanceParamVec_\mathrm{eloss}$, $\paramVec_2=\nuisanceParamVec_\mathrm{eloss+}$, $\hat{\paramVec}_\mathrm{s}=\hat{\nuisanceParamVec}_\mathrm{eloss}$, $\hat{\paramVec}_2=\hat{\nuisanceParamVec}_\mathrm{eloss+}$ (see equations~\ref{eq:katrinElossElossModelParams}~and~\ref{eq:katrinElossElossModelExtendedParams} for $\nuisanceParamVec_\mathrm{eloss}$ and $\nuisanceParamVec_\mathrm{eloss+}$) and the corresponding estimator for the covariance matrix $\hat{V}_\mathrm{s,2}=\hat{V}_\mathrm{eloss,eloss+}$. The numerical values for the three estimators can be found in the form of the means, the standard deviations and the correlation matrix in appendix~\ref{sec:appendixKatrinElossElossModelParams}. In the same manner, the calibration measurement could instead be identified with the measurement of the Aseev model $\paramVecShared=\transp{(A_1, A_2,\epsilon_2,\omega_1,\omega_2)}$ without additional nuisance parameters $\mathrm{dim}\paramVec_2=0$ and a diagonal, estimated covariance matrix $\hat{V}_\mathrm{s,2}\equiv\hat{V}_\mathrm{Aseev}$ (diagonal because there are no published correlations of the parameters of the Aseev model). See equation~\eqref{eq:intSpecModelAseevEloss} for the meaning  of the parameters of the Aseev model and see the caption of figure~\ref{fig:intSpecModelAseevEloss} for their values.

\subsection{Nuisance Parameters and the Profile-Likelihood Method}
\label{sec:katrinElossStatisticsProfileLikelihood}
\begin{figure}[t]
	\centering
	\includegraphics[width=\textwidth]{\currentFigureFolder/profileLikelihood.pdf}
	\xcaption{Illustration of the profile-likelihood method}{Illustration of the profile-likelihood method.}{The graph illustrates the extraction of a confidence interval from the likelihood in a two-dimensional scenario, where there is only one nuisance parameter~$\pi$ and one parameter of interest~$\theta$. The graph is a contour plot of an exemplary two-dimensional likelihood of Gaussian shape with a correlation between $\pi$ and $\theta$ of $\rho=0.8$. The contour encloses a 1-$\sigma$-confidence region as per equation~\eqref{eq:statMethodsConfidenceContour}. Its width in the dimension of~$\pi$ is indicated as ``model''-uncertainty with reference to an uncertainty stemming from a model established by a calibration measurement as described in section~\ref{sec:katrinElossStatisticsCombMeasurements}. The width in the dimension of~$\theta$ stems from the statistical uncertainty. The dashed line contains the points~$\left(\theta, \hat{\hat{\pi}}(\theta)\right)$ as per equation~\eqref{eq:katrinElossStatisticsProfileLikelihoodRatio}. It always intersects the contour at the point furthest right and left in the dimension of $\theta$ independently of $\rho$, $\sigma_\mathrm{stat}$ and $\sigma_\mathrm{model}$. The point of intersection determines $\sigma_\mathrm{tot}$. For example MINOS of the ROOT software framework is an algorithm that numerically tries to find the intersection of the dashed line and the contour~\cite{James1998}. Under the conditions stated in the main text, the interval of width $2\cdot\sigma_\mathrm{tot}$ on the $\theta$ axis around the maximum likelihood estimator is per construction a confidence \mbox{interval (\SI{68}{\percent} C.L.)} for the true value of $\theta$. A feature that can intuitively be deduced from the graph is the following: No matter how much $\sigma_\mathrm{model}$ is reduced, the total uncertainty $\sigma_\mathrm{tot}$ can never shrink below $\sigma_\mathrm{stat}$. Likewise, whether a longer measurement, that decreases $\sigma_\mathrm{stat}$, can improve $\sigma_\mathrm{tot}$ depends on $\sigma_\mathrm{model}$ and the correlation $\rho$.}
	\label{fig:katrinElossStatisticsProfileLikelihood}
\end{figure}
Apart from the  parameters of interest $\paramVec$, the KATRIN likelihood can depend on further nuisance parameters $\nuisanceParamVec$. With regard to the study presented in this chapter, the parameter of interest is the squared neutrino mass and the nuisance parameters are the further three parameters of a nominal four-parameter KATRIN neutrino mass fit as well as the parameters of the energy loss model (KATRIN or Aseev). The dimensionality (15+4 parameters in the case of the KATRIN model in a nominal KATRIN neutrino mass fit) may cause long run times when deriving a confidence region for the combined parameter set. Furthermore, as indicated by the naming conventions, the dimensions of the nuisance parameters in the confidence region are not of interest. Hence, in order to construct a confidence interval with just one dimension, a test statistic, similar to the one for the full parameter set as in equation~\ref{eq:statMethodsLikelihoodRatio}, but solely depending on the parameters of interest, has to be found. The following paragraph outlines how a corresponding test statistic can be constructed using the profile-likelihood method.

First, the profile likelihood is defined. It only depends on the parameters of interest $\paramVec$ and is independent of the nuisance parameters $\nuisanceParamVec$. Its values correspond to the likelihood values evaluated at $\paramVec$ in the dimensions of the parameters of interest and maximized in the dimensions of the nuisance parameters~\cite{ReviewOfParticlePhysics}
\begin{equation}
\label{eq:katrinElossStatisticsProfileLikelihood}
\profLikelihood(\paramVec) = 
L(\paramVec, \hat{\hat{\nuisanceParamVec}}(\paramVec))
\comma
\end{equation}
where the double-hat indicates the maximization respectively the ``profiling''. Also, the profile-likelihood ratio can be defined~\cite{ReviewOfParticlePhysics}
\begin{equation}
\label{eq:katrinElossStatisticsProfileLikelihoodRatio}
\lambda_\mathrm{p}(\paramVec) = 
\frac{\profLikelihood(\paramVec)}{\profLikelihood(\hat{\paramVec})}
\fullstop
\end{equation}
According to Wilks’ theorem~\cite{wilks1938}, the distribution of $-2\ln\lambda_\mathrm{p}(\hat{\paramVec})$, where $\hat{\paramVec}$ is the \gls{mle} (see section~\ref{sec:statMethodsMLE}), approaches a chi-square distribution in the limit of a large data sample, independently of the values of the nuisance parameters $\nuisanceParamVec$~\cite{ReviewOfParticlePhysics}. Hence, the profile- likelihood ratio offers a test statistic, based on which, values for the parameters of interest can be rejected. In other words, the profile-likelihood method is a constructive approach on how to derive a confidence interval. Whether all the conditions are met for its application to be valid can either be verified by a theoretical argument or put to the test. For the latter, many experiments can be simulated. The application of the profile-likelihood method then yields an ensemble of confidence intervals. How many of them cover the simulation truth determines the confidence level of the confidence interval. This approach is chosen in the scope of this thesis.

Figure~\ref{fig:katrinElossStatisticsProfileLikelihood} illustrates the profile-likelihood method for the case where $\paramVec$ and $\nuisanceParamVec$ are one-dimensional.

\subsection{Ensemble Tests in Relation to an Asimov Data Set in Sensitivity Studies}
\label{sec:katrinElossStatisticsAsimov}
If one were to repeat the KATRIN experiment many times, one would obtain an ensemble of confidence intervals for the neutrino mass (see section~\ref{sec:statMethodsUncertaintyIntervalsConfidence} about confidence intervals and also see table~\ref{tab:statMethodsSensitivityFromEnsembleTests} with sensitivity studies from former works where the statistical portion of the sensitivity is listed with an uncertainty, which implies a distribution of statistical uncertainties). KATRIN's sensitivity can be deduced from a confidence interval (see section~\ref{sec:statMethodsSensitivtyDef} about KATRIN's sensitivity). In that sense, if many KATRIN measurements are simulated, one also obtains an ensemble respectively a distribution of sensitivities. One way to obtain this distribution is to simulate many KATRIN neutrino mass measurements and fluctuate the measured electron counts according to Poissonian statistics (or Gaussian statistics as an approximation). The expectation value for KATRIN's sensitivity can than be extracted from the obtained distribution. This approach was for example applied within the scope of the KATRIN Design Report~\cite{Angrik:2005ep}. It should be noted that such a distribution of sensitivities is narrow compared to its expectation value (see for example table~\ref{tab:statMethodsSensitivityFromEnsembleTests} where the spread of the statistical uncertainty on the squared neutrino mass is smaller than the mean value by three orders of magnitude).

Instead of simulating many experiments, the median of KATRIN's sensitivity might be obtained from one simulation. Simulating a KATRIN neutrino mass measurement can be time-consuming depending on its level of detail. In that regard, using only one simulation is more practical. The median for KATRIN's sensitivity might be retrieved from one simulation, by replacing the electron count rates with their expectation value instead of fluctuating them according to Poissonian statistics~\cite{Cowan2011}. This reasoning is based on Walt's theorem~\cite{Wald1944}. Such a simulated data set is called an Asimov\footnote{The name of the Asimov data set is inspired by the short story ``Franchise'', by Isaac Asimov. In the story, elections are held by selecting the single most representative voter to replace the entire electorate~\cite{Cowan2011}.} data set. However, whether Walt's theorem is applicable may be hard to verify without doing an ensemble test, which would nullify its practicality here.

In the scope of this chapter, both approaches, using an Asimov data set and simulating an ensemble, were applied. The outcome verifies that an Asimov data set would be representative in the study presented in this chapter and indeed yields the median sensitivity.