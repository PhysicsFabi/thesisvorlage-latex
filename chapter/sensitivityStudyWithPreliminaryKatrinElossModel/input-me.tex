\def\currentRootFolder{chapter/sensitivityStudyWithPreliminaryKatrinElossModel}
\def\currentFigureFolder{\currentRootFolder/fig}

\chapter{Statistical Methods in the Context of the Empirical Energy Loss Model Derived from KATRIN Data for Electrons Scattering Inelastically off Deuterium}
\label{sec:katrinEloss}
A quantitative accurate description of the scattering processes of $\upbeta$ electrons within KATRIN's gaseous tritium source is of crucial importance for KATRIN's sensitivity goal. In modeling the corresponding effects, the energy loss function (see section~\ref{sec:intSpecModelResponseEloss}) plays an important role. The KATRIN Design Report states that the precision of the energy loss functions from literature is not sufficient for KATRIN's envisaged sensitivity. It was planned to deduce a sufficiently accurate model from data taken at KATRIN~\cite{Angrik:2005ep}. In that regard, a preliminary model has successfully been established for electrons scattering off deuterium molecules based on data taken in October 2018 by a dedicated subgroup of the KATRIN collaboration. In the following, this model is referred to as the ``KATRIN model'' or the ``KATRIN energy loss model''. This preliminary energy loss model has partially improved uncertainties with respect to the model by~\cite{Aseev2000}, that is presented in the previous section~\ref{sec:intSpecModelResponseEloss}, and which was used in many previous works~\cite{Groh2015,Kleesiek2014, Kleesiek2019, SeitzM2019}. In the following, the latter model is referred to as the ``Aseev model'' after the primary author of the corresponding publication~\cite{Aseev2000}. In addition to the improved uncertainties, the KATRIN model also exhibits features not present in the Aseev model but motivated by data (see section~\ref{sec:katrinElossModel} and figure~\ref{fig:katrinElossElossModel}). 

As the uncertainties associated with the energy loss function influence KATRIN's sensitivity to the neutrino mass, it is of importance to probe the KATRIN energy loss model in neutrino mass inference and provide corresponding feedback to the team that measures it. Beyond that, the implementation of the required statistical tools into the available software frameworks is of general interest as the approach taken in the scope of this thesis may be applied to other model uncertainties, not only to the ones stemming from the energy loss function. 

This chapter is structured as follows: Section~\ref{sec:katrinElossModel} outlines the KATRIN energy loss model. Section~\ref{sec:katrinElossValidity} discusses the scope of the validity of this study; for example what is expected of the comparison of a model for electrons scattering off deuterium and another model for scattering off tritium molecules. Section~\ref{sec:katrinElossStatistics} introduces statistical tools for uncertainty treatment within neutrino mass inference. Special emphasis is put on the use of additional free fit parameters to mitigate systematic effects. These methods find application in section~\ref{sec:katrinElossModelResults} in order to evaluate systematic effects as induced by the energy loss model. And section~\ref{sec:katrinElossModelOutlook} concludes and offers an outlook.

\section{The Empirical KATRIN Energy Loss Model}
\label{sec:katrinElossModel}
\newcommand{\ionEnergyDeu}{E_\mathrm{ion,D_2}}%
\begin{figure}[h!]
	\centering
	\includegraphics[width=\textwidth]{\currentFigureFolder/KATRINandAseevElossModel}
	\xcaption{The preliminary empirical KATRIN energy loss model for electrons scattering off deuterium molecules}{The preliminary empirical KATRIN energy loss model for electrons scattering off deuterium molecules.}{The black line shows the KATRIN energy loss model as established by a dedicated subgroup of the KATRIN collaboration for electrons scattering off deuterium molecules (KATRIN model). The corresponding model for scattering off tritium molecules as established by~\cite{Aseev2000} (Aseev model) is shown for comparison as a shaded area. (That the latter is plotted as an area instead of as a line solely serves readability as the two functions overlap strongly.) The top panel shows the probability densities of the energy loss. The transition from the excitation peak to the ionization tail is marked by the ionization energy of deuterium at $\ionEnergyDeu=\SI{15.467}{eV}$. The bottom panel shows the absolute symmetric 1-$\sigma$ uncertainties of the models. The uncertainties were obtained through uncertainty propagation via derivatives from the uncertainties of the model parameters (see figure~\ref{fig:intSpecModelAseevEloss} for the parameter values of the Aseev model and appendix~\ref{sec:appendixKatrinElossElossModelParams} for the KATRIN model). Correlations are respected for the KATRIN model. However, there are no published correlations for the Aseev model. The KATRIN model shows particularly improved uncertainties in the ionization tail region in comparison to the Aseev model. (KATRIN model adapted from~\cite{Hannen2019_1}.)}
	\label{fig:katrinElossElossModel}
\end{figure}
The KATRIN energy loss model was established by a dedicated subgroup of the KATRIN collaboration. It is partly of empirical nature and partly based on first principles. The model was fitted to data taken at the KATRIN experiment in Fall 2018. A description and a parametrization is given in the following. Section~\ref{sec:katrinElossModelMainParametrization} describes the KATRIN energy model itself and section~\ref{sec:katrinElossModelNuisanceParametrization} describes the fit it was obtained from because there are further fit parameters that can not necessarily be neglected in the uncertainty treatment. It should be noted, that a unified notation for the KATRIN energy loss model has not yet been established and that the notation in this thesis was chosen without following any reference in particular.

\subsection{Description and Parametrization of the KATRIN Energy Loss Model}
\label{sec:katrinElossModelMainParametrization}
In the preceding section~\ref{sec:intSpecModelResponseEloss}, the energy loss function $f_1(\epsilon)$ is introduced. It denotes the probability density for an energy $\epsilon$ that an electron loses when scattering once (inelastically in the current context). The KATRIN model is such an energy loss function.

The KATRIN model distinguishes two types of energy losses when an electron scatters inelastically off a gas molecule~\cite{Hannen2019_2}. The first type is the excitation of electron states within the target molecule for impact energies below the ionization energy of deuterium $\ionEnergyDeu=\SI{15.467}{eV}$~\cite{Shiner1993}. The second type is the ionization of the target molecule for impact energies above $\ionEnergyDeu$. The most probable energy losses are among the ones caused by the excitation of electronic states. In other words, the KATRIN model has a peak around these energies between approximately $\SI{12}{eV}$ and $\ionEnergyDeu$. In the following, this peak is denoted as the ``excitation peak''. Also, the part of the KATRIN model for energies above $\ionEnergyDeu$ is denoted ``ionization tail''.

The excitation peak is empirically modeled by a sum of three scaled Gaussian distributions $\mathcal{N}$. This finds inspiration in the three contributions to the vibrational molecular hydrogen states by the Lyman band, the Werner band and higher terms~\cite{Hannen2019_1} (also see~\cite{Geiger1964}). The excitation peak region $f_\mathrm{exit}^\mathrm{emp}$ is described by nine parameters $\nuisanceParamVec_\mathrm{eloss}$ which comprise the three scales $A_i$, means $m_i$ and standard deviations $s_i$ ($i \in \{1,2,3\}$) of the three Gaussian distributions~\cite{Hannen2019_2}
\newcommand{\katrinElossPhen}[1]{
		f_\mathrm{exit}^\mathrm{emp}\giventhat*{#1}{\nuisanceParamVec_\mathrm{eloss}}
}
\begin{align}
\label{eq:katrinElossElossModelParams}
\nuisanceParamVec_\mathrm{eloss} &= 
\transp{\left(
	A_1, m_1, s_1, 
	A_2, m_2, s_2, 
	A_3, m_3, s_3
	\right)}\comma \\
\katrinElossPhen{\epsilon} &=
\sum_{i=1}^3 A_i \cdot \mathcal{N}(\epsilon, \mu=m_i, \sigma=s_i)
\fullstop
\end{align}
The second part, the ionization tail, follows a modified version of the binary-encounter-dipole (BED) model. For the mathematical expression, the reader is referred to~\cite{Kim1994}. Here, this part will be denoted
\newcommand{\katrinElossBDE}[1]{
		f_\mathrm{ion}^\mathrm{BED}(#1)
}%
\begin{equation}
	\katrinElossBDE{\epsilon}
	\fullstop
\end{equation}%
The BED model is valid for scattering off hydrogen molecules. Therefore, it depends on the ionization energy of hydrogen. The corresponding value was exchanged for the one of deuterium $\ionEnergyDeu$. Whether further modifications with regard to the difference in isotopologues are necessary is currently under investigation. Furthermore, the constant normalization factor of the BED model was removed for the following reason: The transition between the two parts, the excitation peak and the ionization tail, is introduced at the ionization energy of deuterium $\ionEnergyDeu$ and in order for the transition to be continuous, a scaling factor for the ionization tail is included
\newcommand{\katrinElossScaleFactor}{
	S(\nuisanceParamVec_\mathrm{eloss})
}%
\begin{equation}
\katrinElossScaleFactor = \left.	
		\katrinElossPhen{\ionEnergyDeu}
\middle/
		\katrinElossBDE{\ionEnergyDeu}
\right.
\fullstop
\end{equation}
Then, the full parametrization for the KATRIN model reads~\cite{Hannen2019_1}
\begin{equation}
	f_1^\mathrm{KATRIN}
	\giventhat*{\epsilon}{\nuisanceParamVec_\mathrm{eloss}} = 
	\begin{cases}
	0
	&\text{ if } \epsilon < 0 \\
	\katrinElossPhen{\epsilon}
	&\text{ if } 0 \leq \epsilon \leq E_\mathrm{ion,D_2} \\
	\katrinElossScaleFactor \cdot \katrinElossBDE{\epsilon} 
	&\text{ if } \epsilon > E_\mathrm{ion,D_2}
	\end{cases}
	\fullstop
\end{equation}
Figure~\ref{fig:katrinElossElossModel} shows the KATRIN model in comparison to the Aseev model. They are not expected to be fully compatible within their uncertainties as they describe the scattering off two different hydrogen isotopologues. For a comparison of the presented KATRIN model with an energy loss model for deuterium from literature, the reader is referred to~\cite{Rodenbeck2019}. Two of the three Gaussian distributions in the excitation peak region overlap in a way, that they are not distinguishable in the given plot scale. However, one Gaussian forms a second visible, smaller peak in the excitation region which is a feature that is not present in the Aseev model. As the KATRIN model is still in its early stages, this work refrains from a detailed physical interpretation and instead focuses on uncertainty propagation. For a more detailed physical interpretation, the reader is referred to the KATRIN documents~\cite{Rodenbeck2019,Hannen2019_1,Hannen2019_2}. With respect to the uncertainties, the KATRIN model shows an improved uncertainty in the ionization tail and also in large parts of the excitation peak compared to the Aseev model. How this propagates within neutrino mass inference is investigated in this chapter.

\subsection{Nuisance Parameters that Further Influence the KATRIN Energy Loss Model}
\label{sec:katrinElossModelNuisanceParametrization}
The uncertainties of the KATRIN model as evaluated at the time of writing this thesis can be divided into two sets: The KATRIN model itself comprises nine parameters (eq.~\ref{eq:katrinElossElossModelParams}). Moreover, the model was obtained in a 15-parameter fit. The other six fit parameters are correlated with the parameters of the KATRIN model and hence are not necessarily negligible with regard to uncertainty treatment. In the scope of this thesis, they were incorporated in the statistical treatment of the uncertainties. Whether they may be negligible or can be treated in a simpler fashion in future studies needs further investigation. 

For a the full fitting procedure of the KATRIN model to the recorded data, the reader is referred to~\cite{Hannen2019_1}, but a description of the further six fit parameters is given in the following. In order to outline their meaning, an overview of the used data sets is required. Four integral spectra were recorded analogously to the described integral $\upbeta$ spectrum in chapter~\ref{sec:intSpecModel}, but with the electron gun (see section~\ref{sec:katrinExpSetupRearSection}) as electron source instead of tritium. For each data set a different deuterium column density (see section~\ref{sec:katrinExpSetupWGTS}) was set in the \gls{wgts}: \SI{0}{\percent}, \SI{15}{\percent}, \SI{50}{\percent}, \SI{100}{\percent} of the nominal column density of $\rho d = \SI{5e17}{molecules/cm^2}$. The fit model for the \mbox{\SI{15}{\percent}-measurement} was rescaled with respect to the other measurements because the respective data set underwent a different preprocessing than the others. The corresponding scaling factor $N_{\mathrm{int},15}$ was a free fit parameter. A further scaling factor $N_\mathrm{K}$ for the KATRIN model was introduced in the fit as a free parameter in order for the whole KATRIN model to be normalized to unity. Additionally, a fifth data set recorded at $\SI{15}{\percent}$ column density in a time-of-flight mode (see~\cite{Bonn1999}) was fitted simultaneously. For each data set, a different expected scattering count $
\mu_{\mathrm{tof},15}, 
\mu_{\mathrm{int},15},  
\mu_{\mathrm{int},50},  
\mu_{\mathrm{int},100}$ (see eq.~\ref{eq:intSpecModelExpectedScatteringCount}) was fitted, which adds four further parameters (there is no scattering for the $\SI{0}{\percent}$ measurement). In summary, the additional parameters in the fit of the KATRIN model are
\begin{equation}
\label{eq:katrinElossElossModelExtendedParams}
	\nuisanceParamVec_\mathrm{eloss+} = 
	\transp{\left(
		N_\mathrm{K},
		\mu_{\mathrm{tof},15},
		\mu_{\mathrm{int},15}, 
		\mu_{\mathrm{int},50}, 
		\mu_{\mathrm{int},100},
		N_{\mathrm{int},15}
		\right)}
	\fullstop
\end{equation}
The \gls{mle}s, the estimated standard deviations and the estimated correlation matrix of the full 15-parameter set ($\nuisanceParamVec_\mathrm{eloss}$ and $\nuisanceParamVec_\mathrm{eloss+}$) can be found in appendix~\ref{sec:appendixKatrinElossElossModelParams}. The aim of this chapter is to study the impact of the estimated uncertainties from these 15 parameters on KATRIN's sensitivity to the neutrino mass.

\section{Scope of the Conducted Study}
\label{sec:katrinElossValidity}
This section lists considerations that should be kept in mind with regard to the study presented in this chapter.

During a neutrino mass measurement, the nominal tritium purity in the~\gls{wgts} is planed to be above~\SI{95}{\percent}~\cite{Angrik:2005ep}. In this chapter, a sensitivity study is presented that uses an energy loss model for scattering off deuterium molecules. Within the scope of this thesis, the treatment of the uncertainties of the KATRIN energy loss model in neutrino mass inference is of primary interest. Two different, but similar energy loss models are expected for the two different gas species. As long as the KATRIN model for scattering off deuterium molecules is sufficiently similar to a model for scattering off tritium molecules, it is a plausible approach to use it for simulations of KATRIN neutrino mass measurements (for the similarity, see figure~\ref{fig:katrinElossElossModel}). Also, it is plausible to assume that the parameters for the KATRIN energy loss model for scattering off tritium molecules can be measured with a similar precision to the precision for scattering of deuterium molecules. In that regard, the results for KATRIN's sensitivity derived in this chapter should yield practical indications on what to expect, once the KATRIN energy loss model is available in its final version and for scattering off tritium molecules.

Additionally, it should be noted that the KATRIN model as presented is still preliminary and may be subject to change because the analysis of recent measurements is still ongoing. Furthermore, the KATRIN model does not yet incorporate all known systematic uncertainties. Corresponding efforts for their incorporation are made at the time of writing this thesis. In that regard, a future version of the KATRIN model may exhibit larger uncertainties than the version used in this thesis. But at the same time a continuous modeling effort and an improved understanding of the data may also cause reduced uncertainties. Nonetheless, once the KATRIN model is completed, the study presented in this chapter is expected to be easily repeatable because the necessary statistical tools are now readily implemented.
\FloatBarrier
\def\currentRootFolder{chapter/sensitivityStudyWithPreliminaryKatrinElossModel/statisticalPrerequisites}
\def\currentFigureFolder{\currentRootFolder/fig}
\newcommand{\sigmaInel}{\sigma_\mathrm{inel}}
\newcommand{\sigmaAvg}{\sigma_\mathrm{avg}}

\newcommand{\qUmin}{qU_\mathrm{min}}

\newcommand{\Ekin}{E_\mathrm{kin}}
\newcommand{\nSource}{n_\mathrm{S}}
\newacronym{ssc}{SSC}{source and spectrum calculation}


\section{Statistical Prerequisites}
\subsection{Sensitivity from  the Profile Likelihood Method}
In the scope of this thesis, the sensitivity on the neutrino mass is evaluated using the profile likelihood method (section~\ref{sec:statMethodsProfileLikelihood}) in order to account for nuisance parameters and especially their correlations. The profile likelihood method was also used in~\cite{Kleesiek2014} for a KATRIN standard 4-parameter fit (section~\ref{sec:statMethodsProfileLikelihood}). The obtained value for $\statUncert$ is plotted and can be extracted to be between \SIrange[range-phrase=--]{0.0155}{0.0165}{eV^2} which is in agreement with the result $\statUncert=\SI{0.0162}{eV^2}$ from ensemble testing (see table~\ref{tab:statMethodsSensitivityFromEnsembleTests}). ``In case of the standard 4 parameter fit, the confidence intervals calculated from ensemble tests are in very good agreement with alternative methods [...], including likelihood ratio intervals (profile likelihood method)'' Kleesiek, page 160, profile likelihood, optimized mtd, uncertainty on suqred mnu $0.01494 eV^2$.



\subsection{Combination of Commissioning and Neutrino Mass Measurements}
If two measurements share a set of parameters $\paramVecShared$, but have additionally an individual set of parameters $\paramVec_1$ and $\paramVec_2$ and different sets of observations a combined likelihood is given by the product of the single likelihoods $L_1$ and $L_2$
\begin{equation}
-2\ln L(\paramVecShared, \paramVec_1, \paramVec_2) =  
-2\ln L_1(\paramVecShared, \paramVec_1)
-2\ln L_2(\paramVecShared, \paramVec_2)
\fullstop
\end{equation}
In the case of KATRIN the first measurement could be sensitive to the neutrino mass whereas say the second measurement could have been a calibration using the electron gun and be sensitive to parameters of the response function \eqref{eq:SSCresponse}. Combining both likelihoods would incorporate the uncertainties on the parameters of the response function in the neutrino mass determination. Currently, no software framework exists that allows the construction of combined likelihoods of KATRIN neutrino mass and calibration measurements. Instead the following approximation can be made. The calibration measurement is evaluated independently and one obtains estimates $\hat{\paramVec}_\mathrm{s,2}$, and an estimated covariance matrix $\hat{V}_\mathrm{s,2}$ for all components of $\paramVecShared$ that the calibration measurement is sensitive to. These can in turn be used to approximate the likelihood $L_2$ at least in the dimension of $\paramVecShared$. A choice that stands to reason for the approximation of $L_2$ is a multivariate Gaussian distribution. For the purpose of parameter inference through minimization $-2\ln L_2$ needs only to be accurately approximated around its minimum. The choice of a multivariate Gaussian distribution corresponds a symmetric approximation of $-\ln L_2$ around its minimum by a parabola. The KATRIN likelihood for a combination of a neutrino mass and a calibration measurement then reads
\begin{equation}
\begin{split}
\label{eq:penalizedLikelihood}
-2\ln L(\paramVecShared, \paramVec_1, \paramVec_2) &\approx
-2\ln L^\prime(\paramVecShared, \paramVec_1) \\ &=
\underbrace{
	\chi^2(\paramVecShared, \paramVec_1)
	\vphantom{(\paramVecShared - \hat{\paramVec}_\mathrm{s,2})^{\mathsf{T}}}
}_{(1)}
+
\underbrace{
	(\paramVecShared - \hat{\paramVec}_\mathrm{s,2})^{\mathsf{T}}
	\hat{V}_\mathrm{s,2}^{-1}
	(\paramVecShared - \hat{\paramVec}_\mathrm{s,2})
}_{(2)} +\; 
\mathrm{ constants}\\ &=
\chi^2(\paramVecShared, \paramVec_1) 
-2\ln \mathcal{N}(\paramVecShared, \hat{\paramVec}_\mathrm{s,2}, \hat{V}_\mathrm{s,2}^{-1}) +
\mathrm{ constants}
\end{split}
\end{equation}
Here, $(1)$ is the chi-square expression \eqref{eq:katrinChi2} where the $\paramVecShared$ and $\paramVec_1$ can be written as one combined parameter vector $\paramVec$ for a neutrino mass measurement. And $(2)$ resembles the negative log likelihood of the calibration measurement approximated by a multivariate Gaussian distribution. Terms having a form like $(2)$ are also sometimes called ``pull terms'' or ``likelihood penalties''. In the minimization process they ``pull'' the parameters $\paramVecShared$ towards $\hat{\paramVec}_\mathrm{s,2}$ respectively ``penalize''/increase the negative log likelihood if $\paramVecShared$ and $\hat{\paramVec}_\mathrm{s,2}$ differ.


\newcommand{\CombLmax}{-2\ln L(\hat{\paramVec}_\mathrm{s}, \hat{\paramVec}_1)}
The chi-square term $(1)$ is a sum of $n$ standard normal distributed random variables. Hence, as discussed, a likelihood only composed of the chi-square term $(1)$ offers a goodness-of-fit criteria via the the Pearson chi-square statistic. Note that for the combined likelihood this criteria might not hold. Two special cases can be considered where the chi-square characteristics hold approximately: First, the neutrino mass measurement, term $(1)$, is not sensitive to the shared parameters $\d \chi^2(\paramVecShared, \paramVec_1) /\d \paramVecShared \approx 0$. Then the \gls{mle} for the shared parameters will match the \gls{mle} by the calibration measurement $\hat{\paramVec}_\mathrm{s} = \hat{\paramVec}_{\mathrm{s},2}$ and term $(2)$ will be 0. The combined likelihood evaluated at the \gls{mle} $\CombLmax$ then follows a chi-square distribution with $n-\dim\paramVec_1-\dim\paramVecShared$ degrees of freedom. Second, if the neutrino mass measurement is sensitive to some shared parameters $\d \chi^2(\paramVecShared, \paramVec_1) /\d \paramVecShared \neq 0$, then one might argue, that term $(2)$ evaluated at the \gls{mle} $\hat{\paramVec}_\mathrm{s} \neq \hat{\paramVec}_{\mathrm{s},2}$ is a sum of standard normal distributed random variables. If this holds, the combined likelihood evaluated at the \gls{mle} $\CombLmax$ follows a chi-square distribution with $n-\dim\paramVec_1$ degrees of freedom.


For example, a standard KATRIN 3-year neutrino mass measurement is not at all sensitive to parameters of the energy loss function \eqref{eq:nonAveragedResponse}. Hence, adding a corresponding term $(2)$ from a designated energy loss measurement will not influence the chi-square characteristics. However, a standard KATRIN neutrino mass measurement is even after a short measurement time sensitive to the gas column density \eqref{eq:columnDensity}. Adding a corresponding term $(2)$ from (a naturally more sensitive) monitoring measurement would influence the   


\todo{Add plots from ensemble test that proof statements.}

\subsection{Extension of the KaFit Software Framework}
\label{sec:statLikelihoodExtImpl}
The likelihood $L(\paramVec)$ can be multiplied by a function $g(\paramVec)$
\begin{equation}
\label{eq:likelihoodExtension}
-2\ln L^\prime(\paramVec) = -2\ln L(\paramVec) -2\ln g(\paramVec)
\fullstop
\end{equation}
This procedure may have different interpretations and usage scenarios. E.g. a comparison with \eqref{eq:posterior} shows, if $g$ is a prior probability distribution, $L^\prime$ becomes a non-normalized posterior distribution that can be used in a Bayesian analysis. A further interpretation is given in section \ref{sec:combinationOfMeasurements}.
\label{sec:combinationOfMeasurements}

KaFit allowed to choose $g$ in \ref{eq:likelihoodExtension} as a product of one-dimensional Gaussian distributions. Within this thesis the software was extended to allow products of other functions. Three function types were explicitly made available through a configuration file.
\begin{enumerate}
	\item A reimplementation of a one-dimensional Gaussian distribution: The reimplementation was necessary to conveniently enable the combination of function types.
	\item A multivariate Gaussian distribution: This enables the treatment of uncertainties quantified by calibration or monitor measurements as described in section \ref{sec:combinationOfMeasurements}. It can also be used as a prior distribution in a Bayesian analysis. Particularly, correlations can be respected.
	\item A one-dimensional probability density, that is constant in the square root of a parameter, if it is positive and 0 otherwise:
	\begin{equation}
		g(\theta) =
		\begin{cases}
		0 &\text{ if } \theta \leq 0 \\
		\text{constant} \cdot \frac{1}{\sqrt{\theta}} &\text{ if } \theta > 0
		\end{cases}
		\fullstop
	\end{equation}
	 This can be used as a uniform prior on the neutrino mass ($\theta=m_\nu^2$). Formerly, it was only possible to use a uniform prior on the squared neutrino mass. A derivation of the form of $g$ can be found in appendix \ref{sec:appStatisticPriorOnNu2}.
\end{enumerate}
An example on how to configure KaFit using the new feature is given in appendix \todo{Add appendix}.



\def\currentRootFolder{chapter/sensitivityStudyWithPreliminaryKatrinElossModel}
\def\currentFigureFolder{\currentRootFolder/fig}
\FloatBarrier
\section{Free Fit Parameters of the Energy Loss Model in Neutrino Mass Inference}
\label{sec:katrinElossModelResults}
This section lists the increase of the statisitical uncertainty by additional fit parameters using the Aseev or the KATRIN energy loss model, respectively.  The sensitivity study was conducted using a mathematical model of a KATRIN neutrino mass measurement as described in chapter~\ref{sec:intSpecModel} and the Aseev model as described in section~\ref{sec:intSpecModelResponseEloss} using a nominal configuration based on the KATRIN Design Report~\cite{Angrik:2005ep} with a squared neutrino mass of \SI{0}{eV^2} and a measurement time of three years. Correspondingly, the SSC and KaFit modules were used (see section~\ref{sec:statMethodsKaFitSSC}). The full configuration of the study can be found in appendix~\ref{sec:appendixKatrinElossSSCConfig}. 

The confidence interval for the squared neutrino mass was extracted via the profile-likelihood method as described in section~\ref{sec:katrinElossStatisticsProfileLikelihood}.

The following three cases were investigated using an Asimov data set and the second case also via ensemble testing:\mynobreakpar
\begin{enumerate}
	\item The simulation- and fit-model use the KATRIN energy loss model, but it was assumed to be without uncertainties. In other words, only the four parameters of a nominal KATRIN-neutrino-mass fit (see section~\ref{sec:statMethodsStandardFit}) were treated as free parameters and the parameters of the KATRIN energy loss model were fixed to their best estimates. In the following, this case will be denoted the ``nominal case''.
	\item The simulation- and fit-model use the KATRIN energy loss model, and all its parameters were treated as free in order to incorporate the uncertainties of parameters of the KATRIN model, which results in a 19-parameter fit.
	\item The simulation- and fit-model use the Aseev model, and all its parameters were treated as free in order to incorporate the uncertainties of the parameters of the Aseev model, which results in a nine-parameter fit.
\end{enumerate}

For the last two cases, the parameters of the energy loss models were constrained via ``pull terms'' as described in section~\ref{sec:katrinElossStatisticsCombMeasurements} about the combination of a neutrino mass and a calibration measurement.

Section~\ref{sec:katrinElossModelResultsAsimov} presents the results using an Asimov data set. And section~\ref{sec:katrinElossModelResultsEnsemble} presents the same results as obtained by an ensemble test  for the second of the cases listed above.

\subsection{Sensitivity from an Asimov Data Set}
\label{sec:katrinElossModelResultsAsimov}
\begin{figure}[t]
	\centering
	\includegraphics[width=\textwidth]{\currentFigureFolder/profileLikelihoodKATRINandAseev.pdf}
	\xcaption{Profile-likelihood ratio of a KATRIN measurement from an Asimov data set}{Profile-likelihood ratio of a KATRIN measurement  from an Asimov data set.}{The graph shows the profile-likelihood ratio $\lambda_\mathrm{p}$ incorporating the uncertainties of the following three cases: 1) KATRIN model (dashed line), 2) Aseev model (dotted line) or 3) nominal case (shaded area). For a description of the different cases, the reader is referred to the main text. (That the nominal case is plotted as an area instead of as a line solely serves readability because a line would overlap with the line for the case of the KATRIN model.) The \gls{mle} recovers the true simulated squared neutrino mass of \SI{0}{eV^2} with a corresponding likelihood ratio (and profile likelihood ratio) of 1 ($\Rightarrow-\ln\lambda_\mathrm{p}(\hat{\theta})=0$). The horizontal $s$-$\sigma$ lines are drawn at $s^2/2$ as per equation~\eqref{eq:statMethodsConfidenceContour}. Their intersections with $-\ln\lambda_\mathrm{p}$ mark the confidence intervals for the neutrino mass at \SI{68}{\percent} respectively \SI{90}{\percent} confidence level. The width of the two intervals indeed relate through the factor 1.645 due to the parabolic shape of $-\ln\lambda_\mathrm{p}$. The uncertainties of the KATRIN model hardly have an impact compared to the nominal case because the corresponding profile likelihoods are almost equal (also see table~\ref{tab:katrinElossModelResultsAsimov} for quantitative values).}
	\label{fig:katrinElossResultsProfileLikelihood}
\end{figure}
Figure~\ref{fig:katrinElossResultsProfileLikelihood} shows the profile-likelihood ratios and table~\ref{tab:katrinElossModelResultsAsimov} lists the extracted confidence intervals and obtained sensitivities for the three cases mentioned in the introduction to this section.

The result of the nominal four-parameter KATRIN neutrino mass fit should be comparable to the results of former works listed in table~\ref{tab:statMethodsSensitivityFromEnsembleTests}. Here, a $\sim\SI{4e-4}{eV^2}$ smaller statistical uncertainty on the squared neutrino mass is obtained compared to the results by~\cite{Kleesiek2014, Hoetzel2012}. This is mainly due to the fact, that in the study presented in this chapter a more up-to-date~\cite{Amsbaugh2015} detection efficiency of \SI{95}{\percent} was used, whereas the other works used~\SI{90}{\percent}. 

Also, the tabulated values show, that the extrapolation of the likelihood to negative squared neutrino masses as discussed in section~\ref{sec:statMethodsUncertaintyIntervalsConfidence} yields a slightly asymmetric (order of $\sim10^{-4}$) profile likelihood for the nominal case and the case that treats the parameters of the KATRIN model as free. This might mean, that a perfectly symmetric extrapolation might need fine-tuning of the corresponding parameters on a case-by-case basis.

Under the restrictions listed in section~\ref{sec:katrinElossValidity}, the following conclusions can be drawn:
\begin{itemize}
	\item Treating the parameters of the KATRIN model as free parameters and constraining them as described in section~\ref{sec:katrinElossStatisticsCombMeasurements} about the combination of a calibration and neutrino mass measurement results in a enlarged statistical uncertainty (\SI{68}{\percent} C.L.) for the squared neutrino mass
	\begin{equation}
		\label{eq:katrinElossModelResultsAsimovEnlargedStatUncert}
		\deltaStatUncert(\nuMass^2) = \SI{1.2e-4}{eV^2}
		\fullstop 
	\end{equation} 
	as compared to a nominal four-parameter KATRIN neutrino mass fit. This enlargement is negligible with respect to the KATRIN systematic budget as it would change KATRIN's sensitivity by less than \SI{1}{meV}.
	\item Using the KATRIN model at its current stage yields an improvement of~\SI{10}{meV} in sensitivity to the neutrino mass compared to using the Aseev model.
	\item The logarithm of the profile-likelihood ratio has a parabolic shape, which translates to a Gaussian shape for the likelihood. This justifies the usage of the factor $1.645$ to convert a confidence interval of~\SI{68}{\percent} confidence level into one of~\SI{90}{\percent}.
\end{itemize}

\begin{table}[t]
	\centering
	\xcaption{Neutrino mass confidence intervals and sensitivities obtained from an Asimov data set}{Neutrino mass confidence intervals and sensitivities obtained from an Asimov data set.}{The table lists values that can be extracted from the profile-likelihood ratio depicted in figure~\ref{fig:katrinElossResultsProfileLikelihood} for the three conducted studies denoted in the first column (also refer to the main text for a description of these three cases). The following quantities are reported: the lower bound of the confidence interval (\SI{68}{\percent} C.L.) on the squared neutrino mass $l(\nuMass^2)$, the upper bound $u(\nuMass^2)$, half the width of the interval $\sigma_\mathrm{stat}(\nuMass^2)$ or $\sigma_\mathrm{tot}(\nuMass^2)$ and KATRIN's sensitivity on the neutrino mass as per equation~\eqref{eq:statMethodsSensitivity}. For the calculation of the sensitivity an additional systematic budget of $\SI{0.017}{eV^2}$ was included (see section~\ref{sec:statMethodsSensitivtyFromEnsemble}).}
	\begin{tabular}{lrrrr}
		\toprule
		\makecell[tr]{} &
		\makecell[tr]{$l(\nuMass^2)$ \\ (\SI{e-2}{eV^2})} & 
		\makecell[tr]{$u(\nuMass^2)$ \\ (\SI{e-2}{eV^2})} & 
		\makecell[tr]{} &
		\makecell[tr]{$S_{\nuMass}(\SI{90}{\percent})$ \\ (\SI{}{meV})}  
		\\
		\hline
		1. nominal case \vphantom{\huge B} & -1.586 & 1.592 & 
		$\sigma_\mathrm{stat}(\nuMass^2) = \SI{1.589e-2}{eV^2}$ & 196 \\
		2. KATRIN model & -1.598 & 1.604 & 
		$\sigma_\mathrm{tot}(\nuMass^2) = \SI{1.601e-2}{eV^2}$ & 196 \\
		3. Aseev model & -1.931 & 1.939 & 
		$\sigma_\mathrm{tot}(\nuMass^2) = \SI{1.935e-2}{eV^2}$ & 206 \\
		\bottomrule
	\end{tabular}
	\label{tab:katrinElossModelResultsAsimov}
\end{table}

\subsection{Cross-Check and Extension of the Asimov Data Set via Ensemble Testing}
\label{sec:katrinElossModelResultsEnsemble}
An ensemble of 4046 KATRIN neutrino mass measurements with a true neutrino mass of~\SI{0}{eV} was simulated using the KATRIN energy loss model and incorporating its uncertainties in the same manner as for the Asimov data set described in the last section~\ref{sec:katrinElossStatisticsAsimov}. However, as opposed to the Asimov data set, in the simulation of the ensemble, the detector counts were fluctuated according to Poissonian statistics. Several aspects were investigated as listed below:

\paragraph{Test of Coverage}
The extraction of a confidence interval via the profile-likelihood method as done in the previous section~\ref{sec:katrinElossStatisticsAsimov} for the Asimov data set should per construction yield a coverage probability of \SI{68.2}{\percent}. If all conditions are met for this to hold was tested. In other words, \SI{68.2}{\percent} of the obtained confidence intervals in the conducted ensemble test should cover the true simulated squared neutrino mass of~\SI{0}{eV^2}. The obtained coverage is \SI{67.6}{\percent} as illustrated in figure~\ref{fig:katrinElossResultsCoverage}. The slight undercoverage on the $10^{-3}$ scale may stem from a limited ensemble test size. Furthermore, it must be emphasized that here the likelihood of the measurement of the KATRIN energy loss model is approximated by a multivariate normal distribution and does not fluctuate in the presented ensemble test. For the result to have full validity, the measurement of the KATRIN energy loss model has to be simulated along with the KATRIN neutrino mass measurement. This may be the aim of a future analysis.

\paragraph{Chi-Square Characteristics}
The combined likelihood of a neutrino mass and a calibration measurement evaluated at the \gls{mle} $-2\ln L(\hat{\paramVec})$ might not follow the chi-square statistic as mentioned in section~\ref{sec:katrinElossStatisticsCombMeasurements}. Figure~\ref{fig:katrinElossStatisticsChi2} shows the obtained distribution of $-2\ln L(\hat{\paramVec})$ in the conducted ensemble test. An \gls{mtd} with 41 retarding potentials was used. Hence, there are 41 summands in the likelihood for the KATRIN neutrino mass measurement. In the combined likelihood, there are additional 15 summands to approximate the likelihood of the measurement of the KATRIN energy loss model. In total, there are 19 fit parameters, four for the nominal KATRIN neutrino mass fit (see section~\ref{sec:statMethodsStandardFit}) and 15 in order to incorporate the uncertainties of the KATRIN model. Thus, the hypothesis stands to reason that the obtained distribution follows a chi-square distribution with $41+15-19=37$ degrees of freedom. A corresponding Kolmogorov–Smirnov test yields a $p$-value of $p=\SI{6e-6}{}$. In other words, this hypothesis has to be rejected with a significance of $4\sigma$. Repeating the test for 38 respectively 39 degrees of freedom yields $p=\SI{0.14}{}$ respectively $p=\SI{1e-14}{}$. Hence, a chi-square distribution of 38 degrees of freedom may not be rejected. This is important because it means that the chi-square statistic can not necessarily be used as a measure for goodness-of-fit when incorporating the uncertainties of the KATRIN energy loss model into the neutrino mass inference in the way described in this thesis. Or, at least, one has to be careful about the choice of degrees of freedom. The origin of this effect is that, here, the likelihood of the measurement of the KATRIN energy loss model is approximated by a multivariate normal distribution and hence does not fluctuate in the given ensemble test.

\begin{figure}[th]
	\centering
	\includegraphics[width=\textwidth]{\currentFigureFolder/coverage.pdf}
	\xcaption{Test of coverage for the ensemble of confidence intervals obtained in the sensitivity study using the KARTRIN energy loss model}{Test of coverage for the ensemble of confidence intervals obtained in the sensitivity study using the KARTRIN energy loss model.}{The graph illustrates the coverage probability. An ensemble of 4046 KATRIN neutrino mass measurements with a true squared neutrino mass of~\SI{0}{eV^2} was simulated. For each simulated measurement a confidence interval was constructed using the profile-likelihood method. The obtained confidence intervals were sorted by their lower limit and plotted stacked which yields the gray band. The confidence intervals that cover the true value are depicted in dark gray, while the ones that do not cover the true value are depicted in light gray. In total~\SI{67.8}{\percent} coverage is obtained. In the limit of an infinite ensemble size, a coverage of~\SI{68.2}{\percent} would be expected per construction (also refer to the main text for a limitation of this statement). It should also be noted, that there is little fluctuation in the width of the confidence intervals, which hints at the representative qualities of an Asimov data set because within a }
	\label{fig:katrinElossResultsCoverage}
\end{figure}

\paragraph{Representative Qualities of the Asimov Data Set}
The estimated mean and standard deviation of the distribution of the  confidence intervals (\SI{68}{\percent} C.L.) as obtained by the ensemble test is 
\begin{equation}
	\label{eq:katrinElossModelResultsEnsembleMeanTotUncert}
	\hat{\sigma}_\mathrm{tot}(\nuMass^2)=\SI{1.599\pm0.013e-2}{eV^2}
\end{equation}
in agreement with the one obtained through the Asimov data set in table~\ref{tab:katrinElossModelResultsAsimov}. Furthermore, the median confidence interval
\begin{equation}
	\tilde{\sigma}_\mathrm{tot}(\nuMass^2)=\SI{1.598e-2}{eV^2}
\end{equation}
recovers the one of the Asimov data set on the \SI{e-5}{eV^2} level. This verifies the Asimov data set as representative for the study on the KATRIN energy loss model presented in this chapter.

\begin{figure}[t]
	\centering
	\includegraphics[width=\textwidth]{\currentFigureFolder/katrinElossChi2Distribution.pdf}
	\xcaption{Chi-square distribution for the simulated ensemble of neutrino mass measurements obtained in the sensitivity study using the KATRIN energy loss model}{Chi-square distribution for the simulated ensemble of neutrino mass measurements obtained in the sensitivity study using the KATRIN energy loss model.}{An ensemble of 4046 KATRIN neutrino mass measurements was simulated. The histogram shows the obtained distribution of the likelihood $L$ evaluated at the~\gls{mle}~$\hat{\theta}$. The obtained distribution may follow a chi-square distribution with 38 degrees of freedom. For details the reader is referred to the main text.}
	\label{fig:katrinElossStatisticsChi2}
\end{figure}

\subsection{Reduction of Systematic Effects by Additional Fit Parameters}
The motivation for the inclusion of further free fit parameters is their capability to compensate for systematic shifts. In order to check whether this has a significant effect with regard to the KATRIN energy loss model, the following study was conducted: A KATRIN neutrino mass measurement was simulated using the KATRIN energy loss model as in the previous sections (see appendix~\ref{sec:appendixKatrinElossSSCConfig} for the full configuration). An Asimov data set was used with the electron counts substituted by their expectation values instead of fluctuated according to Poissonian statistics. In the simulation, all but one parameter of the KATRIN energy loss model were taken as the given mean values in appendix~\ref{sec:appendixKatrinElossElossModelParams}. One parameter was displaced from this value by the amount of its 1-$\sigma$-uncertainty (one time negatively and one time positively). The neutrino mass was inferred from the simulated data in order to quantify the systematic shift that such a displacement can induce. Two types of fits were performed:
\begin{enumerate}
	\item The fit parameters of the KATRIN energy loss model were fixed to their means as given in appendix~\ref{sec:appendixKatrinElossElossModelParams}. This causes a systematic shift of the squared neutrino mass because one parameter of the KATRIN energy loss model is displaced in the simulation.
	
	\item The parameters of the KATRIN energy loss model ($\nuisanceParamVec_\mathrm{eloss}$ and $\nuisanceParamVec_\mathrm{eloss+}$ from eq.~\ref{eq:katrinElossElossModelParams} and eq.~\ref{eq:katrinElossElossModelExtendedParams}) are left as free fit parameters constrained by ``pull terms'' as described in section~\ref{sec:katrinElossStatisticsCombMeasurements} about the combination of a calibration and a neutrino mass measurement. This may partly compensate for a systematic shift of the squared neutrino mass.
	\end{enumerate}
This procedure was repeated for each of the nine parameters of the KATRIN energy loss model. In total, 36 shifts of the squared neutrino mass were determined. Two for the positive and negative displacement of a parameter times two for the two fit types listed above times nine repetitions with one for each parameter. Furthermore, the same method was then also applied for the Aseev model (the corresponding parameter means and 1-$\sigma$ uncertainties are given in te caption of figure~\ref{fig:intSpecModelAseevEloss}). The induced shifts on the inferred squared neutrino mass are listed in table~\ref{tab:katrinElossResultsShifts}. 

This study aimed to asses whether further free fit parameters can compensate for systematic shifts in neutrino mass inference with regard to the energy loss model. Indeed a reduction of the systematic effects could be achieved in all cases listed in table~\ref{tab:katrinElossResultsShifts}. For the KATRIN model, the reduction of systematic shifts is of the order of $\SI{e-4}{eV^2}$. This effect is too small to be significant because the systematic budget for the energy loss model is $\SI{6e-3}{eV^2}$ according to the KATRIN Design Report~\cite{Angrik:2005ep}. For the Aseev model, a significant reduction on the order $\SI{e-3}{eV^2}$ could be achieved for the parameters $\omega_1$ and $A_1$. That the additional fit parameters have a greater effect for the Aseev model than for the KATRIN model might be caused by the different dimensionality of the models (5 additional free fit parameters for the Aseev model and 15 additional free fit parameters for the KATRIN energy loss model). A higher dimensionality leaves more degrees of freedom when maximizing the likelihood. The effects of the difference of the fit model and the (simulated) data might therefore be distributed over more parameters. This might allow for finding a parameter configuration that fits the data without reducing the shift of the squared neutrino mass. Also, these effects have to be related to the increase of the statistical uncertainty as per table~\ref{tab:katrinElossModelResultsAsimov}. When using the parameters of the Aseev model as free fit parameters, the increase in the statistical uncertainty of $\Delta\statUncert=\SI{3.46e-3}{eV^2}$ (see table~\ref{tab:katrinElossModelResultsAsimov}) counterbalances the reduction of the systematic effects.

For the KATRIN model, the greatest shifts are caused by the parameters $s_3$ and $s_2$ on a scale that is slightly greater than the allocated systematic budget of $\SI{6e-3}{eV^2}$. The other shifts are below this threshold. Each shift can be understood as the worst-case scenario on its own because the correlations of the parameters make it unlikely that two or more parameters are displaced strongly at the same time. For example $s_3$ and $s_2$ are anti-correlated with a correlation coefficient of $-0.61$ (see appendix~\ref{sec:appendixKatrinElossElossModelParams}) and hence the corresponding systematic effects are likely to compensate each other. For the Aseev model, the greatest shift is caused by the parameter $\omega_1$, which is also above the systematic budget of $\SI{6e-3}{eV^2}$. No correlations are published for the parameters of the Aseev model, which makes it difficult to decide whether it is likely that effects cancel out or intensify each other. That cancellation of effects are of importance can also be seen in figure~\ref{fig:katrinElossElossModel} that compares the uncertainty band of both energy loss models and here the KATRIN model has significantly smaller uncertainties in the ionization tail. A future study should consider to include the correlations for the KATRIN model when studying the induced systematic shifts. In a rough approximation, one can look at the average of the 18 absolute shifts in table~\ref{tab:katrinElossResultsShifts} for the KATRIN energy loss model, which yields 
\begin{equation*}
	\mean{\Delta m_{\nu,\mathrm{fix}}^2} = \SI{4.2e-3}{eV^2}
\end{equation*}
for the case of fixed energy loss parameters and 
\begin{equation*}
	\mean{\Delta m_{\nu,\mathrm{free}}^2} = \SI{4.1e-3}{eV^2}
\end{equation*}
for the case of free and constrained parameters. Both values are within the systematic budget of $\SI{6e-3}{eV^2}$. Given the increased statistical uncertainty by $\deltaStatUncert(\nuMass^2) = \SI{0.1e-3}{eV^2}$ (eq.~\ref{eq:katrinElossModelResultsAsimovEnlargedStatUncert}) in the case of free parameters, the improvement of the systematic shift is completely compensated. 

It should be noted, that the presented study only investigates the uncertainties of the energy loss models, but does not make further statements about their applicability. In other words, there may be reasons to transit to the KATRIN model beyond the improvement of the sensitivity on simulated data sets. For example, the inclusion of the second peak as depicted in figure~\ref{fig:katrinElossElossModel} in the energy loss function may prove to be more correct when used with real data.

\begin{table}[t]
	\centering
	\xcaption{Systematic shifts of the squared neutrino mass induced by offsets of the parameters of the energy loss model}{Systematic shifts of the squared neutrino mass induced by offsets of the parameters of the energy loss model.}{The left column lists all parameters of the energy loss model. Each one was displaced from its mean value by its 1-$\sigma$-uncertainty in a separate simulation. The difference of the inferred squared neutrino mass from the simulation truth of \SI{0}{eV^2} is given in the two central columns. The sub columns show the shift for a negatively (left) or a positively (right) displaced parameter, respectively. The study was done once for fixed parameters of the energy loss model and once having them as free but constrained fit parameters as described in the main text. The last column shows by which quantity the free fit parameters can compensate for the systematic shift (the difference was calculated before rounding). All shifts became smaller as expected. The upper half of the table shows the results for the KATRIN energy loss model and the lower half for the Aseev model.}
	\begin{tabular}{c| rr| rr|| rr}
\toprule
\makecell[tl]{parameter} &
\multicolumn{2}{c|}{\makecell[tc]{
		fixed fit parameters \\ $\Delta\nuMass^2$ (\SI{e-3}{eV^2})
}} &
\multicolumn{2}{c||}{\makecell[tc]{
		constrained fit parameters \\ $\Delta\nuMass^2$ (\SI{e-3}{eV^2}) 
}} &
\multicolumn{2}{c}{\makecell[tc]{
		difference  \\ $\Delta\abs{\Delta\nuMass^2}$ (\SI{e-3}{eV^2})
}}
\\
\hline
\multicolumn{7}{c}{KATRIN energy loss model} \\
\hline
$A_1$ & 3.3 & -3.4 & 3.2 & -3.3& 0.1 & 0.1 \\
$m_1$ & -0.2 & 0.1 & 0.0 & 0.1& 0.2 & 0.0 \\
$s_1$ & 4.1 & -4.2 & 4.1 & -4.1& 0.1 & 0.1 \\
$A_2$ & 5.4 & -5.4 & 5.3 & -5.3& 0.1 & 0.1 \\
$m_2$ & -0.8 & 0.8 & -0.8 & 0.8& 0.0 & 0.0 \\
$s_2$ & 9.3 & -9.3 & 9.1 & -9.1& 0.2 & 0.2 \\
$A_3$ & 3.8 & -3.9 & 3.5 & -3.7& 0.3 & 0.1 \\
$m_3$ & -0.5 & 0.5 & -0.4 & 0.5& 0.1 & 0.0 \\
$s_3$ & 10.2 & -10.1 & 9.9 & -9.9& 0.4 & 0.2 \\
\hline
\multicolumn{7}{c}{Aseev energy loss model} \\
\hline
$A_1$ & 4.1 & -4.1 & 1.4 & -1.4 & 2.7 & 2.7 \\
$A_2$ & 1.1 & -1.1 & 0.2 & -0.2 & 0.9 & 0.9 \\
$\omega_1$ & 8.0 & -8.0 & 3.0 & -3.0 & 5.0 & 5.0 \\
$\omega_2$ & 0.4 & -0.4 & 0.0 & 0.0 & 0.4 & 0.4 \\
$\epsilon_2$ & 0.3 & -0.3 & 0.0 & 0.0 & 0.3 & 0.3 \\
\bottomrule
	\end{tabular}
	\label{tab:katrinElossResultsShifts}
\end{table}
\FloatBarrier
\section{Conclusion}
\label{sec:katrinElossModelOutlook}
A general statistical framework was developed that enables the the treatment of model uncertainties by additional free fit parameters. Leaving the parameters of the KATRIN energy loss model as free in order to reduce systematic effects would increase the statistical uncertainty on the squared neutrino mass by $\Delta \sigma_\mathrm{tot}(\nuMass^2) = \SI{1.2e-4}{eV^2}$, which is negligible given the KATRIN systematic budget. Additionally, the reduction of systematic effects is on the order of $\SI{e-4}{eV^2}$, which is not significant given the systematic budget on the order of $\SI{e-3}{eV^2}$ for the energy loss function.

Aside from systematic effects, further conclusions could be drawn from the statistical studies: It was shown, that Asimov data sets of the presented kind may reasonably be assumed to be representative for an ensemble of neutrino mass measurements. Also, it was demonstrated, that precautions have to be taken when using the chi-square statistic as a measure for goodness-of-fit at the same time as incorporating ``pull terms''. And it was confirmed that the profile-likelihood method yields a confidence interval of the expected coverage of \SI{68}{\percent} in the presented study.