\def\currentRootFolder{chapter/sensitivityStudyWithPreliminaryKatrinElossModel}
\def\currentFigureFolder{\currentRootFolder/fig}

\chapter{Impact on KATRIN's Sensitivity by an Energy Loss Model for Inelastically-Scattering Electrons Derived from KATRIN Data}
\label{sec:katrinEloss}
A quantitative accurate description of the scattering processes of $\upbeta$ electrons within KATRIN's gaseous tritium source is of crucial importance for the neutrino-mass-sensitivity goal. In modeling the corresponding effects the energy loss function as described in section~\ref{sec:intSpecModelResponseEloss} plays an important role. With reference to the described energy loss function, the KATRIN Design Report states, that its precision is not sufficient for the KATRIN-sensitivity goal. It was planned to deduce a sufficiently accurate model from data taken at KATRIN~\cite{Angrik:2005ep}. In that regard, a preliminary model has successfully been established for electrons scattering from deuterium molecules based on data taken in October 2018 by a dedicated subgroup of the KATRIN collaboration. This preliminary energy loss model, from here forth called the ``KATRIN model'', has improved uncertainties with respect to the model from literature (section~\ref{sec:intSpecModelResponseEloss}), from here forth called the ``Aseev model'' after the primary author of the corresponding publication~\cite{Aseev2000}. Within the scope of this thesis the impact on KATRIN's sensitivity by the exchange of the Aseev model for the KATRIN model was studied. Therefore, this chapter is structured as follows: Section~\ref{sec:katrinElossConcept} presents the general idea of this study. Section~\ref{sec:katrinElossModel} outlines the KATRIN model. Section~\ref{sec:katrinElossValidity} discusses the scope of the validity of this study, for example what is expected of the comparison of a model for electrons scattering off deuterium and another model for scattering off tritium. Section~\ref{sec:katrinElossStatistics} introduces the necessary statistical tools. Section~\ref{sec:katrinElossModelResults} lists and discusses the results. And section~\ref{sec:katrinElossModelConclusion} concludes and offers an outlook.

\section{Introduction and Motivation}
\label{sec:katrinElossConcept}
For the conducted sensitivity study KATRIN neutrino mass measurements were simulated assuming a neutrino mass of \SI{0}{eV} and using the KATRIN model. A confidence interval was deduced using the profile likelihood method as described in a subsequent section~\ref{sec:katrinElossStatisticsProfileLikelihood}. This enables the treatment of the model uncertainties and their correlations as nuisance parameters. The implementation of the required statistical tools into the available software frameworks is of general interest as the approach taken in the scope of this thesis may be applied to other model uncertainties, not only the ones stemming from the energy loss function.

\section{The Empirical KATRIN Energy Loss Model}
\label{sec:katrinElossModel}
\begin{figure}[h!]
	\centering
	\includegraphics[width=\textwidth]{\currentFigureFolder/KATRINandAseevElossModel}
	\xcaption{The preliminary model for the energy loss of electrons scattering from deuterium molecules established at KATRIN}{The preliminary model for the energy loss of electrons scattering from deuterium molecules established at KATRIN.}{The black line shows the energy loss model established by a dedicated subgroup of the KATRIN collaboration for scattering off deuterium molecules (KATRIN model). The corresponding model for scattering off tritium molecules as established by~\cite{Aseev2000} (Aseev model) is shown for comparison as a shaded area. (That the latter is plotted as area instead of as a line solely serves readability as the two functions overlap strongly.) The top panel shows the probability densities of the energy loss and the bottom panel shows the corresponding absolute symmetric 1-$\sigma$ uncertainties. The uncertainties were obtained through uncertainty propagation via derivatives from the uncertainties on the model parameters (see figure~\ref{fig:intSpecModelAseevEloss} for the Aseev model and appendix~\ref{sec:appendixKatrinElossElossModelParams} for the KATRIN model). Correlations are respected for the KATRIN model. However, there are no published correlations for the Aseev model.}
	\label{fig:katrinElossElossModel}
\end{figure}
The presented KATIRIN energy loss model is a phenomenological description fitted to data taken at the KATRIN experiment in October 2018. It was established by a dedicated subgroup of the KATRIN collaboration. The model is outlined in the following with an emphasis on its statistical properties.

\subsection{Description}
Figure~\ref{fig:katrinElossElossModel} shows the KATRIN model in comparison to the Aseev model. They are not expected to be fully compatible within their uncertainties as they describe the scattering off two different hydrogen isotopologues. For a comparison of the KATRIN model for deuterium with an energy loss model for deuterium from literature, the reader is referred to~\cite{Rodenbeck2019}. Furthermore, the parametrization of the KATRIN model comprises a second peak for excitation and molecular dissociation of deuterium. As the model is still in its early stages this work refrains from a detailed physical interpretation and instead focuses on the statistical features of the KATRIN model. For a more detailed physical interpretation the reader is referred to the KATRIN documents~\cite{Rodenbeck2019,Hannen2019_1,Hannen2019_2}. With respect to the statistical properties, the KATRIN model shows an improved uncertainty in the ionization tail and also in large parts of the excitation peak.

\subsection{Parametrization}
In section~\ref{sec:intSpecModelResponseEloss}, the energy loss function $f_1(\epsilon)$ was introduced. It denotes the probability density for an energy $\epsilon$ that an electron loses when scattering once. The KATRIN model is such an energy loss function. It can be divided into two parts. The first part is the phenomenological description of the excitation peak region by the sum of three scaled Gaussian distributions $\mathcal{N}_i$ ($i \in \{1,2,3\}$). This part can be described by 9 parameters which comprise the scales $A_i$, means $m_i$ and standard deviations $s_i$ of the three Gaussian distributions~\cite{Hannen2019_2}
\newcommand{\katrinElossPhen}[1]{
	\frac{
		\d\sigma_\mathrm{exit}^\mathrm{phen}\giventhat*{#1}{\nuisanceParamVec_\mathrm{eloss}}
	}{
		\d \epsilon
	}
}
\begin{align}
\label{eq:katrinElossElossModelParams}
\nuisanceParamVec_\mathrm{eloss} &= 
\transp{\left(
	A_1, m_1, s_1, 
	A_2, m_2, s_2, 
	A_3, m_3, s_3
	\right)} \\
\katrinElossPhen{\epsilon} &=
\sum_{i=1}^3 A_i \cdot \mathcal{N}_i(\epsilon, \mu=m_i, \sigma=s_i)
\fullstop
\end{align}
The second part, the ionization tail, follows the binary-encounter-dipole (BED) model. For a full formula, the reader is referred to~\cite{Kim1994}. Here, this part will be denoted
\newcommand{\katrinElossBDE}[1]{
	\frac{
		\d \sigma_\mathrm{ion}^\mathrm{BDE}(#1)
	}{
		\d \epsilon
	}
}
\begin{equation}
	\katrinElossBDE{\epsilon}
	\fullstop
\end{equation}%
\newcommand{\ionEnergyDeu}{E_\mathrm{ion,D_2}}%
The transition between the two parts is introduced at the ionization energy of deuterium $\ionEnergyDeu=\SI{15.467}{eV}$~\cite{Shiner1993}. In order for the transition to be continuous a scaling factor for the ionization tail is introduced
\begin{equation}
c = \left.	
	\left(
		\katrinElossPhen{\ionEnergyDeu}
	\right)
\middle/
	\left(
		\katrinElossBDE{\ionEnergyDeu}
	\right)
\right.
\fullstop
\end{equation}
Then, the full parametrization for the KATRIN model reads~\cite{Hannen2019_1}
\begin{equation}
	f_1^\mathrm{KATRIN}
	\giventhat*{\epsilon}{\nuisanceParamVec_\mathrm{eloss}} = 
	\begin{cases}
	\katrinElossPhen{\epsilon}
	&\text{ if } \epsilon \leq E_\mathrm{ion,D_2} \\
	\vphantom{.} & \\
	c \cdot \katrinElossBDE{\epsilon} 
	&\text{ if } \epsilon > E_\mathrm{ion,D_2}
	\end{cases}
	\fullstop
\end{equation}

\subsection{Uncertainties}
The uncertainties of the KATRIN model as evaluated at the time of writing this thesis can be divided into two sets: The KATRIN model itself comprises nine parameters. However, it was obtained in a 15-parameter fit. The other six fit parameters are correlated with the parameters of the KATRIN model and hence are not necessarily negligible with regard to uncertainties. In the scope of this thesis they were incorporated in the statistical treatment of the uncertainties. Whether they may be neglected in future studies needs further investigation. 

For a detailed description of the full fit of the KATRIN model to the recorded data, the reader is referred to~\cite{Hannen2019_1}, but a short description is given in the following. In order to describe the further fit parameters a short overview of the used data sets is required. Four integral spectra were recorded analogously to the described integral $\upbeta$ spectrum in chapter~\ref{sec:intSpecModel}, but with the electron gun (see section~\ref{sec:katrinExpSetupRearSection}) as source instead of tritium. For each data set a different deuterium column density (see section~\ref{sec:katrinExpSetupWGTS}) was set in the \gls{wgts}: \SI{0}{\percent}, \SI{15}{\percent}, \SI{50}{\percent}, \SI{100}{\percent} of the nominal column density of $\rho d = \SI{5e17}{molecules/cm^2}$. The fit model for the \mbox{\SI{15}{\percent}-measurement} was rescaled with respect to the other measurements. The corresponding scaling factor $N_{\mathrm{int},15}$ was a free fit parameter. A further scaling factor $N_\mathrm{K}$ for the KATRIN model was introduced in the fit as a free parameter in order for the whole KATRIN model to keep its properties of a probability density and integrate to unity. Additionally, a 5th data set recorded at $\SI{15}{\percent}$ column density in a time-of-flight mode~\cite{Bonn1999} was fitted simultaneously. For each data set a different expected scattering count $\mu_\mathrm{tof}, \mu_{\mathrm{int},1}, \mu_{\mathrm{int},2}, \mu_{\mathrm{int},3}$ (see equation~\ref{eq:intSpecModelExpectedScatteringCount}) was fitted, which adds four further parameters (there is no scattering for the $\SI{0}{\percent}$ measurement). In summary, the additional parameters in the fit of the KATRIN model are
\begin{equation}
\label{eq:katrinElossElossModelExtendedParams}
	\nuisanceParamVec_\mathrm{eloss+} = 
	\transp{\left(
		N_\mathrm{K},
		\mu_{\mathrm{tof},15},
		\mu_{\mathrm{int},15}, 
		\mu_{\mathrm{int},50}, 
		\mu_{\mathrm{int},100},
		N_{\mathrm{int},15}
		\right)}
	\fullstop
\end{equation}
The best best-fit values and the covariance matrix of the full 15-parameter fit ($\nuisanceParamVec_\mathrm{eloss}$ and $\nuisanceParamVec_\mathrm{eloss+}$) can be found in appendix~\ref{sec:appendixKatrinElossElossModelParams}. The aim of this chapter is to study the impact of the model uncertainties from these 15 parameters on KATRIN's sensitivity to the neutrino mass.

\section{Scope of Validity}
\label{sec:katrinElossValidity}
This section lists considerations that should be kept in mind with regard to the study presented in this chapter.

First, it should be noted that the KATRIN model as presented is still preliminary and may be subject to change.

Furthermore, during a neutrino mass measurement, the nominal tritium purity in the~\gls{wgts} is approximately~\SI{95}{\percent}. Here, an energy loss model for deuterium is used. Two different, but similar energy loss models are expected for the two different gas species (compare~\cite{Abdurashitov2017} and~\cite{Aseev2000}). Within the scope of this thesis, the impact on KATRIN's sensitivity from the reduction of uncertainties on the energy loss model are of primary interest. As long as the KATRIN model for scattering off deuterium molecules is sufficiently similar to the model for scattering off tritium molecules, it is a plausible approach to use it for simulations of KATRIN neutrino mass measurements. (For the similarity, see figure~\ref{fig:katrinElossElossModel}.) In that regard, the results for KATRIN's sensitivity derived in this chapter are expected to be meaningful.

It should also be noted that the covariances used in this thesis for the parameter sets $\nuisanceParamVec_\mathrm{eloss}$ and $\nuisanceParamVec_\mathrm{eloss+}$ (see equations~\ref{eq:katrinElossElossModelParams} and~\ref{eq:katrinElossElossModelExtendedParams}) were estimated using the MIGRAD algorithm of the ROOT software framework~\cite{Hannen2019_1}. A cross-check by a more reliable estimation of the covariances, for example by means of scanning the likelihood with the MINOS algorithm was not yet done at the time of writing this thesis.

Furthermore, the KATRIN model does not yet incorporate systematic uncertainties. Corresponding efforts for their incorporation are made at the time of writing this thesis. A future version of the KATRIN model might exhibit larger uncertainties than the version used in this thesis.

\def\currentRootFolder{chapter/sensitivityStudyWithPreliminaryKatrinElossModel/statisticalPrerequisites}
\def\currentFigureFolder{\currentRootFolder/fig}
\newcommand{\sigmaInel}{\sigma_\mathrm{inel}}
\newcommand{\sigmaAvg}{\sigma_\mathrm{avg}}

\newcommand{\qUmin}{qU_\mathrm{min}}

\newcommand{\Ekin}{E_\mathrm{kin}}
\newcommand{\nSource}{n_\mathrm{S}}
\newacronym{ssc}{SSC}{source and spectrum calculation}


\section{Statistical Prerequisites}
\subsection{Sensitivity from  the Profile Likelihood Method}
In the scope of this thesis, the sensitivity on the neutrino mass is evaluated using the profile likelihood method (section~\ref{sec:statMethodsProfileLikelihood}) in order to account for nuisance parameters and especially their correlations. The profile likelihood method was also used in~\cite{Kleesiek2014} for a KATRIN standard 4-parameter fit (section~\ref{sec:statMethodsProfileLikelihood}). The obtained value for $\statUncert$ is plotted and can be extracted to be between \SIrange[range-phrase=--]{0.0155}{0.0165}{eV^2} which is in agreement with the result $\statUncert=\SI{0.0162}{eV^2}$ from ensemble testing (see table~\ref{tab:statMethodsSensitivityFromEnsembleTests}). ``In case of the standard 4 parameter fit, the confidence intervals calculated from ensemble tests are in very good agreement with alternative methods [...], including likelihood ratio intervals (profile likelihood method)'' Kleesiek, page 160, profile likelihood, optimized mtd, uncertainty on suqred mnu $0.01494 eV^2$.



\subsection{Combination of Commissioning and Neutrino Mass Measurements}
If two measurements share a set of parameters $\paramVecShared$, but have additionally an individual set of parameters $\paramVec_1$ and $\paramVec_2$ and different sets of observations a combined likelihood is given by the product of the single likelihoods $L_1$ and $L_2$
\begin{equation}
-2\ln L(\paramVecShared, \paramVec_1, \paramVec_2) =  
-2\ln L_1(\paramVecShared, \paramVec_1)
-2\ln L_2(\paramVecShared, \paramVec_2)
\fullstop
\end{equation}
In the case of KATRIN the first measurement could be sensitive to the neutrino mass whereas say the second measurement could have been a calibration using the electron gun and be sensitive to parameters of the response function \eqref{eq:SSCresponse}. Combining both likelihoods would incorporate the uncertainties on the parameters of the response function in the neutrino mass determination. Currently, no software framework exists that allows the construction of combined likelihoods of KATRIN neutrino mass and calibration measurements. Instead the following approximation can be made. The calibration measurement is evaluated independently and one obtains estimates $\hat{\paramVec}_\mathrm{s,2}$, and an estimated covariance matrix $\hat{V}_\mathrm{s,2}$ for all components of $\paramVecShared$ that the calibration measurement is sensitive to. These can in turn be used to approximate the likelihood $L_2$ at least in the dimension of $\paramVecShared$. A choice that stands to reason for the approximation of $L_2$ is a multivariate Gaussian distribution. For the purpose of parameter inference through minimization $-2\ln L_2$ needs only to be accurately approximated around its minimum. The choice of a multivariate Gaussian distribution corresponds a symmetric approximation of $-\ln L_2$ around its minimum by a parabola. The KATRIN likelihood for a combination of a neutrino mass and a calibration measurement then reads
\begin{equation}
\begin{split}
\label{eq:penalizedLikelihood}
-2\ln L(\paramVecShared, \paramVec_1, \paramVec_2) &\approx
-2\ln L^\prime(\paramVecShared, \paramVec_1) \\ &=
\underbrace{
	\chi^2(\paramVecShared, \paramVec_1)
	\vphantom{(\paramVecShared - \hat{\paramVec}_\mathrm{s,2})^{\mathsf{T}}}
}_{(1)}
+
\underbrace{
	(\paramVecShared - \hat{\paramVec}_\mathrm{s,2})^{\mathsf{T}}
	\hat{V}_\mathrm{s,2}^{-1}
	(\paramVecShared - \hat{\paramVec}_\mathrm{s,2})
}_{(2)} +\; 
\mathrm{ constants}\\ &=
\chi^2(\paramVecShared, \paramVec_1) 
-2\ln \mathcal{N}(\paramVecShared, \hat{\paramVec}_\mathrm{s,2}, \hat{V}_\mathrm{s,2}^{-1}) +
\mathrm{ constants}
\end{split}
\end{equation}
Here, $(1)$ is the chi-square expression \eqref{eq:katrinChi2} where the $\paramVecShared$ and $\paramVec_1$ can be written as one combined parameter vector $\paramVec$ for a neutrino mass measurement. And $(2)$ resembles the negative log likelihood of the calibration measurement approximated by a multivariate Gaussian distribution. Terms having a form like $(2)$ are also sometimes called ``pull terms'' or ``likelihood penalties''. In the minimization process they ``pull'' the parameters $\paramVecShared$ towards $\hat{\paramVec}_\mathrm{s,2}$ respectively ``penalize''/increase the negative log likelihood if $\paramVecShared$ and $\hat{\paramVec}_\mathrm{s,2}$ differ.


\newcommand{\CombLmax}{-2\ln L(\hat{\paramVec}_\mathrm{s}, \hat{\paramVec}_1)}
The chi-square term $(1)$ is a sum of $n$ standard normal distributed random variables. Hence, as discussed, a likelihood only composed of the chi-square term $(1)$ offers a goodness-of-fit criteria via the the Pearson chi-square statistic. Note that for the combined likelihood this criteria might not hold. Two special cases can be considered where the chi-square characteristics hold approximately: First, the neutrino mass measurement, term $(1)$, is not sensitive to the shared parameters $\d \chi^2(\paramVecShared, \paramVec_1) /\d \paramVecShared \approx 0$. Then the \gls{mle} for the shared parameters will match the \gls{mle} by the calibration measurement $\hat{\paramVec}_\mathrm{s} = \hat{\paramVec}_{\mathrm{s},2}$ and term $(2)$ will be 0. The combined likelihood evaluated at the \gls{mle} $\CombLmax$ then follows a chi-square distribution with $n-\dim\paramVec_1-\dim\paramVecShared$ degrees of freedom. Second, if the neutrino mass measurement is sensitive to some shared parameters $\d \chi^2(\paramVecShared, \paramVec_1) /\d \paramVecShared \neq 0$, then one might argue, that term $(2)$ evaluated at the \gls{mle} $\hat{\paramVec}_\mathrm{s} \neq \hat{\paramVec}_{\mathrm{s},2}$ is a sum of standard normal distributed random variables. If this holds, the combined likelihood evaluated at the \gls{mle} $\CombLmax$ follows a chi-square distribution with $n-\dim\paramVec_1$ degrees of freedom.


For example, a standard KATRIN 3-year neutrino mass measurement is not at all sensitive to parameters of the energy loss function \eqref{eq:nonAveragedResponse}. Hence, adding a corresponding term $(2)$ from a designated energy loss measurement will not influence the chi-square characteristics. However, a standard KATRIN neutrino mass measurement is even after a short measurement time sensitive to the gas column density \eqref{eq:columnDensity}. Adding a corresponding term $(2)$ from (a naturally more sensitive) monitoring measurement would influence the   


\todo{Add plots from ensemble test that proof statements.}

\subsection{Extension of the KaFit Software Framework}
\label{sec:statLikelihoodExtImpl}
The likelihood $L(\paramVec)$ can be multiplied by a function $g(\paramVec)$
\begin{equation}
\label{eq:likelihoodExtension}
-2\ln L^\prime(\paramVec) = -2\ln L(\paramVec) -2\ln g(\paramVec)
\fullstop
\end{equation}
This procedure may have different interpretations and usage scenarios. E.g. a comparison with \eqref{eq:posterior} shows, if $g$ is a prior probability distribution, $L^\prime$ becomes a non-normalized posterior distribution that can be used in a Bayesian analysis. A further interpretation is given in section \ref{sec:combinationOfMeasurements}.
\label{sec:combinationOfMeasurements}

KaFit allowed to choose $g$ in \ref{eq:likelihoodExtension} as a product of one-dimensional Gaussian distributions. Within this thesis the software was extended to allow products of other functions. Three function types were explicitly made available through a configuration file.
\begin{enumerate}
	\item A reimplementation of a one-dimensional Gaussian distribution: The reimplementation was necessary to conveniently enable the combination of function types.
	\item A multivariate Gaussian distribution: This enables the treatment of uncertainties quantified by calibration or monitor measurements as described in section \ref{sec:combinationOfMeasurements}. It can also be used as a prior distribution in a Bayesian analysis. Particularly, correlations can be respected.
	\item A one-dimensional probability density, that is constant in the square root of a parameter, if it is positive and 0 otherwise:
	\begin{equation}
		g(\theta) =
		\begin{cases}
		0 &\text{ if } \theta \leq 0 \\
		\text{constant} \cdot \frac{1}{\sqrt{\theta}} &\text{ if } \theta > 0
		\end{cases}
		\fullstop
	\end{equation}
	 This can be used as a uniform prior on the neutrino mass ($\theta=m_\nu^2$). Formerly, it was only possible to use a uniform prior on the squared neutrino mass. A derivation of the form of $g$ can be found in appendix \ref{sec:appStatisticPriorOnNu2}.
\end{enumerate}
An example on how to configure KaFit using the new feature is given in appendix \todo{Add appendix}.


\section{A Sensitivity Study using the Recent Preliminary KATRIN Energy Loss Model}
\label{sec:katrinElossModelResults}
\def\currentRootFolder{chapter/sensitivityStudyWithPreliminaryKatrinElossModel}
\def\currentFigureFolder{\currentRootFolder/fig}
\begin{figure}[th]
	\centering
	\includegraphics[width=\textwidth]{\currentFigureFolder/profileLikelihoodKATRINandAseev.pdf}
	\xcaption{}{}{}
	\label{fig:katrinElossResultsProfileLikelihood}
\end{figure}

\begin{figure}[th]
	\centering
	\includegraphics[width=\textwidth]{\currentFigureFolder/coverage.pdf}
	\xcaption{Coverage Test for confidence interval}{}{}
	\label{fig:katrinElossResultsCoverage}
\end{figure}


\section{Conclusion and Outlook}
\label{sec:katrinElossModelConclusion}