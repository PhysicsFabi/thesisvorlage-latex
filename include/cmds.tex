%% -----------------------
%% |    Abbreviations    |
%% -----------------------
\newcommand{\op}[1]{\operatorname{#1}}                 % to write operators that
                                                       % are not predefined;
                                                       % it's just an abbrev.
                                                       % for the long command

\newcommand{\arr}[2]{\begin{array}{#1}#2\end{array}}   % to create arrays. very 
                                                       % useful in math env

\renewcommand{\d}{\ensuremath{\text{d}}}               % as the differential 
                                                       % operator, e.g. 
                                                       % \frac{\d x}{\d t}

\newcommand{\NN}{\mathbb{N}}                           % or change it to
\newcommand{\RR}{\mathbb{R}}                           % \mathbbm and include
\newcommand{\CC}{\mathbb{C}}                           % pkg bbm if you prefer

\newcommand{\pdb}[2]{\frac{\partial #1}{\partial #2}}  % partial derivative


%% -----------------------------------
%% |    Commands and Environments    |
%% -----------------------------------
\newcommand{\margtodo}                                 % used by \todo command
{\marginpar{\textbf{\textcolor{kitcolor}{ToDo}}}{}}
\newcommand{\todo}[1]
{{\textbf{\textcolor{kitcolor}{[\margtodo{}#1]}}}{}}   % for todo-notes inside 
                                                       % the document
\newenvironment{deprecated}                            % for something that you
{\begin{color}{gray}}{\end{color}}                     % want to use no more

\newcommand{\xcaption}[2]{\caption[#1]{\textbf{#1} #2}}% nice caption cmd for
                                                       % short and long descrip.

\newcommand{\xfigure}[5]{\begin{figure}[#1]            % a quick command for
\centering                                             % including graphics 
\includegraphics[scale=#2]{./fig/#3}                   % with all necessary vars
\xcaption{#4}{#5}
\label{fig:#3}
\end{figure}}

\newcommand{\xfigurerot}[5]{\begin{figure}[#1]        % same as above, only
\centering                                            % image is rotated
\includegraphics[angle=270,scale=#2]{./fig/#3}
\xcaption{#4}{#5}
\label{fig:#3}
\end{figure}}

\newcommand{\xtable}[4]{\begin{table}[#1]             % same for tables
\centering
\xcaption{#3}{#4}
\rowcolors{3}{gray!10}{white}
\include{./tab/#2}
\label{tab:#2}
\end{table}}



%% ------------------------------------
%% |    Quantum Mechanics and Math    |
%% ------------------------------------
\newcommand{\ket}[1]{\left|#1\right\rangle}           % \ket{X}  ->  |X>
\newcommand{\bra}[1]{\left\langle#1\right|}           % \bra{X}  ->  <X|
\newcommand{\braket}[2]                               % \braket{X}{Y}  ->  <X|Y>
{\left\langle#1 \middle| #2\right\rangle}
\newcommand{\bratenket}[3]                            % \bratenket{X}{Y}{Z}  ->
{\left\langle#1 \middle|\middle| #2 \middle|\middle|  % <X|Y|Z>
#3\right\rangle}
\newcommand{\anglemean}[1]                            % \anglemean{X}  ->  <X>
{\left\langle #1 \right\rangle}                       % \norm{X}  ->  || X ||
\newcommand{\norm}[1]{\left\lVert#1\right\rVert}

\newcommand{\updownarrows}                            % \ket\updownarrows  ->
{\text{\rotatebox[origin=c]{90}{$\rightleftarrows$}}} % |↑↓> (cmt is utf8!)
\newcommand{\downuparrows}                            % \ket\updownarrows  ->
{\text{\rotatebox[origin=c]{270}{$\rightleftarrows$}}}% |↓↑>
\newcommand{\neswarrows}                              % \ket\neswarrows  ->
{\text{\rotatebox[origin=c]{45}{$\rightleftarrows$}}} % |↗↙>
\newcommand{\swnearrows}                              % \ket\swnearrows  ->
{\text{\rotatebox[origin=c]{225}{$\rightleftarrows$}}}% |↙↗>

\newcommand{\cre}{c^\dagger}                          % annihalation operator
\newcommand{\anh}{c^{\vphantom{\dagger}}}             % creation operator
\newcommand{\numb}{n^{\vphantom{\dagger}}}            % number operator
